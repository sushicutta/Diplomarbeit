  \section{Anwesende Personen}
    
  \begin{table}[ht]
    \sffamily 
    \begin{center}
      \begin{tabular}{rlc}
        \toprule
        \textbf{Funktion} & \textbf{Person} & \textbf{Anwesend}\\
        \midrule
        Student & Roman Würsch & Ja\\
        Auftraggeber & Bernhard Mäder, ZKB & Nein\\
        Projektbetreuer & Beat Seeliger & Ja\\
        Experte & Marco Schaad & Ja\\
        Vertreter Studiengang Informatik & Matthias Bachmann & Ja\\
        \bottomrule
      \end{tabular}
      \captionsetup{list=no}
      \caption{Anwesende Personen}
      \label{tab:anwesendePersonen}
    \end{center}
  \end{table}


  \section{Beschlüsse}
  \begin{itemize}
      \item Vorschlag von Matthias Bachmann: Es soll eine Sensitivitätsanalyse
      für die Gewichtung der Soll-Kriterien angewendet werden. Aufgrund der
      Abgrenzung in der Aufgabenstellung ``\begin{itshape}Umfragen, Erhebungen
      sowie Feldstudien werden nicht durchgeführt.\end{itshape}'' wird das
      nicht gemacht. Zudem arbeitet der Student nicht intern in der Zürcher
      Kantonalbank, was ihn daran hindert eine Gewichtung durch die Mitarbeiter
      der ZKB-Architektur zu machen.
      \item Es sollen folgende Frameworks evaluiert werden:
      \begin{itemize}
        \item ULC, Canoo RIA Suite
        \item Apache Struts 1.3.10 mit ZIP
        \item Vaadin 6.5.7
        \item Apache Wicket 1.4.17
      \end{itemize}
      \item Die Aufgabenstellung soll im EBS angepasst werden. Der Satz
      ``\begin{itshape}Eine Auflistung von etablierten Java Web
      Frameworks\end{itshape}'' soll durch den Satz
      ``\begin{itshape}Ergebnisse der Evaluation von etablierten Java Web
      Frameworks\end{itshape}'' ersetzt werden.
      \item Auftraggeber ist zufrieden, gemäss Abstimmung vom 19.04.2011.
      \item Die Einladung zum Schlusstermin erfolgt durch Roman Würsch.
      \item Gute Kriterien und Analysemethode gefunden.
      \item KO-Kriterien von der ZKB erhalten
      \item Gute Literatur Recherche
      \item Gute Präsentation, Achtung in Zukunft mit Farben (Titel sind nicht
      lesbar gewesen)
      \item Es soll die Thematik ``Skalierung'' der jeweiligen Frameworks
      erwähnt werden. evtl. mit einem Lasttest, falls nicht möglich in der
      Reflektion abgrenzen.
  \end{itemize}