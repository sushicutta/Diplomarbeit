  \section{Anwesende Personen}
    
  \begin{table}[ht]
    \begin{center}
      \begin{tabular}{rlc}
        \toprule
        Funktion & Person & Anwesend\\
        \midrule
        Student & Roman Würsch & Ja\\
        Auftraggeber & Bernhard Mäder, ZKB & Nein\\
        Projektbetreuer & Beat Seeliger & Ja\\
        Experte & --- & Nein\\
        Vertreter Studiengang Informatik & Olaf Stern & Ja\\
        \bottomrule
      \end{tabular}
      \captionsetup{list=no}
      \caption{Anwesende Personen}
      \label{tab:anwesendePersonen}
    \end{center}
  \end{table}

  \section{Beschlüsse}
  \begin{itemize}
      \item Gemäss Email von Herr Stern vom 17. März 2011
      \begin{itemize}
        \item ``Sie führen in Absprache mit Ihrem Betreuer ein inhaltliches
        Kick-Off durch. Gibt Ihr Betreuer sein OK anschliessend, fahren Sie
        mit der Bearbeitung der Arbeit fort (kein Zeitverlust für Sie).''
        \item ``An dem von Ihnen gebuchten Termin am 13. April führen wir
        ein verkürztes Kick-Off durch, dieses wird auch als das formale
        Kick-Off in EBS eingetragen und protokolliert.''
      \end{itemize} 
      \item Gemäss Email von Herr Stern vom 15. März 2011: Es wurden
      die geforderten Änderungen an der Aufgabenstellung angepasst.
      \item Es soll ein RIA - Framework (z.B. ULC) mit in die Evaluation
      genommen werden.
      \item Es soll ein MVC - Framework (z.B. Struts) mit einbezogen
      werden.
      \item Es werden die Bewertungskriterien für die Bachelor Arbeit verwendet.
      \item Der Zeitplan ist straff, sportlich, sollte aber machbar sein.
  \end{itemize}