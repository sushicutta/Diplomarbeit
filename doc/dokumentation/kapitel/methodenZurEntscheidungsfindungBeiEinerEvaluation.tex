  Dieses Kapitel zeigt Methoden zur Entscheidungsfindung bei einer Evaluation
  auf. Dabei werden die Methoden der gewichteten Nutzwertanalyse und des
  \ac{AHP} gezeigt. Es existieren wahrscheinlich noch weitere Ansätze, welche
  hier nicht behandelt werden.
  
  \section{Grundlagen}
  
  Es gibt verschiedene Methoden wie man bei der Auswahl einer Softwarelösung
  vorgehen kann. Um eine möglichst objektive Betrachtung zu gewährleisten, wurde
  die Methode der gewichteten \ac{NWA} gewählt, siehe \cite{Nutzwertanalyse}.
  Diese Methode stammt aus dem Bereich der quantitativen Analysemethoden der
  Eintscheidungstheorie. Um eine möglichst präzise objektive Gewichtung der
  einzelnen Faktoren zu erhalten wurde die Methode \ac{AHP} gewählt, siehe
  \cite{AnalyticHierarchyProcess}. Diese Methode stammt aus dem Bereich der
  präskriptiven Entscheidungstheorie. Der \ac{AHP} wurde von Thomas L. Saaty in
  den 70er Jahren des 20. Jahrhunderts entwickelt und im Buch ``The Analytic
  Hierarchy Process: Planning, Priority Setting, Resource
  Allocation'', siehe \cite{AnalyticHierarchyProcessBook}, veröffentlicht.
  
  \section{Gewichtete Nutzwertanalyse}
  
  Bei der gewichteten \ac{NWA} wird eine Menge von Kandidaten, auf deren
  Nutzen, miteinander verglichen. Der Vergleich wird über \(n\) vergleichbare
  Alternativen geführt. Dabei werden die einzelnen Alternativen mit einem
  Erfüllungsgrad \(e_i\) bewertet. Die Skala der Erfüllungsgrade ist in der
  Tabelle \ref{tab:erfuellungsgrade} ersichtlich.
  \newline
  
  \begin{table}[ht]
    \sffamily 
    \begin{center}
      \begin{tabular}{lc}
        \toprule
        Erfüllungsgrad & Skala\\
        \midrule
        nicht erfüllt & 0\\
        schlecht & 1, 2\\
        mittel & 3 - 5\\
        gut & 6 - 8\\
        sehr gut & 9\\
        \bottomrule
      \end{tabular}
      \caption{Skala der Erfüllungsgrade}
      \label{tab:erfuellungsgrade}
    \end{center}
  \end{table}
  
  Jede Alternative wird durch einen Gewichtungsfaktor \(g_i\) versehen, was
  die Präferenz der Alternative wiederspiegelt. Dabei gilt: Die Gewichte gi werden
  so gewählt, dass ihre Summe 1 (100\%) ergibt, siehe Formel \ref{eq:gewicht}.

  \begin{equation}
    \label{eq:gewicht}
    Gewicht := \sum \limits_{i=1}^n g_i = 1
  \end{equation}

  \newpage

  Der Nutzwert ergibt sich durch die Formel \ref{eq:nutzwert}:

  \begin{equation}
    \label{eq:nutzwert}
    Nutzwert := \sum \limits_{i=1}^n e_i \cdot g_i
  \end{equation}
  
  Für jeden zu evaluierenden Kandidaten soll geprüft werden, ob eines der
  KO-Kriterien erfüllt ist, was zu einem Ausschluss des Kandidaten führen würde.
  Falls das nicht der Fall ist, wird der Nutzwert des Kandidaten berechnet.
  Derjenige Kandidat mit dem grössten Nutzwert entspricht am meisten den
  Anforderungen.
  
  \subsection{Anschauliches Beispiel}
  
  Als anschauliches Beispiel sollen zwei Kandidaten - Auto \(A\) und \(B\) -
  miteinander verglichen werden. Sie sollen auf deren Alternativen - Leistung,
  Aussehen und Alltagstauglichkeit - verglichen werden. In der Tabelle
  \ref{tab:beispielNwa} ist das Beispiel ersichtlich. Das Resultat zeigt, dass
  das Auto \(A\) dem Auto \(B\)  gegenüber bevorzugt werden soll, da der
  Nutzwert \(5.0 > 4.7\) ist. 
  
  \begin{table}[ht]
    \sffamily 
    \begin{center}
      \begin{tabular}{lrrrrr}
        \toprule
        Alternativen & Gewichtung \(g\) & \(e_A\) & Wertigkeit \(A\) & \(e_B\)
        & Wertigkeit \(B\)\\
        \midrule
        Leistung            & 0.3 & 7 & 2.1 & 9 & 2.7 \\
        Aussehen            & 0.2 & 2 & 0.4 & 5 & 1.0 \\
        Alltagstauglichkeit & 0.5 & 5 & 2.5 & 2 & 1.0 \\
        \midrule
        \midrule
        Ergebnis            & 1.0 &   & 5.0 &   & 4.7 \\
        \bottomrule
      \end{tabular}
      \caption{Beispiel einer Nutzwertanalyse}
      \label{tab:beispielNwa}
    \end{center}
  \end{table}
 
  \section{Analytic Hierarchy Process}
  
  Der Analytic Hierarchy Process stamt aus der Feder eines Mathematikers. Aus
  diesem Grund ist das Verfahren auch einiges Anspruchsvoller als eine
  gewichtete Nutzwertanalyse. Ich gehe hier nicht auf die ganzen Details ein, da
  es den Rahmen der Diplomarbeit übersteigen würde. Totzdem soll eine grober
  Überblick über den \ac{AHP} gegeben werden.
  
  Der \ac{AHP} besteht aus drei Phasen, siehe \cite{AnalyticHierarchyProcess}: 
  
  \begin{enumerate}
    \item Sammeln der Daten
    \item Daten vergleichen und gewichten
    \item Daten verarbeiten
  \end{enumerate}
  
  \subsection{Sammeln der Daten}
  
  In der ersten Phase sollen alle Daten, die für eine Entscheidungsfindung
  erheblich sind, gesammelt werden.
  
  \begin{itemize}
    \item Zuerst soll eine konkrete Frage formuliert werden, für welche die
    beste Antwort gesucht wird.
    \item Danach sollen zu der gestellten Frage alle Kriterien  gesucht werden,
    welche die Lösung beeinflussen können.
    \item Als letztes sollen alle Alternativen gesucht werden, welche als
    mögliche Lösung infrage kommen.
  \end{itemize}
  
  \subsection{Daten vergleichen und gewichten}
  
  In der zweiten Phase folgt nun die Gegenüberstellung, Vergleich und Bewertung
  aller Kriterien beziehungsweise Alternativen in zwei Unterschritten.
  
  \begin{itemize}
    \item Jedes Kriterium wird jedem anderen gegenübergestellt und darauf
    verlgichen, was eine grössere Bedeutung in der gestellten Frage hat. Die
    Skala geht von 1 bis 9, siehe Tabelle \ref{tab:vergleichsgrade}.
    \item Für jedes Kriterium wird jede mögliche Alternativen mit jeder anderen
    gegenübergestellt und auf ihre Eignung hin untersuchen, welche Alternative
    am besten zur Erfüllung des jeweiligen Kriteriums passt. Die Skala geht von
    1 bis 9, siehe Tabelle \ref{tab:vergleichsgrade}.
  \end{itemize}
  
  \begin{table}[ht]
    \sffamily 
    \begin{center}
      \begin{tabular}{lc}
        \toprule
        Bedeutung & Skala\\
        \midrule
        gleiche Bedeutung & 1\\
        leicht grössere Bedeutung & 2 - 3\\
        viel grössere Bedeutung & 4 - 6\\
        erheblich grössere Bedeutung & 7 - 8\\
        absolut dominierend & 9\\
        \bottomrule
      \end{tabular}
      \caption{Skala der Vergleichsgrade}
      \label{tab:vergleichsgrade}
    \end{center}
  \end{table}
    
  \subsection{Daten verarbeiten}
  
  Mit einem mathematischen Modell, kann der \ac{AHP} nun eine präzise Gewichtung
  aller Kriterien errechnen. Mit der Gewichtung der Kriterien und dem Vergleich
  der Alternativen, kann der \ac{AHP} nun berechnen, welches die beste Lösung
  (Alternative) für die gestellt Frage ist.
  
  Diese Berechnungen werden meistens mit der Unterstütztung einer Software
  gemacht. Als Beispiel gibt es JAHP 2.1\footnote{Für Hintergrundinformationen
  zum JAHP 2.1, siehe \cite{JAHP}}, dies ist ein Java Programm mit dem der
  gesammte Prozess des \ac{AHP} abgebildet und berechnet werden kann. Das
  Programm wird unter den Bedingungen der GNU General Public License
  vertrieben.
    
  \section{Kombination beider Methoden}
  
  Diese beiden Methoden können auch kombiniert eingesetzt
  werden\cite{AhpNwaKombination}, da der \ac{AHP} relativ komplex in der
  Umsetztung ist. Als Kombination kann die Nutzwertanalyse als Methode
  verwendet werden, und der \ac{AHP} wird für die präzise berechnung der
  Gewichtung einzelner Entscheidungskriterien verwendet. Aus der Kombination
  entsteht somit eine neue Methode, welche durch die Verständlichkeit der
  \ac{NWA} und der objektiven Gewichtung des \ac{AHP} eine plausable Evaluation
  ermöglicht.