  Swing kommt aus dem Hause Oracle, ehemals SUN, und ist ein Bestandteil der
  \ac{JFC}. Seit der Java Version 1.2 ist Swing Bestandteil der \ac{JRE}. Swing
  wurde in den letzten Jahren immer weiter ausgebaut und ist somit der Standard
  für die Entwicklung von Desktop- und Applet-Applikationen in Java.
  
  \section{Grundlagen}
  
  Unter Swing versteht man eine Reihe von leichtgewichtigen Komponenten zur
  Programmierung von grafischen Oberflächen. Mit leichtgewichtigen Komponenten
  meint man, dass alle Komponenten zu 100\% in Java geschrieben sind. Sie sind
  somit plattformübergreifend einsetzbar und sehen überall gleich aus. Die Swing
  Komponenten sind im \ac{JRE} unter dem Packet \(javax.swing\) eingeordnet,
  zusätzlich gibt es das Packet \(javax.accessibility\), welches als
  Abstraktion des User Interfaces dient.
  
  \subsection{Komponenten}
  
  Swing kennt drei Hirarchien von Komponenten: Top-Level-, Intermediate- und
  Atomic Components.
  Unter Atomic Components versteht man einzelne Bausteine wie ein
  \(javax.swing.JButton\), ein einfach Knopf, oder ein
  \(javax.swing.JTextField\), ein einfaches Textfeld.
  Intermediate Components bieten vielfältige Möglichkeiten an, um andere
  Intermediate Components zu unterteilen oder zu gruppieren. Zusätzlichen
  können sie auch eine beliebige Anzahl von Atomic Components enthalten. Gängige
  Vertreter sind \(javax.swing.JPanel\), eine Komponente zur Gruppierung anderer
  Komponenten, oder \(javax.swing.JSplitPane\), eine Komponente zur
  Unterteilung einer Komponente in zwei Teile.
  Die Top-Level Components sind \(javax.swing.JFrame\),
  zur Darstellung einer vollwertigen Fenster-Applikation,
  \(javax.swing.JDialog\), für Dialogfenster und \(javax.swing.JApplet\), zur
  Entwicklung von Java-Applets mit Swing.
  
  \subsection{Layouts}
  
  Viele Komponenten in Swing haben ein Layout, vorallem die Intermediate
  Components. Das Layout regelt die Anordnung der Komponenten und das Verhalten,
  falls sich die Grösse einer Komponente ändert. Swing definiert ein paar eigene
  Layouts, man kann aber auch die bestehenden Layouts aus dem Paket
  \(java.awt\), wie zum Beispiel \(java.awt.GridBagLayout\) verwenden. Swing
  bietet auch die Möglichkeit eigene Layouts zu definieren, was zum Beispiel
  JGoodies mit dem Freeware Projekt \(JGoodies Forms\) gemacht hat, siehe
  \cite{JGoodiesForms}
    
  \subsection{Eventbasierte Kommunikation der Komponenten}
  
  Das Programmiermodell mit Java Swing ist Eventbasiert. Dabei können Events
  definiert werden, z.B. ein Mausklick auf einen Knopf, bei welchen eine Aktion
  ausgeführt werden soll, z.B. das speichern eines Dokuments. Die eventbasierte
  Kommunikation zwischen den Swing Komponenten ist nach dem Observer Design
  Pattern implementiert, siehe \cite{ObserverDesignPattern}.
  
  \subsection{Hilfsmittel}
  
  Die Klasse \(javax.swing.SwingUtilities\) bietet eine vielzahl von
  Hilfsfunktionen, welche bei der Entwicklung von Swing Applikationen verwendet
  werden können. Zusätzlich bietet Swing es eine handvoll weiterer Klassen,
  welche einem das Leben als Programmierer erleichtern. Ein paar nennenswerte
  sind: \(javax.swing.UIManager\), um das aktuelle Look \& Feel zu managen,
  \(javax.swing.BorderFactory\), für das zeichnen von Rahmen um eine
  Komponente, oder auch \(java.swing.SwingWorkder<T,V>\), um asynchrone Tasks
  zu verarbeiten.
  
  \section{Pluggable Look \& Feel}
  
  Das Erscheinungsbild und das Verhalten der Swing Komponenten, auch genannt
  Look \& Feel, kann für alle Swing Komponenten separat definiert werden. Es
  lässt dem Programmiere die Möglichkeit offen, alle Komponenten individuell zu
  gestalten. Durch die Delegation an ein separates Objekt, kann das Look \& Feel
  zur Laufzeit ausgetauscht werden.
  
  \section{Multithreading}
  
  Swing ist zum grössten Teil nicht thread-save. Das heisst, auf ein Swing
  Objekt sollte nur mit einem Thread zugegriffen werden. Für den Zugriff und
  die Instanziirung von Java Swing Objekt steht der \ac{EDT} zur Verfügung,
  welcher alle Events, welche von einer Swing Komponente generiert wurde,
  abarbeitet.
  
  \subsection{Asynchrone Tasks}
  
  Damit der \ac{EDT} bei länger dauernden Tasks nicht blockert wird, stellt
  Swing das Konzept vom SwingWorker zur Verfügung, siehe \cite{SwingWorker}.
  Dabei können Tasks, zum Beispiel eine lang andauernde Berechnung, an einen
  neuen Thread deligiert werden. Die benötigte Klasse heisst: 
  \(javax.swing.SwingWorker<T,V>\).  