In diesem Kapitel sollen die Anforderungen und Akzeptanztests für einen
Prototypen definiert werden. Als Grundlage sollen die Komponenten und
Verhaltensmuster der Applikation Strukti Live 1.2 dienen, da die Applikation
öffentlich verfügbar ist.

Wie im Kapitel \ref{chapter:MethodenZurEntscheidungsfindungBeiEinerEvaluation}
(\nameref{chapter:MethodenZurEntscheidungsfindungBeiEinerEvaluation}, S.
\pageref{chapter:MethodenZurEntscheidungsfindungBeiEinerEvaluation}ff)
beschrieben, gilt der Proof of Concept als erfolgreich, wenn alle
Akzeptanztests erfolgreich abgenommen wurden. Die Abnahme der Akzeptanztests
erfolgt durch eine visuelle Prüfung.

\section{Anforderungen mit User Stories}

\begin{description}
\item[US-1\label{itm:US-1}]
Es soll das Fenster in sechs Teile (Panels) aufgeteilt werden. Die Aufteilung
soll in drei Spalten und zwei Zeilen aufgeteilt werden, wobei die erste Zeile
eine Höhe von 150px\footnote{px ist die Abkürzung für Pixel, zu Deutsch
Bildpunkt. Ein Pixel bezeichnet die kleinste Einheit einer digitalen
Rastergrafik.} und die zweite Zeile eine höhe von 606px haben soll.

\item[US-2\label{itm:US-2}]
Es soll die erste und die dritte Spalte eine Breite von 220px und die zweite
und mittlere Spalte eine Breite von 552px haben.

\item[US-3\label{itm:US-3}]
Es soll der Panel in der ersten Zeile und der ersten Spalte immer ein Logo der
Applikation dargestellt werden. Wenn mit der linken Maustaste auf das Bild
``geklickt'' wird, soll die Startansicht geladen werden.

\item[US-4\label{itm:US-4}]
Es soll der Panel in der ersten Zeile und der zweiten Spalte der Hintergrund
hellblau eingefärbt werden. In der linken unteren Ecke des Panels soll der Titel
der aktuellen Ansicht angezeigt werden.

\item[US-5\label{itm:US-5}]
Es soll der Panel in der zweiten Zeile und der ersten Spalte als
Navigationspanel dienen. Die verschiedenen Ansichten sollen über den
Navigationspanel angesteuert werden können.

\item[US-6\label{itm:US-6}]
Der Navigationspanel soll eine zweistufige Navigation anbieten. Ansichten,
welche Gruppiert werden können, sollen in der zweiten Stufe untergebracht
werden. Navigationen, welche sich in der zweiten Stufe befinden, sollen
unterhalb der übergeordneten Navigationen eingeruckt dargestellt werden.

\item[US-7\label{itm:US-7}]
Die aktuell angesteuerte Ansicht soll im Navigationspanel andersfarbig
dargestellt werden.

\item[US-8\label{itm:US-8}]
Im Navigationspanel sollen die verschiednen Navigationspunkte auf der ersten
Stufe über ein Separator visuell voneinander getrennt werden.

\item[US-9\label{itm:US-9}]
Wenn ein Navigationspunkt der ersten Stufe im Navigationspanel ``angeklickt''
wird, sollen die darunter liegenden zweitstufigen Navigationspunkt sichtbar
werden.

\item[US-10\label{itm:US-10}]
Es sollen zwei Navigationen auf der ersten Stufen angezeigt werden. Die
Navigationen sind ``Einführung'' und ``Produktauswahl''.

\item[US-11\label{itm:US-11}]
Es sollen zwei Navigationen auf der zweiten Stufen unterhalb der Navigation
``Produktauswahl'' angezeigt werden. Die Navigationen sind
``Produktbeschreibung'' und ``Produktdesign''.

\item[US-12\label{itm:US-12}]
Wenn der Navigationspunkt ``Einführung'' ausgewählt wird, soll in der zweiten
Reihe und der zweiten Spalte ein Einführungstext zum Proof of Concept
dargestellt werden.

\item[US-13\label{itm:US-13}]
Wenn der Navigationspunkt ``Produktauswahl'' ausgewählt wird, soll in der
zweiten Reihe und der zweiten Spalte eine Buttonmatrix mit den Produkten
``Put Option'', ``Soft Runner'' und ``Protein'' dargestellt werden. Die
Navigationen ``Produktbeschreibung'' und ``Produktdesign'' sollen angezeigt
werden und deaktiviert sein.

\item[US-14\label{itm:US-14}]
Wenn der Navigationspunkt ``Produktauswahl'' ausgewählt wird, sollen in der
zweiten Zeile und der dritten Spalte Filteroptionen für die Produkte in der Form
von CheckBoxen dargestellt werden.

\item[US-15\label{itm:US-15}]
Wenn eine Filteroption gewählt wird, sollen die Produkte entsprechend des
Filters aktiv oder inaktiv gesetzt werden.

\item[US-16\label{itm:US-16}]
Wenn ein Produkt ausgewählt wird, dann soll in der zweiten Reihe und der zweiten
Spalte die Produktbeschreibung angezeigt werden. Die Produktbeschreibung
besteht aus einem Fliesstext und einem TabbedPanel welcher sechs Tabs hat:
``Eigenschaften'', ``Chancen \& Risiken'', ``Rückzahlungsmodus'',
``Beispiele'', ``Steuern'' und ``Klassifizierung''. Die Navigationspunkte
``Produktbeschreibung'' und ``Produktdesign'' sollen nun aktiv gesetzt werden.

\item[US-17\label{itm:US-17}]
Wenn ein Tab ausgewählt wird, sollen die entsprechenden Informationen zum
Produkt dargestellt werden.

\item[US-18\label{itm:US-18}]
Wenn der Navigationspunkt ``Produktdesign'' ausgewählt wird, soll das
Auszahlungsprofil eines möglichen Produkts dargestellt werden. Die notwendigen
Eckdaten sollen in der zweiten Zeile und der dritten Spalte in der Form von
Slidern dargestellt werden.

\item[US-19\label{itm:US-19}]
Wenn die Werte in den Slidern geändert werden, dann soll sich das
Auszahlungsprofil entsprechend anpassen.
\end{description}

\section{Priorisierung der User Stories}

Es werden alle UserStories als \begin{itshape}hoch\end{itshape} priorisiert.

\section{Akzeptanztests}

Für alle Akzeptanztests gilt als Voraussetzung, dass der Prototyp auf einem
Applikationsserver läuft. Die Akzeptanztest werden mit den Webbrowsern Chrome
und Safari durchgeführt, um sicherzustellen dass der Prototyp unabhängig vom
Webbrowser implementiert wurde.

\begin{description}
\item[T-1.1\label{itm:T-1.1}]
Die vorgegebenen Masseinheiten der Zeilen sollen mit einem Screenshot geprüft werden.

\item[T-2.1\label{itm:T-2.1}]
Die vorgegebenen Masseinheiten der Spalten sollen mit einem Screenshot geprüft werden.

\item[T-3.1\label{itm:T-3.1}]
Es soll geprüft werden, ob das Logo in der ersten Zeile und der ersten Spalte
vorhanden ist.

\item[T-3.2\label{itm:T-3.2}]
Mit einem Klick auf das Logo soll die Startansicht geladen werden.

\item[T-4.1\label{itm:T-4.1}]
Es soll geprüft werden, ob der Panel in der ersten Zeile und der zweiten Spalte
einen hellblau eingefärbten Hintergrund hat.

\item[T-4.2\label{itm:T-4.2}]
Durch das navigieren über verschiedene Menüpunkte, soll der Titel in der ersten
Zeile und der zweiten Spalte, entsprechend der aktuellen Ansicht, angepasst
werden.

\item[T-5.1\label{itm:T-5.1}]
Es soll geprüft werden, ob die Navigation in der zweiten Zeile und der
ersten Spalte ersichtlich ist.

\item[T-5.2\label{itm:T-5.2}]
Durch das navigieren über verschiedene Menüpunkte, soll sich entsprechend die
Ansicht ändern.

\item[T-6.1\label{itm:T-6.1}]
Es soll geprüft werden, ob gruppierte Navigationen unterhalb, der von ihnen
übergeordnenten Navigation, eingeruckt dargestellt werden.

\item[T-7.1\label{itm:T-7.1}]
Es soll geprüft werden, ob die Farbe sich bei einer angesteuerten Navigation
ändert.

\item[T-8.1\label{itm:T-8.1}]
Es soll geprüft werden, ob die Navigationspunkt auf der ersten Stufe über ein
Separator voneinander getrennt sind.

\item[T-8.2\label{itm:T-8.2}]
Wenn ein Navigationspunkt der ersten Stufe angewählt wird, unter der sich
Navigationspunkte auf der zweiten Stufe befinden, soll überprüft werden, dass
kein Separator zwischen der ersten und der zweiten Stufe existiert.

\item[T-9.1\label{itm:T-9.1}]
Wenn ein Navigationspunkt der ersten Stufe angewählt wird, unter der sich
Navigationspunkte auf der zweiten Stufe befinden, soll überprüft werden, ob
diese angezeigt werden.

\item[T-10.1\label{itm:T-10.1}]
Die beiden Navigationspunkte ``Einführung'' und ``Produktauswahl'' sollen auf
der ersten Stufe angezeigt werden.

\item[T-11.1\label{itm:T-11.1}]
Wenn der Navigationspunkt ``Produktauswahl'' angewählt wurde, sollen die beiden
Navigationspunkt ``Produktbeschreibung'' und ``Produktdesign'' auf der zweiten
Stufe angezeigt werden.

\item[T-12.1\label{itm:T-12.1}]
Wenn der Navigationspunkt ``Einführung'' angewählt wurde, soll in der zweiten
Zeile und der zweiten Spalte ein Einführungstext zum Proof of Concept
dargestellt werden.

\item[T-13.1\label{itm:T-13.1}]
Wenn der Navigationspunkt ``Einführung'' angewählt wurde, soll in der zweiten
Reihe und der zweiten Spalte eine Buttonmatrix mit den möglichen Produkten
angezeigt werden.

\item[T-13.2\label{itm:T-13.2}]
Die Navigationspunkte ``Produktbeschreibung'' und ``Produktdesign'' sollen
deaktiviert sein.

\item[T-14.1\label{itm:T-14.1}]
Es soll geprüft werden, ob die Filteroptionen in der zweiten Reihe und der
dritten Spalte dargestellt werden, wenn der Navigationspunkt ``Produktauswahl''
angewählt wurde.

\item[T-14.2\label{itm:T-14.2}]
Die Filteroptionen sollen in der Form von CheckBoxen angezeigt werden.

\item[T-15.1\label{itm:T-15.1}]
Es soll eine Filteroption ausgewählt werden, dabei sollen sich die Produkte,
welche durch den Filter ausgeschlossen werden, deaktiviert werden.

\item[T-15.2\label{itm:T-15.2}]
Es soll eine Filteroption, welche ausgewählt war, wieder abgewählt werden. Dabei
sollen sich die Produkte, welche durch den Filter ausgeschlossen waren, wieder
aktiviert werden.

\item[T-16.1\label{itm:T-16.1}]
Es soll ein Produkt gewählt werden. Wenn das Produkt ausgewählt wurde, soll in
der zweiten Reihe und der zweiten Spalte die Produktbeschreibung angezeigt
werden.

\item[T-16.2\label{itm:T-16.2}]
Die Produktbeschreibung soll aus einem Fliesstext und einem TabbedPanel
bestehen.

\item[T-16.3\label{itm:T-16.3}]
Wenn die Produktbeschreibung angezeigt wird, sollen sechs Tabs angezeigt
werden: ``Eigenschaften'', ``Chancen \& Risiken'', ``Rückzahlungsmodus'',
``Beispiele'', ``Steuern'' und ``Klassifizierung''.

\item[T-16.4\label{itm:T-16.4}]
Wenn ein Produkt ausgewählt wurde, sollen die Navigationspunkte
``Produktbeschreibung'' und ``Produktdesign'' aktiv gesetzt werden.

\item[T-17.1\label{itm:T-17.1}]
Wenn das Produkt ``Put Option'' ausgewählt wird, sollen die Informationen in den
Tabs angezeigt werden.

\item[T-18.1\label{img:T-18.1}]
Wenn das Produkt ``Put Option'' ausgewählt wird, und danach der Navigationspunkt
``Produktdesign'' angewählt wird, soll das Auszahlungsprofil einer Put Option
angezeigt werden.

\item[T-18.2\label{img:T-18.2}]
Wenn der Navigationspunkt ``Produktdesign'' angewählt wird, soll in der zweiten
Zeile und der dritten Spalte der Strike und die Volatilität mit einem Slider
angepasst werden können.

\item[T-19.1\label{img:T-19.1}]
Wenn die Werte der Slider geändert werden, soll das Auszahlungsprofil
entsprechend angepasst werden.

\end{description}

\section{Resultat}

\subsection{Testabdeckung}

\subsection{Screenshots}