In diesem Kapitel sollen die Anforderungen und Akzeptanztests für einen
Prototypen definiert werden. Als Grundlage sollen die Komponenten und
Verhaltensmuster der Applikation Strukti Live 1.2 dienen, da die Applikation
öffentlich verfügbar ist.

\section{Anforderungen mit User Stories}

\begin{description}
\item[US-1\label{itm:US-1}]
Es soll das Fenster in sechs Teile (Panels) aufgeteilt werden. Die Aufteilung
soll in drei Spalten und zwei Zeilen aufgeteilt werden, wobei die erste Zeile
eine Höhe von 150px\footnote{px ist die Abkürzung für Pixel, zu Deutsch
Bildpunkt. Ein Pixel bezeichnet die kleinste Einheit einer digitalen
Rastergrafik.} und die zweite Zeile die verbleibende Anzahl Pixel des
Browserfenster haben soll.

\item[US-2\label{itm:US-2}]
Es soll die erste Spalte eine Breite von 250px und die dritte Spalte eine Breite
von 400px haben. Die zweite und mittlere Spalte soll mindestens eine Breite von
374px haben.

\item[US-3\label{itm:US-3}]
Es soll der Panel in der ersten Zeile und ersten Spalte immer ein Logo der
Applikation dargestellt werden. Wenn mit der linken Maustaste auf das Bild ``geklickt''
wird, soll die Startansicht geladen werden.

\item[US-4\label{itm:US-4}]
Es soll der Panel in der ersten Zeile und der zweiten Spalte der Hintergrund
hellblau eingefärbt werden. In der linken unteren Ecke des Panels soll der Titel
der aktuellen Ansicht angezeigt werden.

\item[US-5\label{itm:US-5}]
Es soll der Panel in der zweiten Zeile und der ersten Spalte als
Navigationspanel dienen. Die verschiedenen Ansichten sollen über den
Navigationspanel angesteuert werden können.

\item[US-6\label{itm:US-6}]
Der Navigationspanel soll eine zweistufige Navigation anbieten. Ansichten,
welche Gruppiert werden können, sollen in der zweiten Stufe untergebracht
werden. Gruppierte Ansichten sollen unterhalb der übergeordneten Ansicht
eingeruckt dargestellt werden.

\item[US-7\label{itm:US-7}]
Die aktuell angesteuerte Ansicht soll im Navigationspanel andersfarbig
dargestellt werden.

\item[US-8\label{itm:US-8}]
Im Navigationspanel sollen die verschiednen Navigationspunkte auf der ersten
Stufe über ein Separator visuell voneinander getrennt werden.

\item[US-9\label{itm:US-9}]
Wenn ein Navigationspunkt der ersten Stufe im Navigationspanel ``angeklickt''
wird, sollen die darunter liegenden zweitstufigen Navigationspunkt sichtbar
werden.
\end{description}

\section{Priorisierung der User Stories}

\section{Akzeptanztests}

\section{Resultat}

\subsection{Testabdeckung}

\subsection{Screenshots}