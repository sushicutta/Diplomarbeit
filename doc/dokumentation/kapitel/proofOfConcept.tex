In diesem Kapitel sollen die Anforderungen und Akzeptanztests für einen
Prototypen definiert werden. Als Grundlage sollen die Komponenten und
Verhaltensmuster der Applikation Strukti Live 1.2 dienen, da die Applikation
öffentlich verfügbar ist.

\section{Anforderungen mit User Stories}

\begin{description}
\item[US-1\label{itm:US-1}]
Es soll das Fenster in sechs Teile (Panels) aufgeteilt werden. Die Aufteilung
soll in drei Spalten und zwei Zeilen aufgeteilt werden, wobei die erste Zeile
eine Höhe von 150px\footnote{px ist die Abkürzung für Pixel, zu Deutsch
Bildpunkt. Ein Pixel bezeichnet die kleinste Einheit einer digitalen
Rastergrafik.} und die zweite Zeile die verbleibende Anzahl Pixel des
Browserfenster haben soll.

\item[US-2\label{itm:US-2}]
Es soll die erste und die dritte Spalte eine Breite von 220px und die zweite
und mittlere Spalte eine Breite von 552px haben.

\item[US-3\label{itm:US-3}]
Es soll der Panel in der ersten Zeile und der ersten Spalte immer ein Logo der
Applikation dargestellt werden. Wenn mit der linken Maustaste auf das Bild
``geklickt'' wird, soll die Startansicht geladen werden.

\item[US-4\label{itm:US-4}]
Es soll der Panel in der ersten Zeile und der zweiten Spalte der Hintergrund
hellblau eingefärbt werden. In der linken unteren Ecke des Panels soll der Titel
der aktuellen Ansicht angezeigt werden.

\item[US-5\label{itm:US-5}]
Es soll der Panel in der zweiten Zeile und der ersten Spalte als
Navigationspanel dienen. Die verschiedenen Ansichten sollen über den
Navigationspanel angesteuert werden können.

\item[US-6\label{itm:US-6}]
Der Navigationspanel soll eine zweistufige Navigation anbieten. Ansichten,
welche Gruppiert werden können, sollen in der zweiten Stufe untergebracht
werden. Gruppierte Ansichten sollen unterhalb der übergeordneten Ansicht
eingeruckt dargestellt werden.

\item[US-7\label{itm:US-7}]
Die aktuell angesteuerte Ansicht soll im Navigationspanel andersfarbig
dargestellt werden.

\item[US-8\label{itm:US-8}]
Im Navigationspanel sollen die verschiednen Navigationspunkte auf der ersten
Stufe über ein Separator visuell voneinander getrennt werden.

\item[US-9\label{itm:US-9}]
Wenn ein Navigationspunkt der ersten Stufe im Navigationspanel ``angeklickt''
wird, sollen die darunter liegenden zweitstufigen Navigationspunkt sichtbar
werden.

\item[US-10\label{itm:US-10}]
Es sollen zwei Navigationen auf der ersten Stufen angezeigt werden. Die
Navigationen sind ``Einführung'' und ``Produktauswahl''.

\item[US-11\label{itm:US-11}]
Es sollen zwei Navigationen auf der zweiten Stufen unterhalb der Navigation
``Produktauswahl'' angezeigt werden. Die Navigationen sind
``Produktbeschreibung'' und ``Produktdesign''.

\item[US-xx]
Wenn der Navigationspunkt ``Einführung'' ausgewählt wird, soll in der zweiten
Reihe und der zweiten Spalte ein Einführungstext zum Proof of Concept
dargestellt werden.

\item[US-12]
Wenn der Navigationspunkt ``Produktauswahl'' ausgewählt wird, soll in der
zweiten Reihe und der zweiten Spalte eine Buttonmatrix mit den möglichen
Produkten dargestellt werden. Die Navigationen ``Produktbeschreibung'' und
``Produktdesign'' sollen angezeigt werden und deaktiviert sein.

\item[US-13]
Wenn der Navigationspunkt ``Produktauswahl'' ausgewählt wird, sollen in der
zweiten Zeile und der dritten Spalte Filteroptionen für die Produkte in der Form
von CheckBoxen dargestellt werden.

\item[US-14]
Wenn eine Filteroption gewählt wird, sollen die Produkte entsprechend des
Filters aktiv oder inaktiv gesetzt werden.

\end{description}

\section{Priorisierung der User Stories}

\section{Akzeptanztests}

\section{Resultat}

\subsection{Testabdeckung}

\subsection{Screenshots}