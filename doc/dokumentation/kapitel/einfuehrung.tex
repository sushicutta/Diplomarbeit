  Das Internet hat sich in den letzten Jahren zu einem strategisch wichtigen
  Vertriebskanal entwickelt. Viele Unternehmen haben das erkannt und setzen im
  Betriebsalltag die Möglichkeiten, die das Internet als Vertriebskanal bietet,
  ein. In der Finanzindustrie wurde dieser Schritt mit dem Online-Banking gegen
  Ende der Neunzigerjahre gemacht. Dabei wurden die Dienstleistungen, die beim
  klassischen Bankschalter bezogen werden konnten, mit einer Web Applikation,
  über das Internet, nutzbar gemacht. Das eine Onlinestrategie funktioniert und
  die Nachfrage dafür besteht, hat sich in den letzten Jahren in der
  Finanzindustrie gezeigt.
  
  Die Informatik spielt in der Finanzindustrie im Allgemeinen eine wichtige
  Rolle. Viele Finanzinstitute hier in der Schweiz zählen heute zu den grössten
  Arbeitgebern in der Informatik, siehe \cite{WoArbeitenInformatikfachleute}.
  Insgesammt gibt es in der Schweiz acht Unternehmen, welche selber nicht aus
  der Informatik Branche kommen, die mehr als 500 Informatikfachleute
  beschäftigen. Die Hälfte davon sind Finanzinstitute, namentlich sind das
  Credit Suisse, UBS, Zürich Versicherung und die Post. In der \ac{ZKB} ist die
  Informatik ebenfalls stark vertreten.
  
  Durch die zunehmende Komplexität im Finanzgeschäft, reicht manchmal käuflich
  erwerbliche Softwarelösungen nicht vollends aus. Um den Vorsprung zur
  Konkurenz auszubauen, entwickelt die Finanzindustrie vielmals
  Computerprogramme, welche die Automatisierung ihrer Prozesse ermöglicht,
  oder mit denen Arbeiten in ihrem Kerngeschäft gemacht werden können.
  
  Aus einer selbst entwickelten Lösung heraus kann sich ein Geschäft für das
  Finanzinstitut entwicklen. Eine solche Software könnte beispielsweise externen
  Vermögensverwaltern zur Verfügung gestellt werden. Dabei gibt es verschiedene
  Vertriebskanäle für Softwareprodukte. Das Internet ist einer davon, der sich
  mit dem Onlinebanking bewährt hat, somit könnte sich das auch für
  andere bestehende Lösungen bewähren.
  
  In der \ac{ZKB} existieren viele solcher Lösungen bereits. Diese sind
  aber nicht für eine Online-Strategie entwickelt worden, sondern für den
  internen Einsatz. Einige dieser Programme wurden als Desktop Applikationen
  entwickelt, das entspricht den Anforderungen für den internen Einsatz. Um den
  Vertrieb zentral verwalten zu können, macht das Umrüsten, bestehender Desktop
  Applikationen, auf eine Online-Lösung Sinn.
  
  \section{Motivation}
  
  Die Zürcher Kantonalbank setzt bei der Entwicklung von Software als Inhouse
  Lösungen auf Java Swing Applikationen und auf Java Web Applikationen. Die
  Java Web Applikationen basieren auf dem Web Framework Apache Struts 1.3
  mit einer von der ZKB erstellten Erweiterung namens \ac{ZIP}. Da eine Lösung
  als Java Web Applikationen zentral verwaltet werden kann, gibt es erhebliche
  Einsparungen im Bereich Testing, Deployment und Patching. Somit sollen in
  Zukunft Java Swing Applikationen auf Java Web Applikationen umgerüstet werden.
  
  Das Framework Apache Struts 1.3 ist mitlerweilen in die Jahre gekommen. Es
  basiert auf dem Prinzip von ``request und response''. Wann immer eine Aktion
  von einem User in der Applikation gemacht wird, sei es das Ansteuern eines
  Links oder das Absenden eines Formulars, wird die Webseite, welche die
  Applikation repräsentiert, neu geladen. Heutzutage gibt es Frameworks, welche
  einem ermöglichen eine Web Applikation zu entwicklen, die sich bei der
  Bedienung wie eine klassische Desktop Applikation anfühlen. Dies wird durch
  die Technik von \ac{Ajax} realisiert.
  
  Bei \ac{Ajax} können Daten einer Web Applikation, ohne die komplette Webseite
  neu zu laden, verändert werden. Dies erlaubt es Web Applikationen, auf
  Benutzer Aktionen schneller zu reagieren, da vermieden wird, dass statische
  Daten, die sich unter Umständen nicht geändert haben, immer wieder über das
  Netzwerk übertragen werden müssen.
  
  Apache Struts 1.3 unterstützt \ac{Ajax} nicht direkt, das kann aber über
  Erweiterungen ermöglicht werden. Es gibt Java Web Frameworks, welche diese
  Technik out of the box mitbringen. Eine Evaluation soll zeigen welches dieser
  Java Web Frameworks am besten für einen möglichen Einsatz geeignet ist.
  
  \section{Zielsetzung}
  
  Gegenstand der vorliegenden Diplomarbeit ist die Analyse, welche Java Web
  Frameworks für die Ablösung von bestehenden Java Swing Applikationen in Frage
  kommen. Es sollen anhand eines strukturierten Vorgehens eine Menge von
  bestehenden Java Swing Applikationen der Zürcher Kantonalbank auf deren
  Funktionalitäten der Benutzeroberfläche eruiert werden. Aufgrund der
  Anforderungen der IT-Architektur der Zürcher Kantonalbank sollen Java Web
  Frameworks auf die Validität der erhobenen Funktionalitäten verifiziert
  werden. Die Java Web Frameworks, welche diesen Ansprüchen genügen, sollen auf
  einen möglichen Einsatz in der Informatik Infrastruktur der Züricher
  Kantonalbank untersucht werden. Durch die Implementierung eines Prototyen
  soll gezeigt werden, dass die Umsetzung möglich ist. Schlussendlich soll eine
  Empfehlung für ein Java Web Framework ausgesprochen werden, für den Fall,
  dass eine bestehende Java Swing Applikation in eine Web Applikation umgebaut
  werden soll. Ebenso sollen die gewonnen Erkenntnisse, für zukünftige
  Revidierungen im Bereich der Java Web Frameworks der IT-Architektur der
  Zürcher Kantonalbank, als Grundlage dienen können.
  
  \section{Struktur}
  
  tbd
  
  \section{Danksagung}
  
  tbd
  