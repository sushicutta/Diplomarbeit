  Das Internet hat sich in den letzten Jahren zu einem strategisch wichtigen
  Vertriebskanal entwickelt. Viele Unternehmen haben das erkannt und setzen im
  Betriebsalltag die Möglichkeiten, die das Internet als Vertriebskanal bietet,
  ein. In der Finanzindustrie wurde dieser Schritt mit dem Online-Banking gegen
  Ende der Neunzigerjahre gemacht. Dabei wurden die Dienstleistungen, die beim
  klassischen Bankschalter bezogen werden konnten, mit einer Web Applikation,
  über das Internet, nutzbar gemacht. Das eine Onlinestrategie funktioniert und
  die Nachfrage dafür besteht, hat sich in den letzten Jahren in der
  Finanzindustrie gezeigt.
  
  Die Informatik spielt in der Finanzindustrie im Allgemeinen eine wichtige
  Rolle. Viele Finanzinstitute hier in der Schweiz zählen heute zu den grössten
  Arbeitgebern in der Informatik, siehe \cite{WoArbeitenInformatikfachleute}.
  Insgesammt gibt es in der Schweiz acht Unternehmen, welche selber nicht aus
  der Informatik Branche kommen, die mehr als 500 Informatikfachleute
  beschäftigen. Die Hälfte davon sind Finanzinstitute, namentlich sind das
  Credit Suisse, UBS, Zürich Versicherung und die Post. In der \ac{ZKB} ist die
  Informatik ebenfalls stark vertreten.
  
  Durch die zunehmende Komplexität im Finanzgeschäft, reicht manchmal käuflich
  erwerbliche Softwarelösungen nicht vollends aus. Um den Vorsprung zur
  Konkurenz auszubauen, entwickelt die Finanzindustrie vielmals
  Computerprogramme, welche die Automatisierung ihrer Prozesse ermöglicht,
  oder mit denen Arbeiten in ihrem Kerngeschäft gemacht werden können.
  
  Aus einer selbst entwickelten Lösung heraus kann sich ein Geschäft für das
  Finanzinstitut entwicklen. Eine solche Software könnte beispielsweise externen
  Vermögensverwaltern zur Verfügung gestellt werden. Dabei gibt es verschiedene
  Vertriebskanäle für Softwareprodukte. Das Internet ist einer davon, der sich
  mit dem Onlinebanking bewährt hat, somit könnte sich das auch für
  andere bestehende Lösungen bewähren.
  
  In der \ac{ZKB} existieren viele solcher Lösungen bereits. Diese sind
  aber nicht für eine Online-Strategie entwickelt worden, sondern für den
  internen Einsatz. Einige dieser Programme wurden als Desktop Applikationen
  entwickelt, das entspricht den Anforderungen für den internen Einsatz. Um den
  Vertrieb zentral verwalten zu können, macht das Umrüsten, bestehender Desktop
  Applikationen, auf eine Online-Lösung Sinn.
  
  \section{Motivation}
  
  Die Zürcher Kantonalbank setzt bei der Entwicklung von Software als Inhouse
  Lösungen auf Java Swing Applikationen und auf Java Web Applikationen. Die
  Java Web Applikationen basieren auf dem Web Framework Apache Struts 1.3.10
  mit einer von der ZKB erstellten Erweiterung namens \ac{ZIP}. Da eine Lösung
  als Java Web Applikationen zentral verwaltet werden kann, gibt es erhebliche
  Einsparungen im Bereich Testing, Deployment und Patching. Somit sollen in
  Zukunft Java Swing Applikationen auf Java Web Applikationen umgerüstet
  werden.
  
  Das Framework Apache Struts 1.3.10 ist mitlerweilen in die Jahre gekommen. Es
  basiert auf dem Prinzip von ``request und response''. Wann immer eine Aktion
  von einem User in der Applikation gemacht wird, sei es das Ansteuern eines
  Links oder das Absenden eines Formulars, wird die Webseite, welche die
  Applikation repräsentiert, neu geladen. Heutzutage gibt es Frameworks, welche
  einem ermöglichen eine Web Applikation zu entwicklen, die sich bei der
  Bedienung wie eine klassische Desktop Applikation anfühlt. Dabei werden
  nur Bereiche neu geladen, wo sich die zu darstellenden Informationen geändert
  haben. Dies wird durch die Technik von \ac{Ajax} realisiert.
  
  Bei \ac{Ajax} können Daten einer Web Applikation, ohne die komplette Webseite
  neu zu laden, verändert werden. Dies erlaubt es Web Applikationen, auf
  Benutzer Aktionen schneller zu reagieren, da vermieden wird, dass statische
  Daten, die sich unter Umständen nicht verändert haben, immer wieder neu
  übertragen werden. Das beinhaltet nicht nur Informationen die dargestellt
  werden, sondern auch die \ac{HTML}, \ac{CSS} und Java Script Ressourcen, die
  für die Darstellung notwendig sind.
  
  Apache Struts 1.3.10 unterstützt \ac{Ajax} nicht direkt, das kann über
  Erweiterungen ermöglicht werden. Es gibt Java Web Frameworks, welche diese
  Technik out of the box mitbringen. Eine Evaluation soll zeigen ob eines dieser
  Java Web Frameworks vielleicht besser für einen möglichen Einsatz geeignet
  ist.
  
  \section{Zielsetzung}
  
  Gegenstand der vorliegenden Diplomarbeit ist die Analyse, welche Java Web
  Frameworks für die Ablösung von bestehenden Java Swing Applikationen in Frage
  kommen. Es sollen anhand eines strukturierten Vorgehens eine Menge von
  bestehenden Java Swing Applikationen der Zürcher Kantonalbank auf deren
  Funktionalitäten der Benutzeroberfläche eruiert werden. Aufgrund der
  Anforderungen der IT-Architektur der Zürcher Kantonalbank sollen Java Web
  Frameworks auf die Validität der erhobenen Funktionalitäten verifiziert
  werden. Die Java Web Frameworks, welche diesen Ansprüchen genügen, sollen auf
  einen möglichen Einsatz in der Informatik Infrastruktur der Züricher
  Kantonalbank untersucht werden. Durch die Implementierung eines Prototyen
  soll gezeigt werden, dass die Umsetzung möglich ist. Schlussendlich soll eine
  Empfehlung für ein Java Web Framework ausgesprochen werden, für den Fall,
  dass eine bestehende Java Swing Applikation in eine Web Applikation umgebaut
  werden soll. Ebenso sollen die gewonnen Erkenntnisse, für zukünftige
  Revidierungen im Bereich der Java Web Frameworks der IT-Architektur der
  Zürcher Kantonalbank, als Grundlage dienen können.
  
  \section{Struktur}
  
  Die Struktur des Dokuments wurde in sechs verschiedene Teile gegliedert.
  Diese Elemente bauen aufeinander auf und setzten meistens die Erkenntnisse
  der vorhergehenden Bestandteile voraus.
  
  \begin{description}
    
  \item[Einführung]
  
  In der \nameref{chapter:Einfuehrung} wird die Motivation für die Diplomarbeit
  erläutert, ebenso wird auf die Zielsetzung eingegangen. Die Gliederung des
  Berichts wird in der Struktur aufgeschlüsselt. Zudem befindet sich hier auch
  eine Danksagung.
  
  \item[Basisinformationen]
    
  In den Kapiteln 
  \nameref{chapter:JavaSwingApplikationen}, 
  \nameref{chapter:RichInternetApplikationen} und
  \nameref{chapter:InfrastrukturDerZuercherKantonalbank} werden
  Hintergrundinformationen zu diesen Themen vermittelt. Dabei werden
  hauptsächlich Begriffe aus diesen Bereichen erklärt und mit Hilfe von
  Abbildungen dargestellt. Die Informationen welche erläutert werden, dienen
  als Grundlage für die weiteren Kapitel.

  \item[Methodiken]
  
  In den Kapiteln \nameref{chapter:MethodenZurAnalyseVonJavaSwingApplikationen}
  und \nameref{chapter:MethodenZurEntscheidungsfindungBeiEinerEvaluation} werden
  Methodiken erarbeitet. Diese Methoden sollen für die Durchführung, der in der
  Aufgabenstellung gestellten Aufgaben, verwendet werden.
  
  \item[Durchführung der gestellten Aufgaben]
  
  In den Kapiteln \ref{chapter:AnalyseDerJavaSwingApplikationen} bis
  \ref{chapter:ProofOfConcept} werden die erwarteten Ergebnisse
  entsprechend der Aufgabenstellung erarbeitet. Das beinhaltet die
  \nameref{chapter:AnalyseDerJavaSwingApplikationen} und die
  \nameref{chapter:EvaluationDerJavaWebFrameworks}. Mit den erhaltenen
  Resultaten, wird die Integration in die ZKB Infrastruktur und die
  Implementierung der gefundenen Swingkomponenten geprüft. Als Abschluss wird
  mit einem Proof of Concept die erarbeiteten Ergebnisse bestätigt.
  
  \item[Erkenntnisse]
  
  Die Kapitel \ref{chapter:EmpfehlungFuerEinJavaWebFramework} bis
  \ref{chapter:Reflexion} wiederspiegeln die Erkenntnisse, welche im Laufe
  der Diplomarbeit gewonnen wurden. Dabei wird eine
  \nameref{chapter:EmpfehlungFuerEinJavaWebFramework} für die Züricher
  Kantonalbank ausgepsrochen, ebenso wird ein Fazit gezogen und ein Ausblick
  gewährt. Mit einer \nameref{chapter:Reflexion} sollen dargelegten
  theoretischen Grundlagen, die gewählten Methoden sowie die Auswertung der
  Ergebnisse evaluiert werden.
  
  \item[Anhang]
  
  Im Anhang sind Informationen zu finden, die entweder mit den
  \nameref{chapter:Rahmenbedingungen} der Diplomarbeit zu tun haben, wie das
  \nameref{chapter:Personalienblatt}, die \nameref{chapter:Aufgabenstellung}, die
  \nameref{chapter:Projektadministration} oder die Protokolle zu den offiziellen
  Terminen, oder es handelt sich um Zusatzinformationen, welche vom Umfang her
  zu kostspielig gewesen sind, dass sie in der Arbeit Platz gefunden hätten.
  Zudem befinden sich hier auch sämtliche Verzeichnisse, insbesondere das
  \bibname.
  
  \end{description}
  
  \section{Danksagung}
  
  An dieser Stelle möchte ich mich bei all jenen bedanken, die mich in meiner
  Studienzeit unterstützt haben. Der grösste Dank gilt meiner Freundin und
  unserem gemeinsamen Sohn Linus, die in dieser stressigen Zeit ertragen haben.
  
  Ich möchte mich in dieser Form bei Beat Seeliger bedanken, der mich als
  Betreuer bei meiner Diplomarbeit unterstützt und mir mit seiner hilfsbereiten
  und unkomplizierten Art und Weise zur Seite gestanden ist.
  
  Ebenfalls bedanken möchte ich mich bei Berhard Mäder von der Zürcher
  Kantonalbank, der mir die Arbeit ermöglicht hat und dessen Türen für mich
  immer offen standen.
  
  Weiterhin bedanke ich mich bei meinem Arbeitgeber und Freund Silvan Spross,
  der mir genügend Zeit für die Vollendung der Diplomarbeit gewährt und mit der
  notwendigen Infrastruktur der allink GmbH versorgt hat. Der Dank gilt
  auch ihm und auch Stefan Laubenberger, weil sie die Arbeit gegengelesen
  haben.
  
  Zum Schluss bedanke ich mich auch bei Stefan Pudig, XXX XXX und YYY YYY, die
  mich mit wichtigen Informationen unterstützt haben.
    