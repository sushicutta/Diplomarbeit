  Das Internet hat sich in den letzten Jahren zu einem strategisch wichtigen
  Vertriebskanal entwickelt. Viele Unternehmen haben das erkannt und setzen im
  Betriebsalltag die Möglichkeiten, die das Internet bietet, ein. In der
  Finanzindustrie wurde dieser Schritt mit dem Online-Banking gegen Ende der
  Neunzigerjahre gemacht. Das eine Onlinestrategie funktioniert und die
  Nachfrage dafür besteht hat sich in den letzten Jahren gezeigt.
  
  Die Informatik spielt in der Finanzindustrie im Allgemeinen eine Wichtige
  Rolle. Viele Finanzinstitute hier in der Schweiz zählen heute zu den grössten
  Arbeitgebern in der Informatik\cite{WoArbeitenInformatikfachleute}. Insgesammt
  gibt es in der Schweiz acht Unternehmen, welche selber nicht aus der
  Informatik Branche kommen, die mehr als 500 Informatikfachleute beschäftigen.
  Die Hälfte davon sind Finanzinstitute, namentlich sind das Credit Suisse,
  UBS, Zürich Versicherung und die Post. In der Zürcher Kantonalbank ist die
  Informatik ebenfalls stark vertreten.
  
  Durch die zunehmende Komplexität im Bankengeschäft, reichen manchmal käuflich
  erwerbliche Softwarelösungen nicht vollends. Um den Vorsprung zur Konkurenz
  auszubauen, entwickeln die Banken vielmals Computerprogramme selber, welche
  die Automatisierung ihrer Prozesse übernehmen, oder mit denen
  finanzmathematische Modelle berechnet werden können.
  
  Aus einer solchen Lösung heraus kann sich ein Geschäft für die Bank
  entwicklen. Eine Software, die von einem Finanzinstitut selber entwickelt
  wurde, könnte beispielsweise externen Vermögensverwaltern zur Verfügung
  gestellt werden. Dabei gibt es verschiedene Vertriebskanäle. Was sich mit dem
  Onlinebanking bewährt hat, kann sich auch für bestehende Lösungen bewähren.
  
  In der Zürcher Kantonalbank existieren viele solcher Lösungen bereits, sind
  aber nicht für eine Online-Strategie entwickelt worden, sonder für den
  internen Einsatz. Einige solcher Programme wurden als Desktop Applikationen
  entwickelt, das entsprach durchaus den Anforderungen für einen internen
  Einsatz. Um den Vertrieb zentral verwalten zu können, macht das Umrüsten
  auf eine Online-Lösung, durchaus Sinn.
  
  \section{Motivation}
  
  Die Zürcher Kantonalbank setzt bei der Entwicklung von Software als Inhouse
  Lösungen auf Java Swing Applikationen und auf Java Web Applikationen. Die
  Java Web Applikationen basieren auf dem Web Framework Apache Struts 1.3
  mit einer von der ZKB erstellten Erweiterung namens ZIP (ZKB Internet
  Plattform). Da eine Lösung als Java Web Applikationen zentral verwaltet
  werden kann gibt es erhebliche Einsparungen im Bereich Testing, Deployment
  und Patching. Somit sollen in Zukunft Java Swing Applikationen auf Java Web
  Applikationen umgerüstet werden.
  
  Das Framework Apache Struts 1.3 ist mitlerweilen in die Jahre gekommen. Es
  basiert auf dem Prinzip von ``request und response''. Wann immer eine Aktion
  von einem User in der Applikation gemacht wird, sei es das Ansteuern eines
  Links oder das Absenden eines Formulars, wird die Webseite, welche die
  Applikation repräsentiert, neu geladen. Heutzutage gibt es Frameworks, welche
  einem ermöglichen eine Web Applikation zu entwicklen, die sich bei der
  Bedienung wie eine klassische Desktop Applikation anfühlen. Dies wird durch
  die Technik von \ac{Ajax} realisiert.
  
  Bei \ac{Ajax} können Daten einer Web Applikation, ohne die komplette Webseite
  neu zu laden, verändert werden. Dies erlaubt es Web Applikationen, auf
  Benutzer Aktionen schneller zu reagieren, da vermieden wird, dass statische
  Daten, die sich unter Umständen nicht geändert haben, immer wieder über das
  Netzwerk übertragen werden müssen.
  
  Apache Struts 1.3 unterstützt \ac{Ajax} nicht direkt, kann aber über
  Erweiterungen ermöglicht werden. Es gibt Java Web Frameworks, welche diese
  Technik out of the box mitbringen. Eine Evaluation soll zeigen welches dieser
  Java Web Frameworks am besten für einen möglichen Einsatz geeignet ist.
  
  \section{Struktur}
  
  tbd
  
  \section{Danksagung}
  
  tbd
  