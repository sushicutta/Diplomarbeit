  Anhand einer Menge von KO-Kriterien wird die Auswahl der Alternativen
  eingeschränkt. Die KO-Kriterien sind aus dem \begin{itshape}Handbuch der
  IT-Architektur\end{itshape}, siehe \cite{ZkbHandbuchDerItArchitektur}, der
  \ac{ZKB} entnommen. Die Namensgebung in dem Handbuch unterscheidet sich
  leicht, es wird nicht von KO-Kriterien, sondern von Grundsätzen gesprochen.
  
  KO-Kriterien sind Vorgaben, welche zwingend erfüllt sein müssen. Falls ein
  Kriterium nicht erfüllt werden kann, fällt die Entscheidung auf diese
  Alternative negativ aus. Jedem KO-Kriterium wird für die Identifikation eine
  eindeutige ID vergeben. Die ID setzt sich folgendermassen zusammen: 
  \{KO\}-\{Laufnummer\}.

  \section{Grundsätze aus der IT-Architektur der Zürcher Kantonalbank}
  
  Ein Grundsatz wird im Handbuch wer IT-Architektur wie folgt definiert:
  \newline
  
  ``\begin{itshape}Es sind Grundsätze definiert, nach denen sich die Baupläne
  der IT-Systeme zu richten haben. Die Grundsätze sind ein Regelwerk mit
  Weisungscharakter.\end{itshape}''
  \footnote{\cite{ZkbHandbuchDerItArchitektur} Kapitel 1.3 - \begin{itshape}Was
  ist die IT-Architektur der ZKB\end{itshape}, Seite 11}
  \newline

  \noindent
  Dabei gibt es eine Hintertür:
  \newline

  ``\begin{itshape}Grundsätze sind verbindliche Vorgaben (Konventionen), von
  denen nur in begründeten Ausnahmen abgewichen werden kann.\end{itshape}''
  \footnote{\cite{ZkbHandbuchDerItArchitektur} Kapitel 1.8 -
  \begin{itshape}Leseanleitung\end{itshape}, Seite 14}
  \newline
  
  \noindent
  Das Dokument wurde analysiert und alle Grundsätze, welche für diese
  Diplomarbeit relevanten sind, werden im Anhang
  \ref{chapter:GrundsaetzeDerZkbItArchitektur}
  \nameref{chapter:GrundsaetzeDerZkbItArchitektur} (auf Seite 
  \pageref{chapter:GrundsaetzeDerZkbItArchitektur}ff) aufgelistet und mit einer
  ID versehen. Insgesamt wurden 44 Grundsätze erkannt, welche einen Einfluss
  auf die Evaluation haben könnten.
  
  \section{KO-Kriterien die im Rahmen der Diplomarbeit nicht beachtet werden
  sollen}\label{section:KoKriterienDieNichtBeachtetWerden}
  
  Damit die Durchführung der Evaluation einen Sinn ergibt, sollen einige
  KO-Kriterien nicht beachtet werden. In Absprache mit dem Auftraggeber
  betrifft das folgende KO-Kriterien, die im Rahmen der Diplomarbeit nicht
  berücksichtigt werden:
  
  \begin{itemize}
    \item \ref{itm:KO-09} - Für Java-Applikationen (Internet, Extranet und
    Intranet) wird das ZIP-Framework eingesetzt.
    \item \ref{itm:KO-16} - Die Internet-Applikationen funktionieren auch
    eingeschränkt, ohne dass die Skript-Funktion im Browser aktiviert ist.
    \item \ref{itm:KO-20} - Für Ultra Thin Clients bzw. Browser-basierende
    Applikationen muss das aktuelle, Struts-basierende HTML-Client-Framework
    der ZKB Internet Plattform verwendet werden.
  \end{itemize}
