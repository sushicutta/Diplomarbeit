  Folgende Anforderungen stammen aus dem Dokument ``\begin{itshape}18
  Anforderungen an Webframeworks -
  OpenDoc\end{itshape}'', siehe \cite{AnforderungenAnWebframeworks}, und werden
  hier zusammengefasst.
  
  \section{Auflistung der Anforderungen}
  
  Die Anforderungen sind hier aufgelistet und mit einer ID in der Form
  Soll\}-\{Laufnummer\} versehen, da die Anforderungen als Soll-Kriterien für
  die Evaluation der Java Web Frameworks dienen.
  
  \begin{description}

    \item[Soll-01 - Zugriffskontrolle
    (Authentifizierung/Authorisation/Rollenverwaltung)\label{itm:Soll-01}]

    Ein Webframework sollte EntwicklerInnen verschiedene Mechanismen
    bereitstellen, um die Anwendung vor fremden und unerlaubten Zugriff schützen
    zu können.

    Authentifizierung/Autorisierung: Üblicherweise werden im Vorfeld Rollen für
    verschiedenen Gruppen festgelegt. Das Ziel einer sicheren Webanwendung ist
    es, bestimmte Bereiche einer Seite abzusichern und die Rechte aller
    Benutzer je nach Rolle einzuschränken. Anhand der Rolle wird deren
    Benutzer für die festgelegten Bereiche autorisiert.

    Vertraulichkeit / Verschlüsselung: Sensible Daten, wie Passwörter und
    Personendaten, müssen vor dem Zugriff und der Kenntnisnahme von Dritten
    geschützt werden. Hierfür werden die Daten während der Übertragung
    verschlüsselt. Für eine sichere Datenübertragung werden meist
    Verschlüsselungsprotokolle wie SSL (Secure Sockets Layer), sowie dessen
    Nachfolger TLS (Transport Layer Security) eingesetzt. Diese gelten als
    relativ sicher und sind bei Transaktionen bei einer Bank unverzichtbar.

    \item[Soll-02 - Form-Validierung\label{itm:Soll-02}]
    Das Verarbeiten von Formularen bzw. die Handhabung von Benutzereingaben und
    -aktionen gehört zu den täglichen Aufgaben der Webentwicklung. Die Logik
    für server- und clientseitiges Validieren ist im Idealfall nur einmal
    implementiert.

    Server-Side Validation: Das Webframework soll die Mögichkeit bieten,
    eingegebene Daten einfach zu überprüfen. Dabei soll für jede Eigenschaft
    eines Datenmodells (Model) ein Wertebereich definierbar sein, zusätzlich
    soll geprüft werden, ob die Eingabe erforderlich ist. Die Programmierlogik
    soll minimal sein. Nach dem Senden der Daten wird alles überprüft und ggf.
    entsprechende Fehlermeldungen zurückgegeben.

    Client-Side Validation: Das Webframework soll wenn möglich schon auf dem
    Client validieren. Wenn möglich, sollen die Daten gleich bei oder kurz nach
    der Eingabe überprüft werden, wobei die Logik auf dem Server implementiert
    ist. Beispiel: Ist der Anmeldename schon vergeben. Dadurch werden
    Serverressourcen gespart und der Benutzer hat ein direktes Feedback.

    \item[Soll-03 - Modulare Architektur\label{itm:Soll-03}]

    Eine Webanwendung stellt ein Zusammenspiel verschiedenster
    Internettechnologien dar, die wiederum hinsichtlich ihrer Entwicklung einem
    stetigen Wandel unterliegen. Die Herausforderung für die Entwickler ist es,
    die Entwicklung der Technologien im Auge zu behalten um gegebenenfalls
    Neuheiten oder Änderungen im System anzupassen. Eine Webanwendung sollte
    daher in all ihren Bestandteilen möglichst wartbar bleiben.

    \item[Soll-04 - Schnittstellen und Webservices\label{itm:Soll-04}]

    Interoperabilität beschreibt den Austausch von Informationen verschiedener
    Softwaresysteme. Es gibt verschiedene Technologien, mit denen ein
    Informationsaustausch umgesetzt werden kann - REST, SOAP und RPC sind am
    weitesten verbreitet. Das Webframework sollte Mechanismen und Funktionen
    zur Umsetzung von Schnittstellen bereitstellen.
    
    \item[Soll-05 - MVC-Entwurfsmuster\label{itm:Soll-05}]
    Besonders im Web hat sich das Model-View-Controller Entwurfsmuster als
    quasi Standard-Architekturmuster für Webanwendungen etabliert und hält
    daher Einzug bei den meisten Webframeworks. Es dient zur Strukturierung der
    Software in drei Einheiten. Das Model (Datenmodell) enthält die
    Geschäftslogik - die Informationen die dargestellt werden. Der Controller
    (Steuerung) ist die Schnittstelle zwischen der View und dem Datenmodell. Er
    nimmt Benutzeraktionen entgegen (z.B. Formulardaten) und leitet sie an ein
    bestimmtes Datenmodell weiter. Der Controller führt dann Operationen wie
    speichern, ändern und löschen auf dem Datenmodell (besser gesagt dem
    Objekt) aus. Die View ist die Präsentationsschicht und stellt die Daten
    dar, die es vom Controller entgegennimmt. Jedoch sollte eine View nicht
    ohne einen Controller neue Objekte erzeugen oder speichern (Trennung von
    Logik und Darstellung). Mit dem MVC-Entwurfsmuster können u.a.
    ProgrammiererInnen und DesignerInnen während der Entwicklung unabhängig
    voneinander arbeiten.

    \item[Soll-06 - Testing\label{itm:Soll-06}]

    Testing ist mit test-driven development (TDD), vor allem in der agilen
    Softwareentwicklung, ein fester Bestandteil während der Projektentwicklung.
    Vor der Implementierung überprüft der Programmierer, mittels Unit-Tests, 
    konsequent das Verhalten jeglicher Komponenten. Gerade kritische Prozesse
    und Transaktionen (zum Beispiel eine Banküberweisung) sollten ausgiebig
    getestet werden. Das Webframework soll die Möglichkeit von Unit-Tests
    bieten.

    \item[Soll-07 - Internationalisierung und Lokalisierung\label{itm:Soll-07}]

    Viele Webanwendungen richten sich mittlerweile an ein internationales
    Publikum. Dank der Offenheit und der weiten Verbreitung des Internets lassen
    sich dadurch sehr einfach neue Zielgruppen (BenutzerInnen) erschließen.
    Jedoch gilt es, einige Vorraussetzungen und Besonderheiten bei der
    Internationalisierung von Webanwendungen zu beachten. Zunächst müssen
    sämtliche Texte übersetzt und möglicherweise in neue Datenbanken ausgelagert
    werden. Hierbei ist zu berücksichtigen, dass es länderspezifische
    Zeichensätze, Zahlen, Datum und Währungswerte gibt. Hinzu kommt, dass unter
    Umständen auch spezielle Grafiken neu erstellt werden müssen.
    
    \item[Soll-08 - Object Relational Mapping (ORM)\label{itm:Soll-08}]
    Die meisten Webframeworks unterstützen das objektorientierte
    Programmierparadigma. Im Zusammenhang mit relationalen Datenbanken kommt es
    allerdings zu grundlegenden Problemen, weil der Zustand und das Verhalten
    eines Objekts nicht in einer relationalen Datenbank gespeichert werden
    kann. Dies ist auf die beiden widersprüchlichen Konzepte von
    objektorientierter Programmierung (OOP) und relationalen
    Datenbankmanagementsystemen (RDMS) zurückzuführen. Im Gegensatz zu
    objektorientierten Datenbanken, werden beim ORM die Tabellen aus der
    Datenbank als Klassen abgebildet (gemappt). Durch die Klassenabbildung wird
    für den/die ProgrammiererIn wieder die gewohnte OOP-Umgebung geschaffen.
    Wenn ein Objekt erstellt oder geändert wird, ist der Mapper verantwortlich,
    um diese Informationen in der Datenbank zu speichern und auch ggfs. wieder
    zu löschen. ORM bildet somit eine Schnittstelle zwischen OOP und
    relationalen Datenbanken.

    Unterstützung von Transaktionen: Bei der Speicherung von Informationen in
    mehreren Datenbanktabellen muss sicher gestellt sein, dass entweder alle
    oder keine Informationen gespeichert werden.

    \item[Soll-09 - Scaffolding / Rapid Prototyping\label{itm:Soll-09}]

    Mit Scaffolding ist das automatische Generien von den sogenannten CRUD-Pages
    (Create, Read, Update, Delete) gemeint.

    Scaffolding und das dadurch verstandene Rapid Prototyping ist ein
    wesentlicher Aspekt der agilen Softwareentwicklung (zum Beispiel in
    Verwendung mit der Scrum-Methodik). Gerade zu Beginn eines Projekts eignet
    sich das Rapid Prototyping, da Änderungen in den Models umgehend in die
    Views eingebunden werden können.

    \item[Soll-10 - Caching\label{itm:Soll-10}]

    Neben der Übertragunsgeschwindigkeit gibt es eine Menge anderer Faktoren,
    die für eine leistungsstarke Webanwendung entscheidend sind. Ein Aspekt ist
    das Caching - das Zwischenspeichern von Daten, die häufig verwendet bzw.
    aufgerufen werden.

    Das Caching kann meist auf verschiedenen Ebenen implementiert werden: Auf
    dem Client, auf dem Webserver und auf der Datenbankebene. Dabei soll das
    Webframework diese Mechanismen unterstützen und es ermöglichen, diese
    einfach zu aktivieren und zu kalibrieren.  

    \item[Soll-11 - View-Engine\label{itm:Soll-11}]

    View Engines werden eingesetzt, um das Arbeiten mit den Views zu
    erleichtern. Sie unterstützen vor allem das Templating - das Erstellen von
    Vorlagen. Durch typisierte Views wird eine Webanwendung robuster, weil
    weniger fehleranfällig; durch partial Views (oft auch nur als partials
    bezeichnet) werden einzelne Elemente oder Bereiche der Benutzeroberfläche
    wiederverwendbar gemacht. Diese Vorlagen können von mehreren Views genutzt
    werden. Dadurch wird deutlich weniger redundanter Code erstellt. Dies ist
    ein wichtiger Aspekt in Bezug auf das Don’t repeat yourself (DRY) Prinzip.

    \item[Soll-12 - Dokumentation\label{itm:Soll-12}]

    Eine Webanwendung ist aus Entwicklersicht eine Zusammenfassung
    verschiedenster Technologien (Webframeworks, Bibliotheken, Schnittstellen).
    Über das Application Programming Interface (API) haben Programmierer Zugriff
    auf die verfügbaren Funktionen. Nicht selten erstreckt sich die
    Dokumentation einer API über mehrere hundert Seiten. Für Programmierer ist
    es daher von enormer Bedeutung, dass die Bibliotheken gut strukturiert und
    verständlich beschrieben sind. Schlecht oder gar nicht dokumentierte
    Technologien erhöhen die Fehlerquote und sind oft ausschlaggebend für einen
    nicht flüssigen Workflow.

    \item[Soll-13 - Community\label{itm:Soll-13}]

    Die beteiligten Personen in den Foren, Mailing-Listen oder Wikis bilden im
    Zusammenhang mit Webframeworks die Community. Es findet dabei ein
    Wissensaustausch statt, der weit über die standard-Dokumentation hinaus
    geht. Die Nutzer helfen sich gegenseitig bei Problemen und Fehlern und oft
    hinterlassen sie mit ihren Einträgen wiederum einen Lösungsansatz, der
    zukünftig von anderen wieder aufgegriffen werden kann. Es ist daher 
    wichtig, dass den Anwendern eine Möglichkeit geboten wird, sich
    auszutauschen, gemeinsam Fehlermeldungen zu deuten und Lösungsansätze zu
    entwickeln.

    \item[Soll-14 - IDE-Unterstützung\label{itm:Soll-14}]

    In der Softwareentwicklung ist die IDE das Basiswerkzeug für die
    Programmierer. Die Entwicklungsumgebung stellt den Entwicklern verschiedene
    Komponenten zur Verfügung, wie Editor, Compiler, Linker oder Debugger.
    Hinzu kommen mit Syntaxhighlighting, Refactoring und Code-Formatierung
    weitere wichtige Funktionen, die die Entwickler in vielerlei Hinsicht enorm
    unterstützen.

    \item[Soll-15 - Kosten für Entwicklungswerkzeuge\label{itm:Soll-15}]

    Je nach verfügbaren finanziellen Mitteln spielen die Kosten von
    Entwicklungswerkzeugen und Technologien durchaus eine Rolle. Bei beiden
    Faktoren hat man meist die Wahl zwischen kostenlosen (OpenSource) und
    kommerziellen Produkten. Je nach Webanwendung müssen somit Lizenzgebühren
    für die Entwicklungswerkzeuge, Server zum Ausführen der Anwendung und
    Datenbankserver berücksichtigt werden. Dabei ist es wichtig die richtige
    Mischung verschiedener Komponenten zu finden.
  
    \item[Soll-16 - Eignung für agile Entwicklung\label{itm:Soll-16}]

    Die herkömmlichen Methoden der Software-Entwicklung werden heute oft durch
    neue agile Methoden, wie Extreme Programming oder Scrum abgelöst. Sie
    fokussieren auf das Wesentliche und stehen für deutlich mehr Flexibilität in
    der Entwicklungsphase als konventionelle Methoden. Verschiedene Technologien
    unterstützen die agilen Methoden. Refactoring, Testing spielt dabei eine
    wichtige Rolle und soll unterstützt werden.

    \item[Soll-17 - Lernkurve für EntwicklerInnen\label{itm:Soll-17}]

    Zur Umsetzung einer komplexen Webanwendung wird den Entwicklern ein Wissen
    über verschiedenste Bereiche der Softwareentwicklung abverlangt.
    Glücklicherweise gibt es für die Webentwicklung keinen einheitlichen
    Standard, der festlegt, wie eine Webanwendung entwickelt werden muss. Das
    Internet bietet für jeden Bereich eine Auswahl unterschiedlicher
    Technologien an. Daher müssen vor der Entwicklung mehrere Entscheidungen
    getroffen werden - hinsichtlich Programmiersprache, Javascript-Framework
    oder Datenbankserver. Es werden daher oft Technologien gewählt, die leicht
    zu erlernen und verstehen sind.

    Wichtig ist, dass man einen schnellen Einstieg bekommt und Erfolge bald
    sichtbar werden, um die Motivation der Entwickler zu erhöhen.

    \item[Soll-18 - AJAX-Unterstützung\label{itm:Soll-18}]

    Durch das Aufkommen von Javascript-Bibliotheken wie jQuery und Prototype
    haben sich vollkommen neue Möglichkeiten eröffnet mit Javascript auf dem
    Client zu arbeiten und mit AJAX wurde die Kommunikation zwischen Server und
    Client im Web revolutioniert. Die direkte Unterstützung eines
    Javascript-Frameworks ist für ein Webframework sinnvoll und erwünscht.

    Darüber hinaus sollte die unterstützte Bibliothek lose gekoppelt sein, und
    damit austauschbar. Sogenanntes unobtrusive Javascript, bei dem auch bei
    abgeschaltetem Javascript die Anwendung funktioniert, ist auch eine Anforderung.
  \end{description}