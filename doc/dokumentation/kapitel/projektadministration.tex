  Durch eine Detailanalyse der Aufgabenstellung werden die die wichtigsten
  Meilensteine und Arbeitsschritte ausgearbeitet. Aufgrund dieser Erkenntnisse
  wird eine Grobplanung festgelegt. Damit eine Nachvollziehbarkeit garantiert
  ist, werden Termine, Meilensteine und Arbeitsschritte aufgelistet.
  
  \section{Detailanalyse der Aufgabenstelltung}
  
  \section{Projekt Termine}
  
  Die Projekt Termine wurden alle eingehalten, siehe Tabelle \ref{tab:termine}.
  \newline
  
  \begin{table}[h]
    \begin{center}
      \begin{tabular}{lp{7cm}ll}
        \toprule
        Termin & Datum & Ort \\
        \midrule
        21. 03. 2011 & Inhaltliches Kick-off Meeting & Panter llc\\
        13. 04. 2011 & Offizielles Kick-off Meeting & HSZ-T\\
        xx. xx. 2011 & Design-Review Meeting & HSZ-T\\
        xx. xx. 2011 & Abgabe der Dokumentation & HSZ-T\\
        xx. xx. 2011 & Schlusspräsentation & HSZ-T\\
        \bottomrule
      \end{tabular}
      \caption{Projekt Termine}
      \label{tab:termine}
    \end{center}
  \end{table}
  
  \section{Projekt Meilensteine}
  
  In der Projekt Historie sind die wichtigsten Meilensteine ersichtlich, siehe
  Tabelle \ref{tab:projekthistorie}.
  \newline
  
  \begin{table}[h]
    \begin{center}
      \begin{tabular}{lp{7cm}ll}
        \toprule
        Datum & Status & Wer \\
        \midrule
        14. 03. 2011 & Ein Dozierender hat die Arbeit inkl. Aufgabenstellung
        ausgeschrieben und wartet auf einen Studierenden der  diese Arbeit
        durchführt & Roman Würsch\\
        14. 03. 2011 & Eingabe Aufgabenstellung & Roman Würsch\\
        15. 03. 2011 & Die Arbeit ist freigegeben (Eine Semester- oder
        Bachelorarbeit kann nur durch die Studiengangsleitung freigegeben
        werden) & Olaf Stern\\
        16. 03. 2011 & Der Kickoff-Termin wurde reserviert & Roman Würsch\\
        \bottomrule
      \end{tabular}
      \caption{Projekt Historie}
      \label{tab:projekthistorie}
    \end{center}
  \end{table}
  
  \section{Arbeitsschritte}
  
  Alle vorgenommenen Arbeitsschritte werden in einem Wiki für die
  Nachvollziehbarkeit protokolliert. Das Wiki ist im Internet öffentlich
  zugänglich unter der \ac{URL}:\\
  \\
  \url{https://github.com/sushicutta/Diplomarbeit/wiki/Arbeitsprotokoll}.