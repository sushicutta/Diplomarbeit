  In der Analyse der Webframeworks sollen fünf Alternativen untersucht werden.
  Es soll eine objektive Methode zur Evaluation dafür gewählt werden. Mit der
  definition von sinnvollen Rahmenbedingungen soll die Evaluation durchgeführt
  und die Resultate vorgelegt werden.
  
  \section{Methoden zur Entscheidungsfindung bei einer Evaluation}
  
  Es gibt verschiedene Methoden wie man bei der Auswahl einer Softwarelösung
  vorgehen kann. Um eine möglichst objektive Betrachtung zu gewährleisten, wurde
  die Methode der gewichteten \ac{NWA} gewählt, siehe \cite{Nutzwertanalyse}.
  Diese Methode stammt aus dem Bereich der quantitativen Analysemethoden der
  Eintscheidungstheorie. Um eine möglichst präzise objektive Gewichtung der
  einzelnen Faktoren zu erhalten wurde die Methode \ac{AHP} gewählt, siehe
  \cite{AnalyticHierarchyProcess}. Diese Methode stammt aus dem Bereich der
  präskriptiven Entscheidungstheorie. Der \ac{AHP} wurde von Thomas L. Saaty in
  den 70er Jahren des 20. Jahrhunderts entwickelt und im Buch ``The Analytic
  Hierarchy Process: Planning, Priority Setting, Resource
  Allocation''\cite{AnalyticHierarchyProcessBook} veröffentlicht.
  
  \subsection{Gewichtete Nutzwertanalyse}
  
  Bei der gewichteten \ac{NWA} wird eine Menge von Kandidaten, auf deren
  Nutzen, miteinander verglichen. Der Vergleich wird über \(n\) vergleichbare
  Alternativen geführt. Dabei werden die einzelnen Alternativen mit einem
  Erfüllungsgrad \(e_i\) bewertet. Die Skala der Erfüllungsgrade ist in der
  Tabelle \ref{tab:erfuellungsgrade} ersichtlich.
  \newline
  
  \begin{table}[h]
    \begin{center}
      \begin{tabular}{lc}
        \toprule
        Erfüllungsgrad & Skala\\
        \midrule
        nicht erfüllt & 0\\
        schlecht & 1, 2\\
        mittel & 3 - 5\\
        gut & 6 - 8\\
        sehr gut & 9\\
        \bottomrule
      \end{tabular}
      \caption{Skala der Erfüllungsgrade}
      \label{tab:erfuellungsgrade}
    \end{center}
  \end{table}
  
  Jede Alternative wird durch einen Gewichtungsfaktor \(g_i\) versehen, was
  die Präferenz der Alternative wiederspiegelt. Dabei gilt: Die Gewichte gi werden
  so gewählt, dass ihre Summe 1 (100\%) ergibt, siehe Formel \ref{eq:gewicht}.

  \begin{equation}
    \label{eq:gewicht}
    \(Gewicht:= \sum \limits_{i=1}^n g_i = 1\)  
  \end{equation}

  \newpage

  Der Nutzwert ergibt sich durch die Formel \ref{eq:nutzwert}:

  \begin{equation}
    \label{eq:nutzwert}
    \(Nutzwert:= \sum \limits_{i=1}^n e_i \cdot g_i\)  
  \end{equation}
  
  Für jeden zu evaluierenden Kandidaten soll geprüft werden, ob eines der
  KO-Kriterien erfüllt ist, was zu einem Ausschluss des Kandidaten führen würde.
  Falls das nicht der Fall ist, wird der Nutzwert des Kandidaten berechnet.
  Derjenige Kandidat mit dem grössten Nutzwert entspricht am meisten den
  Anforderungen.
  
  \subsubsection{Anschauliches Beispiel}
  
  Als anschauliches Beispiel sollen zwei Kandidaten - Auto \(A\) und \(B\) -
  miteinander verglichen werden. Sie sollen auf deren Alternativen - Leistung,
  Aussehen und Alltagstauglichkeit - verglichen werden. In der Tabelle
  \ref{tab:beispielNwa} ist das Beispiel ersichtlich. Das Resultat zeigt, dass
  das Auto \(A\) dem Auto \(B\)  gegenüber bevorzugt werden soll, da der
  Nutzwert \(5.0 > 4.7\) ist. 
  
  \begin{table}[h]
    \begin{center}
      \begin{tabular}{lrrrrr}
        \toprule
        Alternativen & Gewichtung \(g\) & \(e_A\) & Wertigkeit \(A\) & \(e_B\)
        & Wertigkeit \(B\)\\
        \midrule
        Leistung            & 0.3 & 7 & 2.1 & 9 & 2.7 \\
        Aussehen            & 0.2 & 2 & 0.4 & 5 & 1.0 \\
        Alltagstauglichkeit & 0.5 & 5 & 2.5 & 2 & 1.0 \\
        \midrule
        \midrule
        Ergebnis            & 1.0 &   & 5.0 &   & 4.7 \\
        \bottomrule
      \end{tabular}
      \caption{Beispiel einer Nutzwertanalyse}
      \label{tab:beispielNwa}
    \end{center}
  \end{table}
 
  \subsection{Analytic Hierarchy Process}
  
  Der Analytic Hierarchy Process stamt aus der Feder eines Mathematikers. Aus
  diesem Grund ist das Verfahren auch einiges Anspruchsvoller als eine
  gewichtete Nutzwertanalyse. Ich gehe hier nicht auf die ganzen Details ein, da
  es den Rahmen der Diplomarbeit übersteigen würde. Totzdem soll eine grober
  Überblick über den \ac{AHP} gegeben werden.
  
  Der \ac{AHP} besteht aus drei Phasen, siehe \cite{AnalyticHierarchyProcess}: 
  
  \begin{enumerate}
    \item Sammeln der Daten
    \item Daten vergleichen und gewichten
    \item Daten verarbeiten
  \end{enumerate}
  
  \subsubsection{1. Sammeln der Daten}
  
  In der ersten Phase sollen alle Daten, die für eine Entscheidungsfindung
  erheblich sind, gesammelt werden.
  
  \begin{itemize}
    \item Zuerst soll eine konkrete Frage formuliert werden, für welche die
    beste Antwort gesucht wird.
    \item Danach sollen zu der gestellten Frage alle Kriterien  gesucht werden,
    welche die Lösung beeinflussen können.
    \item Als letztes sollen alle Alternativen gesucht werden, welche als
    mögliche Lösung infrage kommen.
  \end{itemize}
  
  \subsubsection{2. Daten vergleichen und gewichten}
  
  In der zweiten Phase folgt nun die Gegenüberstellung, Vergleich und Bewertung
  aller Kriterien beziehungsweise Alternativen in zwei Unterschritten.
  
  \begin{itemize}
    \item Jedes Kriterium wird jedem anderen gegenübergestellt und darauf
    verlgichen, was eine grössere Bedeutung in der gestellten Frage hat. Die
    Skala geht von 1 bis 9, siehe Tabelle \ref{tab:vergleichsgrade}.
    \item Für jedes Kriterium wird jede mögliche Alternativen mit jeder anderen
    gegenübergestellt und auf ihre Eignung hin untersuchen, welche Alternative
    am besten zur Erfüllung des jeweiligen Kriteriums passt. Die Skala geht von
    1 bis 9, siehe Tabelle \ref{tab:vergleichsgrade}.
  \end{itemize}
  
  \begin{table}[h]
    \begin{center}
      \begin{tabular}{lc}
        \toprule
        Bedeutung & Skala\\
        \midrule
        gleiche Bedeutung & 1\\
        leicht grössere Bedeutung & 2 - 3\\
        viel grössere Bedeutung & 4 - 6\\
        erheblich grössere Bedeutung & 7 - 8\\
        absolut dominierend & 9\\
        \bottomrule
      \end{tabular}
      \caption{Skala der Vergleichsgrade}
      \label{tab:vergleichsgrade}
    \end{center}
  \end{table}
    
  \subsubsection{3. Daten verarbeiten}
  
  Mit einem mathematischen Modell, kann der \ac{AHP} nun eine präzise Gewichtung
  aller Kriterien errechnen. Mit der Gewichtung der Kriterien und dem Vergleich
  der Alternativen, kann der \ac{AHP} nun berechnen, welches die beste Lösung
  (Alternative) für die gestellt Frage ist.
  Diese Berechnungen werden meistens mit der Unterstütztung einer Software
  gemacht. Als Beispiel gibt es JAHP 2.1, was ein Java Programm ist, mit 
  welchem der gesammte Prozess des \ac{AHP} abgebildet und berechnet werden
  kann. Das Programm wird unter den Bedingungen der GNU General Public License
  vertrieben.
    
  \subsection{Kombination beider Methoden}
  
  Diese beiden Methoden können auch kombiniert eingesetzt
  werden\cite{AhpNwaKombination}, da der \ac{AHP} relativ komplex in der
  Umsetztung ist. Als Kombination kann die Nutzwertanalyse als Methode
  verwendet werden, und der \ac{AHP} wird für die präzise berechnung der
  Gewichtung einzelner Entscheidungskriterien verwendet. Aus der Kombination
  entsteht somit eine neue Methode, welche durch die Verständlichkeit der
  \ac{NWA} und der objektiven Gewichtung des \ac{AHP} eine plausable Evaluation
  ermöglicht.
    
  \section{Rahmenbedingungen zur Auswahl der Web Frameworks}
  
  Die Auswahl der zu verwendenden Web Frameworks wird durch eine Menge von
  Soll-Kriterien bestimmt. Diese Menge bestimmt die Anforderungen, welche
  erfüllt werden sollen. Die Anforderungen stammen von einer Projektgruppe der
  Hochschule für Technik und Wirtschaft Berlin.
  
  Anhand einer Menge von KO-Kriterien wird die Auswahl eingeschränkt. Es werden
  dabei die Vorgaben aus der IT-Architektur der Zürcher Kantonalbank verwendet,
  welche klare Regeln definiert.

  \subsection{Soll-Kriterien}
  
  Soll-Kriterien sind Vorgaben, die möglichst weitgehend Erfüllt werden sollen.
  Wenn ein solches Kriterium nicht erfüllt werden kann, schliesst das die
  Alternative nicht aus. Jedem Soll-Kriterium wird für die Identifikation eine
  eindeutige ID vergeben. Die ID setzt sich folgendermassen zusammen.
  \{Soll\}-\{Laufnummer\}.
  
  \subsubsection{Anforderungen an Webframeworks nach AgileLearn}
  
  Eine Projektgruppe namens AgileLearn von der Hochschule für Technik und
  Wirtschaft Berlin befasst sich mit dem Thema Web Frameworks. Die
  Projektgruppe hat sich die Frage gestellt: ``\begin{itshape}Welche
  Anforderungen müssen bei der Wahl eine Webframeworks berücksichtigt
  werden?\end{itshape}'' Aus der Fragestellung heraus, haben sie 18
  Anforderungen an ein Web Framework ausgearbeitet, welche öffentlich in einem
  Google-Doc\footnote{Ein Service von Google um Dokumente öffentlich zu
  bearbeiten.} ersichtlich sind.
  
  Folgende Anforderungen stammen aus dem Dokument ``\begin{itshape}18
  Anforderungen an Webframeworks -
  OpenDoc\end{itshape}''\cite{AnforderungenAnWebframeworks} und werden hier
  zusammengefasst. Da auf zwei Anforderungen verzichtet wurde, sind 16
  Anforderungen übrig geblieben, welche in den Katalog der Soll-Kriterien
  aufgenommen werden. Es wurde auf folgende Punkte als Anforderung an ein
  Webframework verzichtet:
  \newline
  
  ``MVC-Entwurfsmuster''
  \newline
  \newline
  \noindent
  Das MVC-Konzept ist sicher gut, sollte aber nicht eine Anforderung an ein Web
  Framework sein, da es viele ebenbürdige Alternativen gibt.
  \newline
  
  ``Object Relational Mapping (ORM)''
  \newline
  \newline
  \noindent
  In der Java Welt gibt es viele ORM Lösungen, welche sich in den Jahren
  durchgesetzt haben, ein paar nennenswerte sind Hibernate, TopLink und
  EclipseLink. Diese Anforderung entfällt also.
  \newline
  
  Folgende Punkte wurden in den Katalog der Soll-Kriterien aufgenommen und somit
  mit einer eindeutigen ID versehen:

  \begin{description}

    % Kapitel 2.2.2, N-Tier Applikationen, Seite 20
    \item[Soll-01 - Zugriffskontrolle
    (Authentifizierung/Authorisation/Rollenverwaltung)\label{itm:Soll-01}]

    Ein Webframework sollte EntwicklerInnen verschiedene Mechanismen
    bereitstellen, um die Anwendung vor fremden und unerlaubten Zugriff schützen
    zu können.

    Authentifizierung/Autorisierung: Üblicherweise werden im Vorfeld Rollen für
    verschiedenen Gruppen festgelegt. Das Ziel einer sicheren Webanwendung ist
    es, bestimmte Bereiche einer Seite abzusichern und die Rechte aller
    Benutzer je nach Rolle einzuschränken. Anhand der Rolle wird deren
    Benutzer für die festgelegten Bereiche autorisiert.

    Vertraulichkeit / Verschlüsselung: Sensible Daten, wie Passwörter und
    Personendaten, müssen vor dem Zugriff und der Kenntnisnahme von Dritten
    geschützt werden. Hierfür werden die Daten während der Übertragung
    verschlüsselt. Für eine sichere Datenübertragung werden meist
    Verschlüsselungsprotokolle wie SSL (Secure Sockets Layer), sowie dessen
    Nachfolger TLS (Transport Layer Security) eingesetzt. Diese gelten als
    relativ sicher und sind bei Transaktionen bei einer Bank unverzichtbar.

    \item[Soll-02 - Form-Validierung\label{itm:Soll-02}]
    Das Verarbeiten von Formularen bzw. die Handhabung von Benutzereingaben und
    -aktionen gehört zu den täglichen Aufgaben der Webentwicklung. Die Logik
    für server- und clientseitiges Validieren ist im Idealfall nur einmal
    implementiert.

    Server-Side Validation: Das Webframework soll die Mögichkeit bieten,
    eingegebene Daten einfach zu überprüfen. Dabei soll für jede Eigenschaft
    eines Datenmodells (Model) ein Wertebereich definierbar sein, zusätzlich
    soll geprüft werden, ob die Eingabe erforderlich ist. Die Programmierlogik
    soll minimal sein. Nach dem Senden der Daten wird alles überprüft und ggf.
    entsprechende Fehlermeldungen zurückgegeben.

    Client-Side Validation: Das Webframework soll wenn möglich schon auf dem
    Client validieren. Wenn möglich, sollen die Daten gleich bei oder kurz nach
    der Eingabe überprüft werden, wobei die Logik auf dem Server implementiert
    ist. Beispiel: Ist der Anmeldename schon vergeben. Dadurch werden
    Serverressourcen gespart und der Benutzer hat ein direktes Feedback.

    \item[Soll-03 - Modulare Architektur\label{itm:Soll-03}]

    Eine Webanwendung stellt ein Zusammenspiel verschiedenster
    Internettechnologien dar, die wiederum hinsichtlich ihrer Entwicklung einem
    stetigen Wandel unterliegen. Die Herausforderung für die Entwickler ist es,
    die Entwicklung der Technologien im Auge zu behalten um gegebenenfalls
    Neuheiten oder Änderungen im System anzupassen. Eine Webanwendung sollte
    daher in all ihren Bestandteilen möglichst wartbar bleiben.

    \item[Soll-04 - Schnittstellen und Webservices\label{itm:Soll-04}]

    Interoperabilität beschreibt den Austausch von Informationen verschiedener
    Softwaresysteme. Es gibt verschiedene Technologien, mit denen ein
    Informationsaustausch umgesetzt werden kann - REST, SOAP und RPC sind am
    weitesten verbreitet. Das Webframework sollte Mechanismen und Funktionen
    zur Umsetzung von Schnittstellen bereitstellen.

    \item[Soll-05 - Testing\label{itm:Soll-05}]

    Testing ist mit test-driven development (TDD), vor allem in der agilen
    Softwareentwicklung, ein fester Bestandteil während der Projektentwicklung.
    Vor der Implementierung überprüft der Programmierer, mittels Unit-Tests, 
    konsequent das Verhalten jeglicher Komponenten. Gerade kritische Prozesse
    und Transaktionen (zum Beispiel eine Banküberweisung) sollten ausgiebig
    getestet werden. Das Webframework soll die Möglichkeit von Unit-Tests
    bieten.

    \item[Soll-06 - Internationalisierung und Lokalisierung\label{itm:Soll-06}]

    Viele Webanwendungen richten sich mittlerweile an ein internationales
    Publikum. Dank der Offenheit und der weiten Verbreitung des Internets lassen
    sich dadurch sehr einfach neue Zielgruppen (BenutzerInnen) erschließen.
    Jedoch gilt es, einige Vorraussetzungen und Besonderheiten bei der
    Internationalisierung von Webanwendungen zu beachten. Zunächst müssen
    sämtliche Texte übersetzt und möglicherweise in neue Datenbanken ausgelagert
    werden. Hierbei ist zu berücksichtigen, dass es länderspezifische
    Zeichensätze, Zahlen, Datum und Währungswerte gibt. Hinzu kommt, dass unter
    Umständen auch spezielle Grafiken neu erstellt werden müssen.

    \item[Soll-07 - Scaffolding / Rapid Prototyping\label{itm:Soll-07}]

    Mit Scaffolding ist das automatische Generien von den sogenannten CRUD-Pages
    (Create, Read, Update, Delete) gemeint.

    Scaffolding und das dadurch verstandene Rapid Prototyping ist ein
    wesentlicher Aspekt der agilen Softwareentwicklung (zum Beispiel in
    Verwendung mit der Scrum-Methodik). Gerade zu Beginn eines Projekts eignet
    sich das Rapid Prototyping, da Änderungen in den Models umgehend in die
    Views eingebunden werden können.

    \item[Soll-08 - Caching\label{itm:Soll-08}]

    Neben der Übertragunsgeschwindigkeit gibt es eine Menge anderer Faktoren,
    die für eine leistungsstarke Webanwendung entscheidend sind. Ein Aspekt ist
    das Caching - das Zwischenspeichern von Daten, die häufig verwendet bzw.
    aufgerufen werden.

    Das Caching kann meist auf verschiedenen Ebenen implementiert werden: Auf
    dem Client, auf dem Webserver und auf der Datenbankebene. Dabei soll das
    Webframework diese Mechanismen unterstützen und es ermöglichen, diese
    einfach zu aktivieren und zu kalibrieren.  

    \item[Soll-09 - View-Engine\label{itm:Soll-09}]

    View Engines werden eingesetzt, um das Arbeiten mit den Views zu
    erleichtern. Sie unterstützen vor allem das Templating - das Erstellen von
    Vorlagen. Durch typisierte Views wird eine Webanwendung robuster, weil
    weniger fehleranfällig; durch partial Views (oft auch nur als partials
    bezeichnet) werden einzelne Elemente oder Bereiche der Benutzeroberfläche
    wiederverwendbar gemacht. Diese Vorlagen können von mehreren Views genutzt
    werden. Dadurch wird deutlich weniger redundanter Code erstellt. Dies ist
    ein wichtiger Aspekt in Bezug auf das Don’t repeat yourself (DRY) Prinzip.

    \item[Soll-10 - Dokumentation\label{itm:Soll-10}]

    Eine Webanwendung ist aus Entwicklersicht eine Zusammenfassung
    verschiedenster Technologien (Webframeworks, Bibliotheken, Schnittstellen).
    Über das Application Programming Interface (API) haben Programmierer Zugriff
    auf die verfügbaren Funktionen. Nicht selten erstreckt sich die
    Dokumentation einer API über mehrere hundert Seiten. Für Programmierer ist
    es daher von enormer Bedeutung, dass die Bibliotheken gut strukturiert und
    verständlich beschrieben sind. Schlecht oder gar nicht dokumentierte
    Technologien erhöhen die Fehlerquote und sind oft ausschlaggebend für einen
    nicht flüssigen Workflow.

    \item[Soll-11 - Community\label{itm:Soll-11}]

    Die beteiligten Personen in den Foren, Mailing-Listen oder Wikis bilden im
    Zusammenhang mit Webframeworks die Community. Es findet dabei ein
    Wissensaustausch statt, der weit über die standard-Dokumentation hinaus
    geht. Die Nutzer helfen sich gegenseitig bei Problemen und Fehlern und oft
    hinterlassen sie mit ihren Einträgen wiederum einen Lösungsansatz, der
    zukünftig von anderen wieder aufgegriffen werden kann. Es ist daher 
    wichtig, dass den Anwendern eine Möglichkeit geboten wird, sich
    auszutauschen, gemeinsam Fehlermeldungen zu deuten und Lösungsansätze zu
    entwickeln.

    \item[Soll-12 - IDE-Unterstützung\label{itm:Soll-12}]

    In der Softwareentwicklung ist die IDE das Basiswerkzeug für die
    Programmierer. Die Entwicklungsumgebung stellt den Entwicklern verschiedene
    Komponenten zur Verfügung, wie Editor, Compiler, Linker oder Debugger.
    Hinzu kommen mit Syntaxhighlighting, Refactoring und Code-Formatierung
    weitere wichtige Funktionen, die die Entwickler in vielerlei Hinsicht enorm
    unterstützen.

    \item[Soll-13 - Kosten für Entwicklungswerkzeuge\label{itm:Soll-13}]

    Je nach verfügbaren finanziellen Mitteln spielen die Kosten von
    Entwicklungswerkzeugen und Technologien durchaus eine Rolle. Bei beiden
    Faktoren hat man meist die Wahl zwischen kostenlosen (OpenSource) und
    kommerziellen Produkten. Je nach Webanwendung müssen somit Lizenzgebühren
    für die Entwicklungswerkzeuge, Server zum Ausführen der Anwendung und
    Datenbankserver berücksichtigt werden. Dabei ist es wichtig die richtige
    Mischung verschiedener Komponenten zu finden.
  
    \item[Soll-14 - Eignung für agile Entwicklung\label{itm:Soll-14}]

    Die herkömmlichen Methoden der Software-Entwicklung werden heute oft durch
    neue agile Methoden, wie Extreme Programming oder Scrum abgelöst. Sie
    fokussieren auf das Wesentliche und stehen für deutlich mehr Flexibilität in
    der Entwicklungsphase als konventionelle Methoden. Verschiedene Technologien
    unterstützen die agilen Methoden. Refactoring, Testing spielt dabei eine
    wichtige Rolle und soll unterstützt werden.

    \item[Soll-15 - Lernkurve für EntwicklerInnen\label{itm:Soll-15}]

    Zur Umsetzung einer komplexen Webanwendung wird den Entwicklern ein Wissen
    über verschiedenste Bereiche der Softwareentwicklung abverlangt.
    Glücklicherweise gibt es für die Webentwicklung keinen einheitlichen
    Standard, der festlegt, wie eine Webanwendung entwickelt werden muss. Das
    Internet bietet für jeden Bereich eine Auswahl unterschiedlicher
    Technologien an. Daher müssen vor der Entwicklung mehrere Entscheidungen
    getroffen werden - hinsichtlich Programmiersprache, Javascript-Framework
    oder Datenbankserver. Es werden daher oft Technologien gewählt, die leicht
    zu erlernen und verstehen sind.

    Wichtig ist, dass man einen schnellen Einstieg bekommt und Erfolge bald
    sichtbar werden, um die Motivation der Entwickler zu erhöhen.

    \item[Soll-16 - AJAX-Unterstützung\label{itm:Soll-16}]

    Durch das Aufkommen von Javascript-Bibliotheken wie jQuery und Prototype
    haben sich vollkommen neue Möglichkeiten eröffnet mit Javascript auf dem
    Client zu arbeiten und mit AJAX wurde die Kommunikation zwischen Server und
    Client im Web revolutioniert. Die direkte Unterstützung eines
    Javascript-Frameworks ist für ein Webframework sinnvoll und erwünscht.

    Darüber hinaus sollte die unterstützte Bibliothek lose gekoppelt sein, und
    damit austauschbar. Sogenanntes unobtrusive Javascript, bei dem auch bei
    abgeschaltetem Javascript die Anwendung funktioniert, ist auch eine Anforderung.
  \end{description}    
    
  \subsection{KO-Kriterien}
  
  KO-Kriterien sind Vorgaben, welche zwingend erfüllt sein müssen. Falls ein
  Kriterium nicht erfüllt werden kann, fällt die Entscheidung auf diese
  Alternative negativ aus. Jedem KO-Kriterium wird für die Identifikation eine
  eindeutige ID vergeben. Die ID setzt sich folgendermassen zusammen.
  \{KO\}-\{Laufnummer\}.
  
  \subsubsection{Grundsätze aus der IT-Architektur der Zürcher Kantonalbank}
  
  Folgende KO-Kriterien sind aus dem \begin{itshape}Handbuch der
  IT-Architektur\end{itshape}\cite{ZkbHandbuchDerItArchitektur} der Züricher
  Kantonalbank entnommen. Die Namensgebung unterscheidet sich leicht, es wird
  nicht von KO-Kriterien, sondern von Grundsätzen gesprochen. Ein Grundsatz
  wird wie folgt definiert:\\
  
  ``\begin{itshape}Es sind Grundsätze definiert, nach denen sich die Baupläne
  der IT-Systeme zu richten haben. Die Grundsätze sind ein Regelwerk mit
  Weisungscharakter.\end{itshape}''
  \footnote{\cite{ZkbHandbuchDerItArchitektur} Kapitel 1.3 - \begin{itshape}Was
  ist die IT-Architektur der ZKB\end{itshape}, Seite 11}
  \\
  \\
  \noindent
  Dabei gibt es eine Hintertür:\\

  ``\begin{itshape}Grundsätze sind verbindliche Vorgaben (Konventionen), von
  denen nur in begründeten Ausnahmen abgewichen werden kann.\end{itshape}''
  \footnote{\cite{ZkbHandbuchDerItArchitektur} Kapitel 1.8 -
  \begin{itshape}Leseanleitung\end{itshape}, Seite 14}
  \\
  \\
  \noindent
  Das Dokument wurde analysiert und die Grundsätze, welche für diese
  Diplomarbeit relevanten sind, werden hier aufgelistet:
  
  \begin{description}

    % Kapitel 2.2.2, N-Tier Applikationen, Seite 20
    \item[KO-01\label{itm:KO-01}]
    \footnote{\cite{ZkbHandbuchDerItArchitektur} Kapitel 2.2.2 -
    \begin{itshape}N-Tier Applikationen\end{itshape}, Seite 20}
    Applikationen sollen als N-Tier Applikationen designed und
    implementiert werden.

    % Kapitel 2.2.10, Objektorientierung, Seite 22
    \item[KO-02\label{itm:KO-02}]
    \footnote{\cite{ZkbHandbuchDerItArchitektur} Kapitel 2.2.10 -
    \begin{itshape}Objektorientierung\end{itshape}, Seite 22}
    \ac{OO} soll innerhalb der Informatik für Neuentwicklungen
    durchgängig angewandt werden.

    % Kapitel 3.3, Mehrsprachigkeit, Seite 36
    \item[KO-03\label{itm:KO-03}]
    \footnote{\cite{ZkbHandbuchDerItArchitektur} Kapitel 3.3 -
    \begin{itshape}Mehrsprachigkeit\end{itshape}, Seite 36}
    Eine neue Applikation (oder eine neue Komponente einer
    bestehenden Applikation) ist mehrsprachfähig zu realisieren.

    % Kapitel 3.3, Mehrsprachigkeit, Seite 37
    \item[KO-04\label{itm:KO-04}]
    \footnote{\cite{ZkbHandbuchDerItArchitektur} Kapitel 3.3 -
    \begin{itshape}Mehrsprachigkeit\end{itshape}, Seite 37}
    Neue Applikationen sind Unicode-fähig zu realisieren.

    % Kapitel 3.9.1, Skalierbarkeit / Ausfallsicherheit / Perfomance
    % Seite 49
    \item[KO-05\label{itm:KO-05}]
    \footnote{\cite{ZkbHandbuchDerItArchitektur} Kapitel 3.9.1 -
    \begin{itshape}Skalierbarkeit / Ausfallsicherheit / Perfomance\end{itshape},
    Seite 49}
    Eine Applikation muss in mehreren Instanzen lauffähig sein.

    % Kapitel 4.4.1, User Interface, Seite 55
    \item[KO-06\label{itm:KO-06}]
    \footnote{\cite{ZkbHandbuchDerItArchitektur} Kapitel 4.4.1 -
    \begin{itshape}User Interface\end{itshape}, Seite 55}
    Der ZKB GUI Style Guide ist in allen ZKB IT-Projekten anzuwenden.
    
    % Kapitel 5.2, Verwendung der Zentralen Server Infrastruktur ZSI, Seite 57
    \item[KO-07\label{itm:KO-07}]
    \footnote{\cite{ZkbHandbuchDerItArchitektur} Kapitel 5.2 -
    \begin{itshape}Verwendung der Zentralen Server Infrastruktur
    ZSI\end{itshape}, Seite 57}
    Die \ac{ZSI} ist als Server-Plattform für Applikationen, welche Windows-,
    Unix-, Linux-basierte Server einsetzen, zu verwenden.
    
    % Kapitel 9.2, Java RMI, Seite 71
    \item[KO-08\label{itm:KO-08}]
    \footnote{\cite{ZkbHandbuchDerItArchitektur} Kapitel 9.2 -
    \begin{itshape}Java RMI\end{itshape}, Seite 71}
    RMI kann für reine Java-Anwendungen eingesetzt werden.
    
    % Kapitel 12.2.1, Einsatz von Frameworks, Seite 140
    \item[KO-09\label{itm:KO-09}]
    \footnote{\cite{ZkbHandbuchDerItArchitektur} Kapitel 12.2.1 -
    \begin{itshape}Einsatz von Frameworks\end{itshape}, Seite 140}
    Für Java-Applikationen (Internet, Extranet und Intranet) wird das
    ZIP-Framework eingesetzt.
    
    % Kapitel 12.3.5, Client-/Server-Schemata von Internet-Applikationen,
    % Seite 141
    \item[KO-10\label{itm:KO-10}]
    \footnote{\cite{ZkbHandbuchDerItArchitektur} Kapitel 12.3.5 -
    \begin{itshape}Client-/Server-Schemata von Internet-Applikationen\end{itshape}, Seite 141}
    Die Validierung und Plausibilisierung der Eingaben erfolgt immer
    abschliessend auf dem bankseitigen Applikations-Server. Es ist aber
    durchaus möglich, dass sich auf der Client-Seite eine Logik zur Überprüfung
    und Validierung der Eingaben für den Benutzerkomfort befindet.
    
    % Kapitel 12.3.5, Client-/Server-Schemata von Internet-Applikationen,
    % Seite 141
    \item[KO-11\label{itm:KO-11}]
    \footnote{\cite{ZkbHandbuchDerItArchitektur} Kapitel 12.3.5 -
    \begin{itshape}Client-/Server-Schemata von Internet-Applikationen\end{itshape}, Seite 141}
    Die Business-Logik in einer Internet-Applikation ist so auszulegen, dass
    diese von verschiedenen Präsentations-Logiken im Rahmen von Ultra-Thin- und
    Thin-Client-Applikationen genutzt werden kann.
    
    % Kapitel 12.3.6.1, Browser-Abhängigkeiten, Seite 142
    \item[KO-12\label{itm:KO-12}]
    \footnote{\cite{ZkbHandbuchDerItArchitektur} Kapitel 12.3.6.1 -
    \begin{itshape}Browser-Abhängigkeiten\end{itshape}, Seite 142}
    Die Internet-Applikationen der ZKB werden nicht mit Browser-Abhängigkeiten
    versehen und orientieren sich an den neutralen Standards der W3C-Komission.
    
    % Kapitel 12.3.6.2, Session-Mechanismus, Seite 142
    \item[KO-13\label{itm:KO-13}]
    \footnote{\cite{ZkbHandbuchDerItArchitektur} Kapitel 12.3.6.2 -
    \begin{itshape}Session-Mechanismus\end{itshape}, Seite 142}
    Für Ultra-Thin-Client-Applikationen wird als Session-Mechanismus die
    Cookie- oder die URL-Rewriting-Methode angewendet.
    
    % Kapitel 12.3.6.4, Einfache Internet-Applikationen, Seite 143
    \item[KO-14\label{itm:KO-14}]
    \footnote{\cite{ZkbHandbuchDerItArchitektur} Kapitel 12.3.6.4 -
    \begin{itshape}Einfache Internet-Applikationen\end{itshape}, Seite 143}
    Einfache Internet-Applikationen mit dem Schwerpunkt Information können
    ausschliesslich Java Server Pages verwenden.
    
    % Kapitel 12.3.6.5, Komplexe Internet-Applikationen, Seite 143
    \item[KO-15\label{itm:KO-15}]
    \footnote{\cite{ZkbHandbuchDerItArchitektur} Kapitel 12.3.6.5 -
    \begin{itshape}Komplexe Internet-Applikationen\end{itshape}, Seite 143}
    Komplexe Internet-Applikationen verwenden eine Kombination von Java Server
    Pages und mindestens einem Servlet als Dispatcher-Mechanismus.
    
    % Kapitel 12.3.6.6, Verwendung von JavaScript beziehungsweise ECMAScript
    % Seite 143
    \item[KO-16\label{itm:KO-16}]
    \footnote{\cite{ZkbHandbuchDerItArchitektur} Kapitel 12.3.6.6 -
    \begin{itshape}Verwendung von JavaScript beziehungsweise
    ECMAScript\end{itshape}, Seite 143}
    Die Internet-Applikationen funktionieren auch eingeschränkt, ohne dass die
    Skript-Funktion im Browser aktiviert ist.
    
    % Kapitel 12.3.6.7, Einstz von ActiveX und Cookies, Seite 143
    \item[KO-17\label{itm:KO-17}]
    \footnote{\cite{ZkbHandbuchDerItArchitektur} Kapitel 12.3.6.6 -
    \begin{itshape}Verwendung von JavaScript beziehungsweise
    ECMAScript\end{itshape}, Seite 143}
    ActiveX wird wegen der Möglichkeit für direkte Zugriffe auf das
    Betriebssystem nicht eingesetzt.
    
    % Kapitel 12.3.6.8, Applets, Seite 144
    \item[KO-18\label{itm:KO-18}]
    \footnote{\cite{ZkbHandbuchDerItArchitektur} Kapitel 12.3.6.8 -
    \begin{itshape}Applets\end{itshape}, Seite 144}
    Es werden keine neuen Applikationen als Java Applets entwickelt. Der
    Einsatz von Applets beschränkt sich auf einfache Funktionen wie Börsen-
    oder News-Ticker.
    
    % Kapitel 12.3.6.9, Browser-Plugins, Seite 144
    \item[KO-19\label{itm:KO-19}]
    \footnote{\cite{ZkbHandbuchDerItArchitektur} Kapitel 12.3.6.9 -
    \begin{itshape}Browser-Plugins\end{itshape}, Seite 144}
    Internet-Applikation werden ohne Plugins entwickelt.
    
    % Kapitel 12.3.7, Client-Technologien, Seite 144
    \item[KO-20\label{itm:KO-20}]
    \footnote{\cite{ZkbHandbuchDerItArchitektur} Kapitel 12.3.7 -
    \begin{itshape}Client-Technologien\end{itshape}, Seite 144}
    Für Ultra Thin Clients bzw. Browser-basierende Applikationen muss das
    aktuelle, Struts-basierende HTML-Client-Framework der ZKB Internet
    Plattform verwendet werden.
    
    % Kapitel 12.3.8, Layering von Internet-Applikationen, Seite 145
    \item[KO-21\label{itm:KO-21}]
    \footnote{\cite{ZkbHandbuchDerItArchitektur} Kapitel 12.3.8 -
    \begin{itshape}Layering von Internet-Applikationen\end{itshape}, Seite 145}
    Neue Internet- und Extranet-Applikationen müssen sich an das Layering
    gemäss nachfolgender Grafik @@@@@@@@@ HIER NOCH DIE GRAFIK REFERENZIEREN
    @@@@@@@ halten. Für Intranet-Applikationen ist die Validator-Komponente
    fakultativ.
    
    % Kapitel 12.3.8, Layering von Internet-Applikationen, Seite 146
    \item[KO-22\label{itm:KO-22}]
    \footnote{\cite{ZkbHandbuchDerItArchitektur} Kapitel 12.3.8 -
    \begin{itshape}Layering von Internet-Applikationen\end{itshape}, Seite 146}
    Eine Applikation muss ohne Änderung von der Intranet-Anwendung zur Extra-
    oder Internet-Applikation gemacht werden können. Es geschieht dies
    lediglich durch das Vorschalten der Validator-Komponente.
    
    % Kapitel 12.3.10.2, Web Server, Seite 146
    \item[KO-23\label{itm:KO-23}]
    \footnote{\cite{ZkbHandbuchDerItArchitektur} Kapitel 12.3.10.2 -
    \begin{itshape}Web Server\end{itshape}, Seite 146}
    Der ZKB Standard-Web-Server ist der Apache HTTP-Server.
    
    % Kapitel 12.4.2, Architektur beim Einsatz von Application-Servern,
    % Seite 147
    \item[KO-24\label{itm:KO-24}]
    \footnote{\cite{ZkbHandbuchDerItArchitektur} Kapitel 12.4.2 -
    \begin{itshape}Architektur beim Einsatz von
    Application-Servern\end{itshape}, Seite 147}
    Der Application-Server wird als die integrierte technische Middleware für
    die Unterstützung von Java Server Pages (JSP), Servlets, Enterprise Java
    Beans (EJB) und der sicheren Kommunikation zwischen Client und Server
    eingesetzt.
    
    % Kapitel 12.4.2, Architektur beim Einsatz von Application-Servern,
    % Seite 148
    \item[KO-25\label{itm:KO-25}]
    \footnote{\cite{ZkbHandbuchDerItArchitektur} Kapitel 12.4.2 -
    \begin{itshape}Architektur beim Einsatz von
    Application-Servern\end{itshape}, Seite 148}
    Der ZKB Standard-J2EE-Application-Server ist der JBoss Application Server.
    
    % Kapitel 12.4.2, Architektur beim Einsatz von Application-Servern,
    % Seite 148
    \item[KO-26\label{itm:KO-26}]
    \footnote{\cite{ZkbHandbuchDerItArchitektur} Kapitel 12.4.2 -
    \begin{itshape}Architektur beim Einsatz von
    Application-Servern\end{itshape}, Seite 148}
    Der J2EE-Application-Server wird in der \ac{ZSI} eingesetzt.
    
    % Kapitel 12.4.2, Architektur beim Einsatz von Application-Servern,
    % Seite 148
    \item[KO-27\label{itm:KO-27}]
    \footnote{\cite{ZkbHandbuchDerItArchitektur} Kapitel 12.4.2 -
    \begin{itshape}Architektur beim Einsatz von
    Application-Servern\end{itshape}, Seite 148}
    Die Serverplattform für den Einsatz von Web-Application-Servern für
    Internet-/Intranet-/Extranet-Applikationen ist Linux.
    
    % Kapitel 12.4.2, Architektur beim Einsatz von Application-Servern,
    % Seite 150
    \item[KO-28\label{itm:KO-28}]
    \footnote{\cite{ZkbHandbuchDerItArchitektur} Kapitel 12.4.2 -
    \begin{itshape}Architektur beim Einsatz von
    Application-Servern\end{itshape}, Seite 150}
    Die technischen Services wie Session Management, Load Balancing,
    Transaction Management und Instance Pooling werden vom J2EE Application
    Server zur Verfügung gestellt.
    
    % Kapitel 12.4.3, Einsatz von Enterprise Java Beans, Seite 150
    \item[KO-29\label{itm:KO-29}]
    \footnote{\cite{ZkbHandbuchDerItArchitektur} Kapitel 12.4.3 -
    \begin{itshape}Einsatz von Enterprise Java Beans\end{itshape}, Seite 150}
    Das Muster JSP/Servlets mit EJBs ist anzuwenden, wenn die Applikation eine
    umfangreiche, komplexe Business-Logik aufweist, die Business-Logik
    wiederverwendbar sein soll, mehrere unterschiedliche Clients (Browser
    (Ultra-Thin-)(HTML), Thin-(Java), Mobile, …) mit einer Business Logik
    bedient werden müssen, hohe Anforderungen an die Skalierbarkeit gestellt
    werden und ein lange Lebenszyklus der Applikation erwartet wird.
    
    % Kapitel 12.4.3, Einsatz von Enterprise Java Beans, Seite 150
    \item[KO-30\label{itm:KO-30}]
    \footnote{\cite{ZkbHandbuchDerItArchitektur} Kapitel 12.4.3 -
    \begin{itshape}Einsatz von Enterprise Java Beans\end{itshape}, Seite 150}
    Das Muster JSP/Servlets ohne EJBs ist anzuwenden, wenn die Applikation eine
    einfache Business-Logik aufweist, nur einen Client, zum Beispiel ein
    Browser (Ultra-Thin-)-Interface unterstützt, niedrige Anforderungen an die
    Skalierbarkeit stellt und nur eine vergleichsweise kurzer Lebenszyklus der
    Applikation erwartet wird.
    
    % Kapitel 12.9.2, Standards, Seite 171
    \item[KO-31\label{itm:KO-31}]
    \footnote{\cite{ZkbHandbuchDerItArchitektur} Kapitel 12.9.2 -
    \begin{itshape}Standards\end{itshape}, Seite 171}
    In der ZKB werden folgende Standards für Web Services eingesetzt: SOAP,
    WSDL, W3C X Schema (XSD, WXS).
    
    % Kapitel 13.1, Client/Server-Konzept, Seite 175
    \item[KO-32\label{itm:KO-32}]
    \footnote{\cite{ZkbHandbuchDerItArchitektur} Kapitel 13.1 -
    \begin{itshape}Client/Server-Konzept\end{itshape}, Seite 175}
    Alle Neuentwicklungen sind konsequent in Client-/Server-Komponenten
    aufzuteilen.
    
    % Kapitel 13.2, Anwendungsstruktur (MVC/Model, View, Controller), Seite 175
    \item[KO-33\label{itm:KO-33}]
    \footnote{\cite{ZkbHandbuchDerItArchitektur} Kapitel 13.2 -
    \begin{itshape}Anwendungsstruktur (MVC/Model, View,
    Controller)\end{itshape}, Seite 175}
    Die gewünschte Isolation der Systemteile wird durch eine Anwendungsstruktur
    mit Trennung in Model (Verarbeitung, Datenhaltung), View
    (Benutzeroberfläche) und Controller (Organisation, Ereignisvermittlung)
    erreicht.
    
    % Kapitel 13.9.7.2, Allgemeine Grundsätze, Seite 189
    \item[KO-34\label{itm:KO-34}]
    \footnote{\cite{ZkbHandbuchDerItArchitektur} Kapitel 13.9.7.2 -
    \begin{itshape}Allgemeine Grundsätze\end{itshape}, Seite 189}
    Jede Applikation ist dafür verantwortlich, dass diejenigen Daten, für die
    sie den Lead hat, validiert sind.
    
    % Kapitel 13.9.7.2, Allgemeine Grundsätze, Seite 189
    \item[KO-35\label{itm:KO-35}]
    \footnote{\cite{ZkbHandbuchDerItArchitektur} Kapitel 13.9.7.2 -
    \begin{itshape}Allgemeine Grundsätze\end{itshape}, Seite 189}
    Jede Applikation benutzt Datenvalidierung um sicherzustellen, dass sie die
    erhaltenen Daten verarbeiten kann und dass ihr Betrieb nicht gefährdet ist
    (Stabilität / Verfügbarkeit).
    
    % Kapitel 13.9.7.3.1, Datenvalidierung an der Benutzersczhnittstelle
    % Seite 191
    \item[KO-36\label{itm:KO-36}]
    \footnote{\cite{ZkbHandbuchDerItArchitektur} Kapitel 13.9.7.3.1 -
    \begin{itshape}Datenvalidierung an der Benutzersczhnittstelle\end{itshape},
    Seite 191}
    Benutzereingaben werden so früh wie möglich validiert.
    
    % Kapitel 13.10, Programmiersprachen und Entwicklungsumgebungen
    % Seite 192
    \item[KO-37\label{itm:KO-37}]
    \footnote{\cite{ZkbHandbuchDerItArchitektur} Kapitel 13.10 -
    \begin{itshape}Programmiersprachen und Entwicklungsumgebungen\end{itshape},
    Seite 192}
    Für die Entwicklung neuer Applikationen und beim neuen Design bestehender
    Applikationen wird Java eingesetzt.
    
    % Kapitel 13.10.4.8, Zusätzliche Java Klassenbibliotheken, Seite 195
    \item[KO-38\label{itm:KO-38}]
    \footnote{\cite{ZkbHandbuchDerItArchitektur} Kapitel 13.10.4.8 -
    \begin{itshape}Zusätzliche Java Klassenbibliotheken\end{itshape}, Seite 195}
    Zusätzliche Java-Klassenbibliotheken, also solche, die nicht im JDK
    enthalten sind, werden nur in begründeten Ausnahmen eingesetzt.
    Normalerweise müssen solche Bibliotheken dem 100\%-Pure-Java-Grundsatz
    entsprechen. Ausnahmen können systemnahe Funktionen für Security und
    dergleichen sein.
    
    % Kapitel 13.11, Einsatz von .NET-basierten Applikationen, Seite 196
    \item[KO-39\label{itm:KO-39}]
    \footnote{\cite{ZkbHandbuchDerItArchitektur} Kapitel 13.11 -
    \begin{itshape}Einsatz von .NET-basierten Applikationen\end{itshape}, Seite
    196}
    .NET-basierte Applikationen werden in der ZKB Informatik nicht entwickelt
    oder zur Entwicklung in Auftrag gegeben ausser Kleinapplikationen der
    Kategorie „Office“.
    
    % Kapitel 13.13, Einsatz von Open Source Software, Seite 206
    \item[KO-40\label{itm:KO-40}]
    \footnote{\cite{ZkbHandbuchDerItArchitektur} Kapitel 13.13 -
    \begin{itshape}Einsatz von Open Source Software\end{itshape}, Seite 206}
    Die Nutzung von Open Source Software ist erlaubt.
    
    % Kapitel 13.13, Einsatz von Open Source Software, Seite 206
    \item[KO-41\label{itm:KO-41}]
    \footnote{\cite{ZkbHandbuchDerItArchitektur} Kapitel 13.13 -
    \begin{itshape}Einsatz von Open Source Software\end{itshape}, Seite 206}
    Open Source Software unterliegt denselben Kriterien wie kommerzielle
    Software. Sie muss evaluiert, registriert, homologiert, intern supported
    und gepflegt werden.    
    
    % Kapitel 13.13.1, Kriterien für die Evaluation von Open Source
    % Software, Seite 207
    \item[KO-42\label{itm:KO-42}]
    \footnote{\cite{ZkbHandbuchDerItArchitektur} Kapitel 13.13.1 -
    \begin{itshape}Kriterien für die Evaluation von Open Source
    Software\end{itshape}, Seite 207}
    Produktiv eingesetzte Open Source Software muss durch angemessene
    Informationsquellen unterstützt sein (Newsgroups, Mailinglists,
    FAQ-Listen, WebSites, User Groups).    
    
    % Kapitel 13.13.1, Kriterien für die Evaluation von Open Source
    % Software, Seite 207
    \item[KO-43\label{itm:KO-43}]
    \footnote{\cite{ZkbHandbuchDerItArchitektur} Kapitel 14.14.1 -
    \begin{itshape}Kriterien für die Evaluation von Open Source
    Software\end{itshape}, Seite 207}
    Die Weiterentwicklung von produktiv eingesetzter Open Source Software
    ausserhalb der ZKB muss öffentlich einsehbar sein und aktiv verfolgt werden
    können.
    
    % Kapitel 13.13.1, Kriterien für die Evaluation von Open Source
    % Software, Seite 207
    \item[KO-44\label{itm:KO-44}]
    \footnote{\cite{ZkbHandbuchDerItArchitektur} Kapitel 13.13.1 -
    \begin{itshape}Kriterien für die Evaluation von Open Source
    Software\end{itshape}, Seite 207}
    Die Lizenzbedingungen einer Open Source Software müssen vor dem Einsatz
    geprüft werden und von der Zürcher Kantonalbank akzeptiert werden können.
  \end{description}

  \section{Hirarchie der Anforderungen}
  
  \section{Auswahl der Frameworks}
  
  \section{Gewichtete Nutzwertanalyse mit Analytic Hirarchy Process}
  
  \section{Resultat}
