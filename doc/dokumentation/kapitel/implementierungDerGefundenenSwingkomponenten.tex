In diesem Kapitel soll behandelt werden, ob eine Implementierung, der im Kapitel 
\ref{chapter:AnalyseDerJavaSwingApplikationen}
(\nameref{chapter:AnalyseDerJavaSwingApplikationen}, S.
\pageref{chapter:AnalyseDerJavaSwingApplikationen}ff) gefundenen Java Swing
Komponenten, äquivalent möglich ist. Dabei sollen die Dokumentationen und
Tutorials, welche für die Frameworks vorhanden sind, als Grundlage dienen. Die
Prüfung soll durch einen visuellen Vergleich statt finden, ob die 32 gefundenen
Komponenten und die sieben gefundenen GUI Paradigmen abgedeckt sind. Das
Resultat soll in Form einer Tabelle mit dem jeweiligen Abdeckungsgrad
dargestellt werden.

Es wird auf die Prüfung der Komponenten vom Framework ``A1 - ULC, Canoo RIA
Suite'' verzichtet, da es aufgrund der Evaluation als mögliche Alternative
entfällt.

\section{A2 - Struts 1.3.10 mit ZIP-Framework}

\section{A3 - Vaadin 6.5.7}

In der Dokumentation zu Vaadin gibt es eine komplette Übersicht aller
unterstützten Komponenten, siehe \cite{VaadinKomponenten}. Diese können online
mit einem Browser betrachtet und auf deren Usability ausprobiert werden.
Folgende Komponenten werden unterstützt:

\begin{itemize}
  \item Es werden alle Top-Level Komponenten unterstützt
  \item Es werden alle Intermediate-Komponenten unterstützt
  \item Es werden alle Atomic-Komponenten bis auf das
  \(JFormattedTextField\) unterstützt.
  \item Es werden alle speziellen Komponenten bis auf
  \(JFreeChart\) unterstützt. Dafür gibt es ein offizielles
  Add-on mit dem Namen ``JFreeChart wrapper for Vaadin'', welches die
  Unterstützung von \(JFreeChart\) in den Formaten \ac{PNG} und \ac{SVG}
  sichert.
  \item Die ``Neuen'' Komponenten können alle implementiert werden. Das
  Akkordeon und der Zeit-Panel werden schon von Haus aus unterstützt.
\end{itemize}

\noindent
Folgende Design-Patterns werden unterstützt:

\begin{itemize}
  \item Das MVC-Pattern wird unterstützt
  \item Das Observer-Pattern wird unterstützt
  \item Tabellen-Filter werden unterstützt
  \item Duch die Unterstützung von Ajax ist das Worker-Pattern abgedeckt.
\end{itemize}

\noindent
Es werden alle Komponenten bis auf das \(JFormattedTextField\) und
\(JFreeChart\)\footnote{Es ist zu bedenken, dass für das \(JFreeChart\) ein
Add-on existiert, diese Komponente wurde nicht in die Abdeckung
miteingerechnet.} unterstützt, das ist eine Abdeckung von 93.75\%. Es werden
alle GUI Paradigmen unterstützt, das ist eine Abdeckung von 100.00\%.

\section{A4 - Apache Wicket 1.4.17}

\section{Resultat}

In der Tabelle \ref{tab:unterstuetztungDerKomponenten} wird die Abdeckung der
unterstützten Komponenten und GUI Paradigmen dargestellt.
\newline

\begin{table}[!h]
  \sffamily 
  \begin{center}
    \begin{tabular}{lrr}
      \toprule
      \textbf{Java Web Framework} & \textbf{GUI Komponenten} & \textbf{GUI
      Paradigmen}\\
      \midrule
      A1 - ULC, Canoo RIA Suite & Prüfung entfällt & Prüfung entfällt\\
      A2 - Struts 1.3.10 mit ZIP-Framework & xx.xx\% & xx.xx\%\\
      A3 - Vaadin 6.5.7 & 93.75\% & 100.00\%\\
      A4 - Apache Wicket 1.4.17 & xx.xx\% & xx.xx\%\\
      \bottomrule
    \end{tabular}
    \caption{Abdeckungsgrad der gefundenen GUI Komponenten und Paradigmen}
    \label{tab:unterstuetztungDerKomponenten}
  \end{center}
\end{table}