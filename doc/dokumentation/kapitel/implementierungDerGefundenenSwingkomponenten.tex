In diesem Kapitel soll behandelt werden, ob eine Implementierung, der im Kapitel 
\ref{chapter:AnalyseDerJavaSwingApplikationen}
(\nameref{chapter:AnalyseDerJavaSwingApplikationen}, S.
\pageref{chapter:AnalyseDerJavaSwingApplikationen}ff) gefundenen Java Swing
Komponenten, äquivalent möglich ist. Dabei sollen die Dokumentationen und
Tutorials, welche für die Frameworks vorhanden sind, als Grundlage dienen. Die
Prüfung soll durch einen visuellen Vergleich statt finden, ob die 32 gefundenen
Komponenten und die sieben gefundenen GUI Paradigmen abgedeckt sind. Das
Resultat soll in Form einer Tabelle mit dem jeweiligen Abdeckungsgrad
dargestellt werden.

Wie im Kapitel \ref{chapter:MethodenZurEntscheidungsfindungBeiEinerEvaluation}
(\nameref{chapter:MethodenZurEntscheidungsfindungBeiEinerEvaluation}, S.
\pageref{chapter:MethodenZurEntscheidungsfindungBeiEinerEvaluation}ff)
beschrieben, soll eine Mindestabdeckung von 80\% erzielt werden. Wenn das nicht
erreicht wird, bedeutet das den Ausschluss des Frameworks aus der Evaluation.

Es wird auf die Prüfung der Komponenten vom Framework ``A-1 - ULC, Canoo RIA
Suite'' verzichtet, da es aufgrund der geprüften KO-Kriterien als mögliche
Alternative entfällt.

\section{A-2 - Struts 1.3.10 mit ZIP-Framework}

In der Dokumentation zu Apache Struts 1.3.10 befindet sich eine Übersicht der
unterstützten Komponenten, siehe \cite{StrutsComponentes} und
\cite{StrutsHtmlTag}. Es wird schnell klar, dass es sich dabei um die standard
\ac{HTML} Komponenten handelt, welche über eine Tag-Bibliothek abstrahiert
wurden. Folgende Komponenten werden unterstützt:

\begin{itemize}
  \item Struts 1.3.10 kennt keinen Tag um ein Dialogfenster zu öffnen. Dies kann
  bewerkstelligt werden, indem auf JavaScript zurückgegriffen wird. Das wird so
  von Struts 1.3.10 so nicht direkt unterstützt. Somit fehlt bei den Top-Level
  Komponenten der Dialog
  \item Es werden alle Intermediate-Komponenten bis auf den
  \(JTabbedPane\) unterstützt. Der \(JScrollPane\) wird über das ``FrameTag''
  gelöst. \(JPanel\), \(JLayerdPane\) und \(JRootPane\) werden mit dem
  ``HtmlTag'' abgedeckt.
  \item Für folgende Atomic-Komponenten fehlt die Unterstützung:
  \begin{itemize}
    \item \(JFormattedTextField\)
    \item \(JMenuBar\)
    \item \(JMenu\)
    \item \(JMenuItem\)
    \item \(JProgressBar\)
    \item \(JSlider\)
    \item \(JSpinner\)
    \item \(JToolTip\) (wird nur bedingt auf z.B. Images unterstützt)
  \end{itemize}
  \item Für folgende speziellen Komponenten fehlt die Unterstützung:
  \begin{itemize}
    \item \(JFreeChart\)
    \item \(JXBusyLabel\)
    \item \(JXDatePicker\)
  \end{itemize}
  \item Die ``Neuen'' Komponenten können grundsätzlich alle implementiert
  werden.
\end{itemize}

Folgende Design-Patterns werden unterstützt:

\begin{itemize}
  \item Das MVC-Pattern wird unterstützt
  \item Das Observer-Pattern wird nicht unterstützt
  \item Tabellen-Filter werden unterstützt
  \item Duch die fehlende \ac{Ajax} Unterstützung ist das
  Worker-Pattern nicht abgedeckt.
\end{itemize}

Es ist zu erwähnen, das mit der fehlenden \ac{Ajax} Unterstützung das
\ac{GUI} Verhalten einer Java Swing Applikation nicht nachgebaut werden kann.
Alle Views funktionieren nach dem klassischen ``Click und Reload'' Prinzip.

Das ergibt eine Abdeckung der GUI-Komponenten von 59.38\%. Die Abdeckung der
GUI-Paradigmen ist ein wenig höher mit 71.43\%.

\section{A-3 - Vaadin 6.5.7}

In der Dokumentation zu Vaadin gibt es eine komplette Übersicht aller
unterstützten Komponenten, siehe \cite{VaadinKomponenten}. Diese können online
mit einem Browser betrachtet und auf deren Usability ausprobiert werden.
Folgende Komponenten werden unterstützt:

\begin{itemize}
  \item Es werden alle Top-Level Komponenten unterstützt
  \item Es werden alle Intermediate-Komponenten unterstützt
  \item Es werden alle Atomic-Komponenten bis auf das
  \(JFormattedTextField\) unterstützt.
  \item Es werden alle speziellen Komponenten bis auf
  \(JFreeChart\) unterstützt. Dafür gibt es ein offizielles
  Add-on mit dem Namen ``JFreeChart wrapper for Vaadin'', welches die
  Unterstützung von \(JFreeChart\) in den Formaten \ac{PNG} und \ac{SVG}
  sichert.
  \item Die ``Neuen'' Komponenten können alle implementiert werden. Das
  Akkordeon und der Zeit-Panel werden schon von Haus aus unterstützt.
\end{itemize}

Folgende Design-Patterns werden unterstützt:

\begin{itemize}
  \item Das MVC-Pattern wird unterstützt
  \item Das Observer-Pattern wird unterstützt
  \item Tabellen-Filter werden unterstützt
  \item Duch die Unterstützung von Ajax ist das Worker-Pattern abgedeckt.
\end{itemize}

Es werden alle Komponenten bis auf das \(JFormattedTextField\) und
\(JFreeChart\)\footnote{Es ist zu bedenken, dass für das \(JFreeChart\) ein
Add-on existiert, diese Komponente wurde nicht in die Abdeckung der
GUI-Komponenten miteingerechnet.} unterstützt, das ist eine Abdeckung der
GUI-Komponenten von 93.75\%. Es werden alle GUI-Paradigmen unterstützt, das ist
eine Abdeckung der GUI-Paradigmen von 100.00\%.

\section{A-4 - Apache Wicket 1.4.17}

Es existiert eine Komponentenübersicht für das Framework Apache Wicket 1.4.17,
siehe \cite{WicketComponents}. Die Komponenten basieren auf dem \ac{HTML}
Standard. Es gibt ein Projekt namens Wicketstuff, welches zusätzliche
Komponenten für das Framework zur Verfügung stellt. Diese Komponenten wurden
ebenfalls miteinbezogen, da es sozusagen zu Apache Wicket dazugehört. Folgende
Komponenten werden unterstützt:

\begin{itemize}
  \item Es werden alle Top-Level Komponenten unterstützt
  \item Es werden alle Intermediate-Komponenten unterstützt
  \item Für folgende Atomic-Komponenten fehlt die Unterstützung:
  \begin{itemize}
    \item \(JFormattedTextField\)
    \item \(JSlider\) (würde mit dem Projekt wiquery funktionieren)
  \end{itemize}
  \item Es werden alle speziellen Komponenten unterstützt.
  \item Die ``Neuen'' Komponenten können alle implementiert werden. Das
  Akkordeon wird von Wicketstuff schon unterstützt.
\end{itemize}

Folgende Design-Patterns werden unterstützt:

\begin{itemize}
  \item Das MVC-Pattern wird unterstützt
  \item Das Observer-Pattern wird unterstützt
  \item Tabellen-Filter werden unterstützt
  \item Duch die Unterstützung von Ajax ist das Worker-Pattern abgedeckt.
\end{itemize}

Es werden alle Komponenten bis auf das \(JFormattedTextField\) und
\(JSlider\) unterstützt, das ist eine Abdeckung der GUI-Komponenten von 93.75\%.
Es werden alle GUI-Paradigmen unterstützt, das ist eine Abdeckung der
GUI-Paradigmen von 100.00\%.

\section{Resultat}

In der Tabelle \ref{tab:unterstuetztungDerKomponenten} wird die Abdeckung der
unterstützten GUI-Komponenten und Paradigmen dargestellt. Es werden nur die
Alternativen mit einer Abdeckung von mindestens 80\% im weiteren
Verlauf der Evaluation berücksichtigt. In der Tabelle
\ref{tab:gesamtabdeckungUndObImplementierungMoeglichIst} ist für den
Zweifelsfall, dass die Frameworks den selben Nutzwert ausweisen, die
Gesamtabdeckung berechnet. Zudem ist ersichtlich, ob eine Implementierung der
GUI-Komponenten und GUI-Paradigmen möglich ist, was bedeutet, dass mindestens
80\% abgedeckt sind.

\begin{table}[!h]
  \sffamily 
  \begin{center}
    \begin{tabular}{lrr}
      \toprule
      \textbf{Java Web Framework} & \textbf{GUI-Komponenten} &
      \textbf{GUI-Paradigmen} \\
      \midrule
      A1 - ULC, Canoo RIA Suite & Prüfung entfällt & Prüfung entfällt\\
      A2 - Struts 1.3.10 mit ZIP-Framework & 59.38\% & 71.43\%\\
      A3 - Vaadin 6.5.7 & 93.75\% & 100.00\%\\
      A4 - Apache Wicket 1.4.17 & 93.75\% & 100.00\%\\
      \bottomrule
    \end{tabular}
    \caption{Abdeckungsgrad der gefundenen GUI-Komponenten und Paradigmen}
    \label{tab:unterstuetztungDerKomponenten}
  \end{center}
\end{table}

\begin{table}[!h]
  \sffamily 
  \begin{center}
    \begin{tabular}{lrp{3cm}}
      \toprule
      \textbf{Java Web Framework} & \textbf{Gesamtabdeckung} &
      \textbf{Implementierung möglich}\\
      \midrule
      A1 - ULC, Canoo RIA Suite & Prüfung entfällt & Prüfung entfällt\\
      A2 - Struts 1.3.10 mit ZIP-Framework & 61.54\% & Nein\\
      A3 - Vaadin 6.5.7 & 94.87\% & Ja\\
      A4 - Apache Wicket 1.4.17 & 94.87\% & Ja\\
      \bottomrule
    \end{tabular}
    \caption{Gesamtabdeckung der Komponenten und ob die Implementierung möglich
    ist}
    \label{tab:gesamtabdeckungUndObImplementierungMoeglichIst}
  \end{center}
\end{table}