Die Arbeit war in meinen Auge ein voller Erfolg, da ich alle Ziele erreicht und
viel gelernt habe. Ich bin froh das Studium hinter mir zu lassen und bereit für
neue Herausforderungen.

\section{Fazit}

Rückblickend auf die Diplomarbeit gibt es einige Punkte welche ich bezüglich der
gewählten Methoden, und der daraus resultierenden Ergebnisse, ansprechen möchte.
Zudem gibt es Fragen zum Vorgehen, welche ich vorweg beantworten möchte.
\newline

\begin{quote}\begin{itshape}Wieso die gewichtete Nutzwertanalyse und der
Analytic Hierarchy Process?\end{itshape}\end{quote}

Die gewichtete Nutzwertanalyse ist eine prädestinierte Methode zur
Entscheidungsfindung bei einer Evaluation. Sie ist einfach verständlich und
umsetzbar. Natürlich gibt es weitere Methoden, wie zum Beispiel die
SWOT-Analyse\footnote{Die SWOT-Analyse wird im Bereich der Betriebswirtschaft
häufig übersetzt mit „Analyse der Stärken, Schwächen, Chancen und
Risiken“, siehe \cite{SWOT}. SWOT ist ein Werkzeug des strategischen
Managements.}, welche dafür auch verwendet werden könnte.

Der Analytic Hierarchy Process habe ich gewählt, weil er mathematisch begründet
ist, was zum Einsatzgebiet eines Ingenieurs passt. Zudem kann diese Methode für
die Entscheidungsfindung jeglicher Fragen in allen Lebenslagen dienen, auf
was ich in Zukunft sicher wieder zurückgreifen werde.
\newline

\begin{quote}\begin{itshape}Wieso nur drei Swing
Applikationen?\end{itshape}\end{quote}

Die Antwort ist relativ simpel. Da es sich bei den untersuchten Swing
Applikationen um verschiedene Business Lösungen gehandelt hat, sollten die
meisten gängigen GUI Komponenten und Konzepte enthalten sein. Mit der Zunahme
der zu untersuchenden Applikationen nimmt die Anzahl neu gefundener Komponenten
und Paradigmen wahrscheinlich rapide ab.
\newline

\begin{quote}\begin{itshape}Wieso nur vier Java Web
Frameworks?\end{itshape}\end{quote}

Um eine Evaluation stichhaltiger zu machen, wäre es sinnvoll möglichst viele
Java Web Frameworks mit einzubeziehen. Durch den zeitlichen Rahmen der
Diplomarbeit war es leider nicht möglich mehr Alternativen zu untersuchen.
\newline

\begin{quote}\begin{itshape}Wieso die 18 Anforderungen gemäss AgileLearn als
Soll-Kriterien?\end{itshape}\end{quote}

Anforderungen an Web Frameworks variieren immer aus der Sicht des Betrachters.
Aus diesem Grund habe ich ein Anforderungskatalog gesucht, welcher unabhängig
von der aktuellen Situation und vom Auftraggeber existiert. Mit AgileLearn habe
ich genau das gefunden was ich gesucht habe. Die Projektgruppe kommt aus der HTW
Berlin heraus, was den notwenigen akademischen Hintergrund bietet, und
kommuniziert im Stil von der heutigen Internet-Generation, via Blog. Somit waren
das, im Bezug auf das Thema Java Web Frameworks und der wissenschaftlichen
Arbeit als Diplomarbeit, gleich zwei Fliegen auf eine Klappe.
\newline

\begin{quote}\begin{itshape}Wieso Sicherheit, Kosten der Maintenance, Ressourcen
und Image als Konsolidierung der Soll-Kriterien?\end{itshape}\end{quote}

Die Konsolidierung und Priorisierung der gegebenen Soll-Kriterien im Sinne der
Zürcher Kantonalbank ist einer der Schwachpunkte der Diplomarbeit. Die
Begründungen, wieso diese Aspekte gewählt wurden, sind nicht unbedingt
stichfest. Ebenfalls die Vergabe der Punkte pro Soll-Kriterium und die daraus
resultierende Priorisierung.

Um eine, der Realität nähere und präzisere, Konsolidierung und Priorisierung der
Soll-Kriterien zu erhalten, hätte eine Umfrage, bei den betreffenden Stellen und
Ansprechspersonen innerhalb der Zürcher Kantonalbank, dazu durchgeführt werden
müssen. Durch die, von mir, in der Aufgabenstellung verfasste Abgrenzung von
jeglichen Umfragen, habe ich auf das verzichtet. Als Folge davon habe ich in
Treu und Glauben versucht das selber zu erarbeiten, was mit einem fragwürdigen
Resultat endet.

Dennoch darf die Evaluation an diesem Punkt nicht komplett aufgehängt werden.
Die Aspekte Sicherheit, Kosten der Maintenance, Ressourcen und das Image gehören
sicherlich auch zu den Kernkompetenzen einer Bank. Die Frage bleibt offen,
welche Kriterien es noch gibt, und wie sie von der Zürcher Kantonalbank
bevorzugt würden.
\newline

\clearpage
  
\begin{quote}\begin{itshape}Die Evaluation hätte auch anders kommen können -
Vergabe der Punkte in der Nutzwertanalyse, für die einzelnen Kriterien über
einen Kriterien-Bewertungskatalog.\end{itshape}\end{quote}

Ein weiterer Schwachpunkt in der Diplomarbeit ist die Vergabe der Punkte während
der Durchführung der gewichteten Nutzwertanalyse. Es wurden sechs Soll-Kriterien
ausgearbeitet, welche für jede Alternative bewertet werden soll. Die Punkte
wurden nach meinem Ermessen möglichst fair vergeben, jedoch ist die Vergabe für
aussenstehende nicht komplett nachvollziehbar.

Eine sinnvolle Methode für eine faire, objektive und nachvollziehbare Bewertung
der Soll-Kriterien wäre ein vordefinierter Kriterien-Bewertungskatalog mit
messbaren Bewertungskriterien. Für jedes der Soll-Kriterien gäbe es einen
eigenen Katalog. Jedes Bewertungskriterium würde eine Anzahl Punkte ergeben,
wenn es in der Alternative erfüllt wäre. Es würden in allen Alternativen die
selben Kriterien geprüft und bewertet werden, was sicherlich zu einem
präziseren und strukturierteren Resultat führen würde.

Das ausarbeiten eines solchen Kataloges hätte sicherlich viel Zeit in Anspruch
genommen. Der Faktor Zeit ist während der Diplomarbeit leider begrenzt. Trotzdem
hätte ich das auf mich genommen. Leider ist mir die Idee dazu erst nach
Abschluss der Evaluation gekommen.
\newline
  
\begin{quote}\begin{itshape}Proof of Concept mit nur einem Java Web
Framework.\end{itshape}\end{quote}

Nach meinen Erfahrungen werden in der Realität jeweils mindestens zwei Proof of
Concepts durchgeführt. Zumindest beim Einkauf von Softwareprodukten, gibt man
sich mit dem theoretisch besten nie zufrieden, sondern man holt sich noch
wenigstens eine Offerte einer zusätzlich möglichen Alternative ein. Hier ist das
leider wieder die Zeit, welche das nicht zugelassen hat. Ich bedaure aber sehr,
dass ich den Proof of Concept nicht auch mit Apache Wicket machen konnte, da das
Resultat der Evaluation relativ knapp war.
\newline

\begin{quote}\begin{itshape}Wieso sind keine Lasttests gemacht worden,
respektive die Skalierung ist nicht betrachtet worden.\end{itshape}\end{quote}

In der Realität wäre das ein wichtiger Bestandteil bei der Abnahme eines Proof
of Concepts. In der Dokumentation eines Web Frameworks wird einem immer viel
versprochen. Vertrauen ist gut, Kontrolle ist besser! 

Um sinnvolle Lasttest zu machen muss einem das Mengengerüst einer zukünftigen
Applikation bekannt sein. Da es im Rahmen der Diplomarbeit nicht um die
Evaluation für die Ablösung einer bestimmt Applikation gegangen ist, fehlten
diese Informationen. Ich hätte natürlich anstatt Lasttest einfach Performance
Messungen durchführen können, dies konnte ich leider mangels Infrastruktur nicht
bewerkstelligen.

Wenn die Infrastruktur vorhanden gewesen wäre, hätte man mithilfe von
zum Beispiel Selenium\footnote{Bei Selenium handelt es sich um ein
Testframework für Webanwendungen. Es wurde von einem Programmierteam der
Firma ThoughtWorks entwickelt und als Freie Software unter der
Apache-2.0-Lizenz veröffentlicht.} den Proof of Concept mit einer grossen Anzahl
von Browserinstanzen unter Druck setzten können. Dabei wären sicher interessante
und nützliche Resultate zum Vorschein gekommen. Richtig nützlich wären diese
Ergebnisse dann erst, wenn zwei Proof of Concepts zur Verfügung gestanden
hätten, um einen Vergleich anzustellen.

\section{Ausblick}

Im Laufe der Diplomarbeit wurde ich zum Thema Java Web Frameworks, mit
Schwerpunkt auf Präsentations-Logik, innerhalb der Zürcher Kantonalbank
kontaktiert. Dabei handelt es sich um ein Projekt, welches aktuell in den
Startlöchern steht und im Handel und Kapitalmarkt angesiedelt ist. Es ging
darum, ob es mögliche Alternativen zu den bereits eingesetzten Java Web
Frameworks gibt, welche ich empfehlen könnte. Ich habe dabei auf die vier
untersuchten Lösungen hingewiesen und im Nachhinein erfahren, dass Vaadin als
ernsthafte Alternative für das Projekt evaluiert wird. Ob die Wahl auf Vaadin
fallen wird, steht zu diesem Zeitpunkt noch in den Sternen.

Auf alle Fälle ist das letzte Wort in dieser Sache noch nicht gesprochen. Wie
lange sich Java als Plattform für Businesslösungen am Markt halten kann, weiss
niemand. In der Informatik Welt ist bekanntlich alles ein bisschen
schnelllebiger. Das selbe gilt für die zur Verfügung stehenden Web Frameworks.
Ich wette, in fünf Jahren würde ich ganz andere zu evaluierenden Alternativen
wählen, weil diese moderner sein werden und mehr dem aktuellen Trend
entsprechen würden.

Was bleibt ist das Verfahren, wie eine Evaluation durchgeführt werden kann. In
fünf Jahren würde ich es wohl wieder auf die selbe Art machen.

\section{Danksagung}

An dieser Stelle möchte ich mich bei all jenen bedanken, die mich während
meiner Studienzeit unterstützt haben. Der grösste Dank gilt meiner Freundin und
unserem gemeinsamen Sohn Linus, die mich jederzeit bedingungslos unterstützt
haben. Auch möchte ich mich bei meinen Eltern bedankten, welche mich
ebenfalls, im Bezug auf das Studium, gefördert haben.
  
Ich möchte mich in dieser Form bei Beat Seeliger bedanken, der mich als
Betreuer bei meiner Diplomarbeit unterstützte und mir mit seiner hilfsbereiten
und unkomplizierten Art und Weise zur Seite gestanden ist.
  
Ebenfalls bedanken möchte ich mich bei Berhard Mäder von der Zürcher
Kantonalbank, der mir die Arbeit ermöglicht hat und dessen Türen für mich
während der Durchführung immer offen standen.
  
Weiterhin bedanke ich mich bei meinem Arbeitgeber und Freund Silvan Spross,
der mir genügend Zeit für die Vollendung der Diplomarbeit gewährte und mit der
notwendigen Infrastruktur der allink GmbH versorgt hat, damit ich einen
ruhigen Platz für die Durchführung der Arbeit hatte.
  
Silvan Spross und Stefan Laubenberger möchte ich auch, da sie die Arbeit
gegengelesen haben.

Bei Bernhard Böhm möchte ich für den Aufwand bedanken, den er betrieben hat, um
die Arbeit auf mögliche Fehler zu überprüfen.
  
Zum Schluss bedanke ich mich auch bei Stefan Pudig und Marco Spörri die
mich mit wichtigen Informationen, zu \ac{ZKB} spezifischen Themen, unterstützt
haben.
