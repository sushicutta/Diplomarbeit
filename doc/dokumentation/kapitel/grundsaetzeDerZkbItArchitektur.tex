  Die Grundsätze sind aus dem \begin{itshape}Handbuch der
  IT-Architektur\end{itshape}, siehe \cite{ZkbHandbuchDerItArchitektur}, der
  \ac{ZKB} entnommen. Die Grundsätze sind hier aufgelistet und mit einer ID in
  der Form \{KO\}-\{Laufnummer\} versehen, da die Grundsätze als KO-Kriterien
  für die Evaluation der Java Web Frameworks dienen.

  \begin{description}

    % Kapitel 2.2.2, N-Tier Applikationen, Seite 20
    \item[KO-01\label{itm:KO-01}]
    \footnote{\cite{ZkbHandbuchDerItArchitektur} Kapitel 2.2.2 -
    \begin{itshape}N-Tier Applikationen\end{itshape}, Seite 20}
    Applikationen sollen als N-Tier Applikationen designed und
    implementiert werden.

    % Kapitel 2.2.10, Objektorientierung, Seite 22
    \item[KO-02\label{itm:KO-02}]
    \footnote{\cite{ZkbHandbuchDerItArchitektur} Kapitel 2.2.10 -
    \begin{itshape}Objektorientierung\end{itshape}, Seite 22}
    \ac{OO} soll innerhalb der Informatik für Neuentwicklungen
    durchgängig angewandt werden.

    % Kapitel 3.3, Mehrsprachigkeit, Seite 36
    \item[KO-03\label{itm:KO-03}]
    \footnote{\cite{ZkbHandbuchDerItArchitektur} Kapitel 3.3 -
    \begin{itshape}Mehrsprachigkeit\end{itshape}, Seite 36}
    Eine neue Applikation (oder eine neue Komponente einer
    bestehenden Applikation) ist mehrsprachfähig zu realisieren.

    % Kapitel 3.3, Mehrsprachigkeit, Seite 37
    \item[KO-04\label{itm:KO-04}]
    \footnote{\cite{ZkbHandbuchDerItArchitektur} Kapitel 3.3 -
    \begin{itshape}Mehrsprachigkeit\end{itshape}, Seite 37}
    Neue Applikationen sind Unicode-fähig zu realisieren.

    % Kapitel 3.9.1, Skalierbarkeit / Ausfallsicherheit / Perfomance
    % Seite 49
    \item[KO-05\label{itm:KO-05}]
    \footnote{\cite{ZkbHandbuchDerItArchitektur} Kapitel 3.9.1 -
    \begin{itshape}Skalierbarkeit / Ausfallsicherheit / Perfomance\end{itshape},
    Seite 49}
    Eine Applikation muss in mehreren Instanzen lauffähig sein.

    % Kapitel 4.4.1, User Interface, Seite 55
    \item[KO-06\label{itm:KO-06}]
    \footnote{\cite{ZkbHandbuchDerItArchitektur} Kapitel 4.4.1 -
    \begin{itshape}User Interface\end{itshape}, Seite 55}
    Der ZKB GUI Style Guide ist in allen ZKB IT-Projekten anzuwenden.
    
    % Kapitel 5.2, Verwendung der Zentralen Server Infrastruktur ZSI, Seite 57
    \item[KO-07\label{itm:KO-07}]
    \footnote{\cite{ZkbHandbuchDerItArchitektur} Kapitel 5.2 -
    \begin{itshape}Verwendung der Zentralen Server Infrastruktur
    ZSI\end{itshape}, Seite 57}
    Die \ac{ZSI} ist als Server-Plattform für Applikationen, welche Windows-,
    Unix-, Linux-basierte Server einsetzen, zu verwenden.
    
    % Kapitel 9.2, Java RMI, Seite 71
    \item[KO-08\label{itm:KO-08}]
    \footnote{\cite{ZkbHandbuchDerItArchitektur} Kapitel 9.2 -
    \begin{itshape}Java RMI\end{itshape}, Seite 71}
    RMI kann für reine Java-Anwendungen eingesetzt werden.
    
    % Kapitel 12.2.1, Einsatz von Frameworks, Seite 140
    \item[KO-09\label{itm:KO-09}]
    \footnote{\cite{ZkbHandbuchDerItArchitektur} Kapitel 12.2.1 -
    \begin{itshape}Einsatz von Frameworks\end{itshape}, Seite 140}
    Für Java-Applikationen (Internet, Extranet und Intranet) wird das
    ZIP-Framework eingesetzt.
    
    % Kapitel 12.3.5, Client-/Server-Schemata von Internet-Applikationen,
    % Seite 141
    \item[KO-10\label{itm:KO-10}]
    \footnote{\cite{ZkbHandbuchDerItArchitektur} Kapitel 12.3.5 -
    \begin{itshape}Client-/Server-Schemata von Internet-Applikationen\end{itshape}, Seite 141}
    Die Validierung und Plausibilisierung der Eingaben erfolgt immer
    abschliessend auf dem bankseitigen Applikations-Server. Es ist aber
    durchaus möglich, dass sich auf der Client-Seite eine Logik zur Überprüfung
    und Validierung der Eingaben für den Benutzerkomfort befindet.
    
    % Kapitel 12.3.5, Client-/Server-Schemata von Internet-Applikationen,
    % Seite 141
    \item[KO-11\label{itm:KO-11}]
    \footnote{\cite{ZkbHandbuchDerItArchitektur} Kapitel 12.3.5 -
    \begin{itshape}Client-/Server-Schemata von Internet-Applikationen\end{itshape}, Seite 141}
    Die Business-Logik in einer Internet-Applikation ist so auszulegen, dass
    diese von verschiedenen Präsentations-Logiken im Rahmen von Ultra-Thin- und
    Thin-Client-Applikationen genutzt werden kann.
    
    % Kapitel 12.3.6.1, Browser-Abhängigkeiten, Seite 142
    \item[KO-12\label{itm:KO-12}]
    \footnote{\cite{ZkbHandbuchDerItArchitektur} Kapitel 12.3.6.1 -
    \begin{itshape}Browser-Abhängigkeiten\end{itshape}, Seite 142}
    Die Internet-Applikationen der ZKB werden nicht mit Browser Abhängigkeiten
    versehen und orientieren sich an den neutralen Standards der W3C-Komission.
    
    % Kapitel 12.3.6.2, Session-Mechanismus, Seite 142
    \item[KO-13\label{itm:KO-13}]
    \footnote{\cite{ZkbHandbuchDerItArchitektur} Kapitel 12.3.6.2 -
    \begin{itshape}Session-Mechanismus\end{itshape}, Seite 142}
    Für Ultra-Thin-Client-Applikationen wird als Session-Mechanismus die
    Cookie oder die URL-Rewriting-Methode angewendet.
    
    % Kapitel 12.3.6.4, Einfache Internet-Applikationen, Seite 143
    \item[KO-14\label{itm:KO-14}]
    \footnote{\cite{ZkbHandbuchDerItArchitektur} Kapitel 12.3.6.4 -
    \begin{itshape}Einfache Internet-Applikationen\end{itshape}, Seite 143}
    Einfache Internet-Applikationen mit dem Schwerpunkt Information können
    ausschliesslich Java Server Pages verwenden.
    
    % Kapitel 12.3.6.5, Komplexe Internet-Applikationen, Seite 143
    \item[KO-15\label{itm:KO-15}]
    \footnote{\cite{ZkbHandbuchDerItArchitektur} Kapitel 12.3.6.5 -
    \begin{itshape}Komplexe Internet-Applikationen\end{itshape}, Seite 143}
    Komplexe Internet-Applikationen verwenden eine Kombination von Java Server
    Pages und mindestens einem Servlet als Dispatcher-Mechanismus.
    
    % Kapitel 12.3.6.6, Verwendung von JavaScript beziehungsweise ECMAScript
    % Seite 143
    \item[KO-16\label{itm:KO-16}]
    \footnote{\cite{ZkbHandbuchDerItArchitektur} Kapitel 12.3.6.6 -
    \begin{itshape}Verwendung von JavaScript beziehungsweise
    ECMAScript\end{itshape}, Seite 143}
    Die Internet-Applikationen funktionieren auch eingeschränkt, ohne dass die
    Skript-Funktion im Browser aktiviert ist.
    
    % Kapitel 12.3.6.7, Einstz von ActiveX und Cookies, Seite 143
    \item[KO-17\label{itm:KO-17}]
    \footnote{\cite{ZkbHandbuchDerItArchitektur} Kapitel 12.3.6.6 -
    \begin{itshape}Einsatz von ActiveX und Cookies\end{itshape}, Seite 143}
    ActiveX wird wegen der Möglichkeit für direkte Zugriffe auf das
    Betriebssystem nicht eingesetzt.
    
    % Kapitel 12.3.6.8, Applets, Seite 144
    \item[KO-18\label{itm:KO-18}]
    \footnote{\cite{ZkbHandbuchDerItArchitektur} Kapitel 12.3.6.8 -
    \begin{itshape}Applets\end{itshape}, Seite 144}
    Es werden keine neuen Applikationen als Java Applets entwickelt. Der
    Einsatz von Applets beschränkt sich auf einfache Funktionen wie Börsen-
    oder News-Ticker.
    
    % Kapitel 12.3.6.9, Browser-Plugins, Seite 144
    \item[KO-19\label{itm:KO-19}]
    \footnote{\cite{ZkbHandbuchDerItArchitektur} Kapitel 12.3.6.9 -
    \begin{itshape}Browser-Plugins\end{itshape}, Seite 144}
    Internet-Applikation werden ohne Plugins entwickelt.
    
    % Kapitel 12.3.7, Client-Technologien, Seite 144
    \item[KO-20\label{itm:KO-20}]
    \footnote{\cite{ZkbHandbuchDerItArchitektur} Kapitel 12.3.7 -
    \begin{itshape}Client-Technologien\end{itshape}, Seite 144}
    Für Ultra Thin Clients bzw. Browser-basierende Applikationen muss das
    aktuelle, Struts-basierende HTML-Client-Framework der ZKB Internet
    Plattform verwendet werden.
    
    % Kapitel 12.3.8, Layering von Internet-Applikationen, Seite 145
    \item[KO-21\label{itm:KO-21}]
    \footnote{\cite{ZkbHandbuchDerItArchitektur} Kapitel 12.3.8 -
    \begin{itshape}Layering von Internet-Applikationen\end{itshape}, Seite 145}
    Neue Internet- und Extranet-Applikationen müssen sich an das Layering
    gemäss nachfolgender Grafik halten. Für Intranet-Applikationen ist die
    Validator-Komponente fakultativ.
    
    % Kapitel 12.3.8, Layering von Internet-Applikationen, Seite 146
    \item[KO-22\label{itm:KO-22}]
    \footnote{\cite{ZkbHandbuchDerItArchitektur} Kapitel 12.3.8 -
    \begin{itshape}Layering von Internet-Applikationen\end{itshape}, Seite 146}
    Eine Applikation muss ohne Änderung von der Intranet-Anwendung zur Extra-
    oder Internet-Applikation gemacht werden können. Es geschieht dies
    lediglich durch das Vorschalten der Validator-Komponente.
    
    % Kapitel 12.3.10.2, Web Server, Seite 146
    \item[KO-23\label{itm:KO-23}]
    \footnote{\cite{ZkbHandbuchDerItArchitektur} Kapitel 12.3.10.2 -
    \begin{itshape}Web Server\end{itshape}, Seite 146}
    Der ZKB Standard-Web-Server ist der Apache HTTP-Server.
    
    % Kapitel 12.4.2, Architektur beim Einsatz von Application-Servern,
    % Seite 147
    \item[KO-24\label{itm:KO-24}]
    \footnote{\cite{ZkbHandbuchDerItArchitektur} Kapitel 12.4.2 -
    \begin{itshape}Architektur beim Einsatz von
    Application-Servern\end{itshape}, Seite 147}
    Der Application-Server wird als die integrierte technische Middleware für
    die Unterstützung von Java Server Pages (JSP), Servlets, Enterprise Java
    Beans (EJB) und der sicheren Kommunikation zwischen Client und Server
    eingesetzt.
    
    % Kapitel 12.4.2, Architektur beim Einsatz von Application-Servern,
    % Seite 148
    \item[KO-25\label{itm:KO-25}]
    \footnote{\cite{ZkbHandbuchDerItArchitektur} Kapitel 12.4.2 -
    \begin{itshape}Architektur beim Einsatz von
    Application-Servern\end{itshape}, Seite 148}
    Der ZKB Standard-J2EE-Application-Server ist der JBoss Application Server.
    
    % Kapitel 12.4.2, Architektur beim Einsatz von Application-Servern,
    % Seite 148
    \item[KO-26\label{itm:KO-26}]
    \footnote{\cite{ZkbHandbuchDerItArchitektur} Kapitel 12.4.2 -
    \begin{itshape}Architektur beim Einsatz von
    Application-Servern\end{itshape}, Seite 148}
    Der J2EE-Application-Server wird in der \ac{ZSI} eingesetzt.
    
    % Kapitel 12.4.2, Architektur beim Einsatz von Application-Servern,
    % Seite 148
    \item[KO-27\label{itm:KO-27}]
    \footnote{\cite{ZkbHandbuchDerItArchitektur} Kapitel 12.4.2 -
    \begin{itshape}Architektur beim Einsatz von
    Application-Servern\end{itshape}, Seite 148}
    Die Serverplattform für den Einsatz von Web-Application-Servern für
    Internet, Intranet udn Extranet-Applikationen ist Linux.
    
    % Kapitel 12.4.2, Architektur beim Einsatz von Application-Servern,
    % Seite 150
    \item[KO-28\label{itm:KO-28}]
    \footnote{\cite{ZkbHandbuchDerItArchitektur} Kapitel 12.4.2 -
    \begin{itshape}Architektur beim Einsatz von
    Application-Servern\end{itshape}, Seite 150}
    Die technischen Services wie Session Management, Load Balancing,
    Transaction Management und Instance Pooling werden vom J2EE Application
    Server zur Verfügung gestellt.
    
    % Kapitel 12.4.3, Einsatz von Enterprise Java Beans, Seite 150
    \item[KO-29\label{itm:KO-29}]
    \footnote{\cite{ZkbHandbuchDerItArchitektur} Kapitel 12.4.3 -
    \begin{itshape}Einsatz von Enterprise Java Beans\end{itshape}, Seite 150}
    Das Muster JSP/Servlets mit EJBs ist anzuwenden, wenn die Applikation eine
    umfangreiche, komplexe Business-Logik aufweist, die Business-Logik
    wiederverwendbar sein soll, mehrere unterschiedliche Clients (Browser
    (Ultra-Thin-)(HTML), Thin-(Java), Mobile, …) mit einer Business Logik
    bedient werden müssen, hohe Anforderungen an die Skalierbarkeit gestellt
    werden und ein lange Lebenszyklus der Applikation erwartet wird.
    
    % Kapitel 12.4.3, Einsatz von Enterprise Java Beans, Seite 150
    \item[KO-30\label{itm:KO-30}]
    \footnote{\cite{ZkbHandbuchDerItArchitektur} Kapitel 12.4.3 -
    \begin{itshape}Einsatz von Enterprise Java Beans\end{itshape}, Seite 150}
    Das Muster JSP/Servlets ohne EJBs ist anzuwenden, wenn die Applikation eine
    einfache Business-Logik aufweist, nur einen Client, zum Beispiel ein
    Browser (Ultra-Thin-)-Interface unterstützt, niedrige Anforderungen an die
    Skalierbarkeit stellt und nur eine vergleichsweise kurzer Lebenszyklus der
    Applikation erwartet wird.
    
    % Kapitel 12.9.2, Standards, Seite 171
    \item[KO-31\label{itm:KO-31}]
    \footnote{\cite{ZkbHandbuchDerItArchitektur} Kapitel 12.9.2 -
    \begin{itshape}Standards\end{itshape}, Seite 171}
    In der ZKB werden folgende Standards für Web Services eingesetzt: SOAP,
    WSDL, W3C X Schema (XSD, WXS).
    
    % Kapitel 13.1, Client/Server-Konzept, Seite 175
    \item[KO-32\label{itm:KO-32}]
    \footnote{\cite{ZkbHandbuchDerItArchitektur} Kapitel 13.1 -
    \begin{itshape}Client/Server-Konzept\end{itshape}, Seite 175}
    Alle Neuentwicklungen sind konsequent in Client-/Server-Komponenten
    aufzuteilen.
    
    % Kapitel 13.2, Anwendungsstruktur (MVC/Model, View, Controller), Seite 175
    \item[KO-33\label{itm:KO-33}]
    \footnote{\cite{ZkbHandbuchDerItArchitektur} Kapitel 13.2 -
    \begin{itshape}Anwendungsstruktur (MVC/Model, View,
    Controller)\end{itshape}, Seite 175}
    Die gewünschte Isolation der Systemteile wird durch eine Anwendungsstruktur
    mit Trennung in Model (Verarbeitung, Datenhaltung), View
    (Benutzeroberfläche) und Controller (Organisation, Ereignisvermittlung)
    erreicht.
    
    % Kapitel 13.9.7.2, Allgemeine Grundsätze, Seite 189
    \item[KO-34\label{itm:KO-34}]
    \footnote{\cite{ZkbHandbuchDerItArchitektur} Kapitel 13.9.7.2 -
    \begin{itshape}Allgemeine Grundsätze\end{itshape}, Seite 189}
    Jede Applikation ist dafür verantwortlich, dass diejenigen Daten, für die
    sie den Lead hat, validiert sind.
    
    % Kapitel 13.9.7.2, Allgemeine Grundsätze, Seite 189
    \item[KO-35\label{itm:KO-35}]
    \footnote{\cite{ZkbHandbuchDerItArchitektur} Kapitel 13.9.7.2 -
    \begin{itshape}Allgemeine Grundsätze\end{itshape}, Seite 189}
    Jede Applikation benutzt Datenvalidierung um sicherzustellen, dass sie die
    erhaltenen Daten verarbeiten kann und dass ihr Betrieb nicht gefährdet ist
    (Stabilität / Verfügbarkeit).
    
    % Kapitel 13.9.7.3.1, Datenvalidierung an der Benutzersczhnittstelle
    % Seite 191
    \item[KO-36\label{itm:KO-36}]
    \footnote{\cite{ZkbHandbuchDerItArchitektur} Kapitel 13.9.7.3.1 -
    \begin{itshape}Datenvalidierung an der Benutzersczhnittstelle\end{itshape},
    Seite 191}
    Benutzereingaben werden so früh wie möglich validiert.
    
    % Kapitel 13.10, Programmiersprachen und Entwicklungsumgebungen
    % Seite 192
    \item[KO-37\label{itm:KO-37}]
    \footnote{\cite{ZkbHandbuchDerItArchitektur} Kapitel 13.10 -
    \begin{itshape}Programmiersprachen und Entwicklungsumgebungen\end{itshape},
    Seite 192}
    Für die Entwicklung neuer Applikationen und beim neuen Design bestehender
    Applikationen wird Java eingesetzt.
    
    % Kapitel 13.10.4.8, Zusätzliche Java Klassenbibliotheken, Seite 195
    \item[KO-38\label{itm:KO-38}]
    \footnote{\cite{ZkbHandbuchDerItArchitektur} Kapitel 13.10.4.8 -
    \begin{itshape}Zusätzliche Java Klassenbibliotheken\end{itshape}, Seite 195}
    Zusätzliche Java-Klassenbibliotheken, also solche, die nicht im JDK
    enthalten sind, werden nur in begründeten Ausnahmen eingesetzt.
    Normalerweise müssen solche Bibliotheken dem 100\%-Pure-Java-Grundsatz
    entsprechen. Ausnahmen können systemnahe Funktionen für Security und
    dergleichen sein.
    
    % Kapitel 13.11, Einsatz von .NET-basierten Applikationen, Seite 196
    \item[KO-39\label{itm:KO-39}]
    \footnote{\cite{ZkbHandbuchDerItArchitektur} Kapitel 13.11 -
    \begin{itshape}Einsatz von .NET-basierten Applikationen\end{itshape}, Seite
    196}
    .NET-basierte Applikationen werden in der ZKB Informatik nicht entwickelt
    oder zur Entwicklung in Auftrag gegeben ausser Kleinapplikationen der
    Kategorie „Office“.
    
    % Kapitel 13.13, Einsatz von Open Source Software, Seite 206
    \item[KO-40\label{itm:KO-40}]
    \footnote{\cite{ZkbHandbuchDerItArchitektur} Kapitel 13.13 -
    \begin{itshape}Einsatz von Open Source Software\end{itshape}, Seite 206}
    Die Nutzung von Open Source Software ist erlaubt.
    
    % Kapitel 13.13, Einsatz von Open Source Software, Seite 206
    \item[KO-41\label{itm:KO-41}]
    \footnote{\cite{ZkbHandbuchDerItArchitektur} Kapitel 13.13 -
    \begin{itshape}Einsatz von Open Source Software\end{itshape}, Seite 206}
    Open Source Software unterliegt denselben Kriterien wie kommerzielle
    Software. Sie muss evaluiert, registriert, homologiert, intern supported
    und gepflegt werden.    
    
    % Kapitel 13.13.1, Kriterien für die Evaluation von Open Source
    % Software, Seite 207
    \item[KO-42\label{itm:KO-42}]
    \footnote{\cite{ZkbHandbuchDerItArchitektur} Kapitel 13.13.1 -
    \begin{itshape}Kriterien für die Evaluation von Open Source
    Software\end{itshape}, Seite 207}
    Produktiv eingesetzte Open Source Software muss durch angemessene
    Informationsquellen unterstützt sein (Newsgroups, Mailinglists,
    FAQ-Listen, WebSites, User Groups).    
    
    % Kapitel 13.13.1, Kriterien für die Evaluation von Open Source
    % Software, Seite 207
    \item[KO-43\label{itm:KO-43}]
    \footnote{\cite{ZkbHandbuchDerItArchitektur} Kapitel 14.14.1 -
    \begin{itshape}Kriterien für die Evaluation von Open Source
    Software\end{itshape}, Seite 207}
    Die Weiterentwicklung von produktiv eingesetzter Open Source Software
    ausserhalb der ZKB muss öffentlich einsehbar sein und aktiv verfolgt werden
    können.
    
    % Kapitel 13.13.1, Kriterien für die Evaluation von Open Source
    % Software, Seite 207
    \item[KO-44\label{itm:KO-44}]
    \footnote{\cite{ZkbHandbuchDerItArchitektur} Kapitel 13.13.1 -
    \begin{itshape}Kriterien für die Evaluation von Open Source
    Software\end{itshape}, Seite 207}
    Die Lizenzbedingungen einer Open Source Software müssen vor dem Einsatz
    geprüft werden und von der Zürcher Kantonalbank akzeptiert werden können.
  \end{description}