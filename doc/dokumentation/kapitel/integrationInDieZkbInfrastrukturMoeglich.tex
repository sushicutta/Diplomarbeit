In diesem Kapitel soll behandelt werden, ob eine Integration der jeweiligen Java
Web Frameworks in die Infrastruktur der \ac{ZKB} möglich ist. Dabei soll geprüft
werden, ob die notwenigen \acp{API}, auf welchen die Frameworks aufbauen, durch
die zur Verfügung gestellten Applikationsserver, abgedeckt sind.

Es wird auf die Prüfung vom Framework ``A1 - ULC, Canoo RIA Suite'' verzichtet,
da es aufgrund der Evaluation als mögliche Alternative entfällt.

\section{A2 - Struts 1.3.10 mit ZIP-Framework}

Das Framework ``A2 - Struts 1.3.10 mit ZIP-Framework'' kann in die bestehende
Infrastruktur der \ac{ZKB} integriert werden. Für den Betrieb einer Struts
Applikation ist ein Servlet Container notwendig, dieser wird in der
Infrastruktur mit dem JBoss Web, welcher auf Apache Tomcat 6.0 basiert, zur
Verfügung gestellt. Die genauen Anforderungen an die Konfiguration können aus
der Dokumentation entnommen werden, siehe \cite{StrutsDokumentation}.

Das Framework ist zudem in der \ac{ZKB} schon im Einsatz. Eine Applikation,
welche damit entwickelt wurde, ist das Online-Banking, auch genannt ``OnBa''.

\section{A3 - Vaadin 6.5.7}

\section{A4 - Apache Wicket 1.4.17}

Das Framework ``A4 - Apache Wicket 1.4.17'' kann in die bestehende IT
Infrastruktur der \ac{ZKB} integriert werden. Für den Betrieb einer Struts
Applikation ist auch nur ein Servlet Container notwendig, siehe
\cite{WicketDokumentation}.

Das Framework ist ebenfalls bereits in der \ac{ZKB} im Einsatz. Der Einsatz
wurde durch eine Ausnahmegenehmigung geregelt. Eine Applikation, welche damit
entwickelt wurde, ist das Bond-Rating-Tool, auch genannt ``BoReTo''.

\section{Resultat}