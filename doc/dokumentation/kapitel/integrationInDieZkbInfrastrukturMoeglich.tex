In diesem Kapitel soll behandelt werden, ob eine Integration der jeweiligen Java
Web Frameworks in die Infrastruktur der \ac{ZKB} möglich ist. Dabei soll geprüft
werden, ob die notwenigen \acp{API}, auf welchen die Frameworks aufbauen, durch
die zur Verfügung gestellten Applikationsserver, abgedeckt sind. Als Grundlage
werden die Informationen aus dem Kapitel
\ref{chapter:InfrastrukturDerZuercherKantonalbank}
(\nameref{chapter:InfrastrukturDerZuercherKantonalbank}, S.
\pageref{chapter:InfrastrukturDerZuercherKantonalbank}ff) verwendet.

Es wird auf die Prüfung der Integration vom Framework ``A1 - ULC, Canoo RIA
Suite'' verzichtet, da es aufgrund der Evaluation als mögliche Alternative
entfällt.

\section{A2 - Struts 1.3.10 mit ZIP-Framework}

Das Framework ``A2 - Struts 1.3.10 mit ZIP-Framework'' kann in die bestehende
Infrastruktur der \ac{ZKB} integriert werden. Für den Betrieb einer Struts
Applikation ist ein Servlet Container notwendig, dieser wird in der
Infrastruktur mit dem JBoss Web, welcher auf Apache Tomcat 6.0 basiert, zur
Verfügung gestellt. Die genauen Anforderungen an die Konfiguration können aus
der Dokumentation entnommen werden, siehe \cite{StrutsDokumentation}.

Das Framework ist zudem in der \ac{ZKB} schon im Einsatz. Eine Applikation,
welche damit entwickelt wurde, ist das Online-Banking, auch genannt ``OnBa''.

\section{A3 - Vaadin 6.5.7}

Das Framework ``A3 - Vaadin 6.5.7'' kann in die bestehende
Infrastruktur der \ac{ZKB} integriert werden. Der Betrieb setzt ebenfalls ein
Servlet Container voraus. In der Dokumentation ist zu entnehmen, dass es sich
dabei mindestens um einen Apache Tomcat 6.0 handelt, siehe \cite{BookOfVaadin}
S. 11. Als Laufzeitumgebung wird mindestens Java 1.5 vorausgesetzt, was
ebenfalls verfügbar ist.

Das Framework wurde in der \ac{ZKB} noch nie eingesetzt. Der Einsatz müsste mit
den aktuellen Bestimmungen aus der IT Architektur über einen Ausnahmegenehmigung
geregelt werden.

\section{A4 - Apache Wicket 1.4.17}

Das Framework ``A4 - Apache Wicket 1.4.17'' kann in die bestehende IT
Infrastruktur der \ac{ZKB} integriert werden. Für den Betrieb einer Struts
Applikation ist auch nur ein Servlet Container notwendig, siehe
\cite{WicketDokumentation}.

Das Framework ist ebenfalls bereits in der \ac{ZKB} im Einsatz. Der Einsatz
wurde durch eine Ausnahmegenehmigung geregelt. Eine Applikation, welche damit
entwickelt wurde, ist das Bond-Rating-Tool, auch genannt ``BoReTo''.

\section{Resultat}

Alle drei evaluierten Java Web Frameworks entsprechen den Begebenheiten aus der
\ac{ZKB} IT Infrastruktur und sind somit für den Einsatz geeigent. Das Resultat
wird in der Tabelle \ref{tab:integrationMoeglich} dargestellt.
\newline

\begin{table}[!h]
  \sffamily 
  \begin{center}
    \begin{tabular}{lc}
      \toprule
      \textbf{Java Web Framework} & \textbf{Integration möglich}\\
      \midrule
      A1 - ULC, Canoo RIA Suite & Prüfung entfällt\\
      A2 - Struts 1.3.10 mit ZIP-Framework & Ja\\
      A3 - Vaadin 6.5.7 & Ja\\
      A4 - Apache Wicket 1.4.17 & Ja\\
      \bottomrule
    \end{tabular}
    \caption{Integration in die IT Infrastruktur der \ac{ZKB} möglich?}
    \label{tab:integrationMoeglich}
  \end{center}
\end{table}