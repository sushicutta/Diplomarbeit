%
%   Technische Dokumentation für die Semesterarbeit
%
%   Created by Roman Wuersch on 2010-12-23.	
%
% =============================================================================
% Documentdefinition  beginns here
% =============================================================================

\documentclass[abstracton, listof=totocnumbered,
bibliography=totocnumbered]{scrreprt}
% \documentclass[listof=totoc,bibliography=totoc]{scrreprt}
\usepackage[ngerman]{babel}

\usepackage{tocbasic}

% Use utf-8 encoding for foreign characters
\usepackage[utf8]{inputenc}
% \usepackage[applemac]{inputenc}

% Setup for fullpage use
\usepackage{fullpage}

% Silbentrennung kann unterdrückt werden
\usepackage{hyphenat}

% Tradmark
\def\TTra{\textsuperscript{\texttrademark}}

% Running Headers and footers
%\usepackage{fancyhdr}

% Multipart figures
%\usepackage{subfigure}

% More symbols
%\usepackage{amsmath}
%\usepackage{amssymb}
%\usepackage{latexsym}

% Surround parts of graphics with box
\usepackage{boxedminipage}

% Package for including code in the document
\usepackage{listings}

% If you want to generate a toc for each chapter (use with book)
\usepackage{minitoc}

% Abkürzungsverzeichnis erstellen.
\usepackage[printonlyused]{acronym}

% schöne Tabelle zeichnen
\usepackage{booktabs}
\renewcommand{\arraystretch}{1.2} %Die Zeilenabstände in Tabllen angepasst.

% für variable Breiten
\usepackage{tabularx}

% Durchgestrichener Text
\usepackage[normalem]{ulem} %emphasize weiterhin kursiv

% This is now the recommended way for checking for PDFLaTeX:
\usepackage{ifpdf}

\usepackage[hyperfootnotes=false]{hyperref}
\hypersetup{
  bookmarks=true,         % show bookmarks bar?
  unicode=true,           % non-Latin characters in Acrobat’s bookmarks
  pdftoolbar=true,        % show Acrobat’s toolbar?
  pdfmenubar=true,        % show Acrobat’s menu?
  pdffitwindow=true,      % window fit to page when opened
  pdfstartview={FitH},    % fits the width of the page to the window
  pdftitle={Semesterarbeit},   
  pdfauthor={Roman Würsch},
  pdfsubject={Client-Server-Kommunikation mit Android},
  pdfcreator={TeXnicCenter 1.0 RC1},
  pdfproducer={MiKTeX 2.9},
  pdfnewwindow=true,      % links in new window
  colorlinks=false,       % false: boxed links; true: colored links
  linkcolor=red,          % color of internal links
  citecolor=green,        % color of links to bibliography
  filecolor=magenta,      % color of file links
  urlcolor=cyan          % color of external links
}

\ifpdf
    \usepackage[pdftex]{graphicx}
\else
    \usepackage{graphicx}
\fi

\title{Evaluation eines Java Web Frameworks zur Ablösung bestehender Java Swing
Applikationen}

\author{Diplomarbeit in Informatik\\
    \\
    Studierender - Roman Würsch\\
	Auftraggeber - Bernhard Mäder, ZKB\\
    Projektbetreuer - Beat Seeliger\\
    Experte - tbd\\
	\\
	HSZ-T - Technische Hochschule Zürich}

\date{März 2011 bis Juni 2011}

% =============================================================================
% Documenttext beginns here
% =============================================================================

\begin{document}

  \ifpdf
    \DeclareGraphicsExtensions{.pdf, .jpg, .tif}
  \else
    \DeclareGraphicsExtensions{.eps, .jpg}
  \fi
  
  % ===========================================================================
  % Titelblatt beginns here
  % ===========================================================================
  
  \maketitle
  
  % ===========================================================================
  % Abstract beginns here
  % ===========================================================================
  
  \pagenumbering{Alph}
  
  \begin{abstract}

  Es werden Methoden zur Analyse von Java Swing Applikationen und zur
  Evaluation von Java Web Frameworks ausgearbeitet werden.

  Mit der Methode zur Analyse werden bestehende Java Swing Applikationen der
  Zürcher Kantonalbank analysiert. Der Fokus liegt darauf in den Applikationen
  die gemeinsamen Muster, genutzter Swingkomponenten, zu erkennen und zu
  kategorisieren. Java Web Frameworks, welche sich am Markt etabliert haben,
  werden durch einen Evaluationsprozess auf deren Einsatz in der Zürcher
  Kantonalbank geprüft. Zusätzlich wird geprüft, ob mit den jeweiligen
  Frameworks die genutzten Swingkomponenten äquivalent umgesetzt werden können.
  Mit einer Analyse der IT Infrastruktur der Zürcher Kantonalbank, wird
  geprüft, ob eine Integration in die bestehende IT Infrastruktur möglich ist.

  Durch die Umsetzung eines Prototypen soll gezeigt werden, dass das evaluierte
  Java Web Framework den geforderten Funktionalitäten entspricht.

  Mit den gewonnenen Erkenntnissen, wird eine Empfehlung, für den Einsatz eines
  Java Web Frameworks zur Ablösung bestehender Java Swing Applikationen
  ausgesprochen.
  
\end{abstract}
  
  % ===========================================================================
  % Inahltsverzeichnis beginns here
  % ===========================================================================

  \pagenumbering{Roman}
  
  \tableofcontents
  
  \clearpage
  
  % ===========================================================================
  % Kapitel Administratives beginns here
  % ===========================================================================
  
  \pagenumbering{arabic}
  
  \chapter{Personalienblatt}

  \begin{tabbing}
  \hspace*{6cm}\= \kill
  Name, Vorname: \> {\bf Roman Würsch} \\
  Adresse: \> {\bf Murhaldenweg 16} \\
  PLZ, Wohnort: \> {\bf 8057 Zürich} \\
  \\
  Geburtsdatum: \> {\bf 10.11.1980} \\
  Heimatort: \> {\bf Emmetten NW} \\
\end{tabbing}

\noindent
Ich bestätige, dass die vorliegende Diplomarbeit, ``Evaluation eines Java Web
Frameworks zur Ablösung bestehender Java Swing Applikationen'', in allen Teilen
selbständig erarbeitet und durchgeführt wurde.

\vspace*{3cm}

\begin{tabbing}
  \hspace*{8cm}\= \kill
  Ort und Datum \> {Unterschrift} \\
\end{tabbing}
  
  % ===========================================================================
  % Kapitel Aufgabenstellung beginns here
  % ===========================================================================
  
  %\newpage
  
  \chapter{Aufgabenstellung}
  
  \section{Ausgangslage}
  
  Die Zürcher Kantonalbank hat viele Applikationen welche als Client Applikationen
in Swing implementiert sind. Dabei ist das Deployment der Clients und die
Ausbreitung entsprechender Patches, innerhalb der Zürcher Kantonalbank, mit
einem Mehraufwand verbunden. Client Applikationen, die einem Kunden zur
Verfügung gestellt werden, sind an eine spezifische Java Version gebunden. Bei
der Integration in die IT-Landschaft beim Kunden, kann das zur Verzögerung
durch einen erhöhten Testaufwand führen. In der Annahme, dass ein Webbrowser in
der Zürcher Kantonalbank und bei deren Kunden eingesetzt wird, macht der
Einsatz einer Weblösung Sinn.

  
  \section{Ziel der Arbeit}

  Es sollen bestehende Java Swing Applikationen der Zürcher Kantonalbank
analysiert werden. In diesen Applikationen sollen die gemeinsamen Muster,
genutzter Swingkomponenten, erkannt und kategorisiert werden. Java Web
Frameworks, welche sich am Markt etabliert haben, sollen auf einen möglichen
Einsatz geprüft werden. Es soll geprüft werden, ob mit den jeweiligen
Frameworks die genutzten Swingkomponenten äquivalent umgesetzt werden können.
Zudem soll geprüft werden, ob eine Integration in die bestehende IT
Infrastruktur der Zürcher Kantonalbank möglich ist.

  
  %\newpage
  
  \section{Aufgabenstellung}
  
    Die Aufgabenstellung besteht aus den vier Teilen: Ausgangslage, Ziel der
  Arbeit, Aufgabenstellung und Erwartete Resultate. Zusätzlich wurde eine
  Abgrenzung hinzugefügt, damit der Rahmen der Diplomarbeit gesetzt ist.
  
  \section{Ausgangslage}
  
  Die Zürcher Kantonalbank hat viele Applikationen welche als Client Applikationen
in Swing implementiert sind. Dabei ist das Deployment der Clients und die
Ausbreitung entsprechender Patches, innerhalb der Zürcher Kantonalbank, mit
einem Mehraufwand verbunden. Client Applikationen, die einem Kunden zur
Verfügung gestellt werden, sind an eine spezifische Java Version gebunden. Bei
der Integration in die IT-Landschaft beim Kunden, kann das zur Verzögerung
durch einen erhöhten Testaufwand führen. In der Annahme, dass ein Webbrowser in
der Zürcher Kantonalbank und bei deren Kunden eingesetzt wird, macht der
Einsatz einer Weblösung Sinn.

  
  \section{Ziel der Arbeit}

  Es sollen bestehende Java Swing Applikationen der Zürcher Kantonalbank
analysiert werden. In diesen Applikationen sollen die gemeinsamen Muster,
genutzter Swingkomponenten, erkannt und kategorisiert werden. Java Web
Frameworks, welche sich am Markt etabliert haben, sollen auf einen möglichen
Einsatz geprüft werden. Es soll geprüft werden, ob mit den jeweiligen
Frameworks die genutzten Swingkomponenten äquivalent umgesetzt werden können.
Zudem soll geprüft werden, ob eine Integration in die bestehende IT
Infrastruktur der Zürcher Kantonalbank möglich ist.

  
  \section{Aufgabenstellung}
  
    Die Aufgabenstellung besteht aus den vier Teilen: Ausgangslage, Ziel der
  Arbeit, Aufgabenstellung und Erwartete Resultate. Zusätzlich wurde eine
  Abgrenzung hinzugefügt, damit der Rahmen der Diplomarbeit gesetzt ist.
  
  \section{Ausgangslage}
  
  Die Zürcher Kantonalbank hat viele Applikationen welche als Client Applikationen
in Swing implementiert sind. Dabei ist das Deployment der Clients und die
Ausbreitung entsprechender Patches, innerhalb der Zürcher Kantonalbank, mit
einem Mehraufwand verbunden. Client Applikationen, die einem Kunden zur
Verfügung gestellt werden, sind an eine spezifische Java Version gebunden. Bei
der Integration in die IT-Landschaft beim Kunden, kann das zur Verzögerung
durch einen erhöhten Testaufwand führen. In der Annahme, dass ein Webbrowser in
der Zürcher Kantonalbank und bei deren Kunden eingesetzt wird, macht der
Einsatz einer Weblösung Sinn.

  
  \section{Ziel der Arbeit}

  Es sollen bestehende Java Swing Applikationen der Zürcher Kantonalbank
analysiert werden. In diesen Applikationen sollen die gemeinsamen Muster,
genutzter Swingkomponenten, erkannt und kategorisiert werden. Java Web
Frameworks, welche sich am Markt etabliert haben, sollen auf einen möglichen
Einsatz geprüft werden. Es soll geprüft werden, ob mit den jeweiligen
Frameworks die genutzten Swingkomponenten äquivalent umgesetzt werden können.
Zudem soll geprüft werden, ob eine Integration in die bestehende IT
Infrastruktur der Zürcher Kantonalbank möglich ist.

  
  \section{Aufgabenstellung}
  
    Die Aufgabenstellung besteht aus den vier Teilen: Ausgangslage, Ziel der
  Arbeit, Aufgabenstellung und Erwartete Resultate. Zusätzlich wurde eine
  Abgrenzung hinzugefügt, damit der Rahmen der Diplomarbeit gesetzt ist.
  
  \section{Ausgangslage}
  
  \input{../aufgabenstellung/kapitel/ausgangslage.tex}
  
  \section{Ziel der Arbeit}

  \input{../aufgabenstellung/kapitel/zielDerArbeit.tex}
  
  \section{Aufgabenstellung}
  
  \input{../aufgabenstellung/kapitel/aufgabenstellung.tex}
  
  \section{Erwartete Resultate}
  
  \input{../aufgabenstellung/kapitel/erwarteteResultate.tex}
  
  \section{Abgrenzung}
  
  \input{../aufgabenstellung/kapitel/abgrenzung.tex}
  
  \section{Erwartete Resultate}
  
  Der Studierende soll dem Auftraggeber ein Dokument erstellen, das folgendes
beinhaltet:

\begin{itemize}    
  \item Ergebnis der Analyse von bestehenden Java Swing Applikationen der
  Zürcher Kantonalbank.
  \item Kategorisierung von verwendeten Swingkomponenten.
  \item Eine Auflistung von etablierten Java Web Frameworks.
  \item Ergebnis der Analyse, ob eine Integration der Java Web Frameworks, in
  der bestehenden IT Infrastruktur der Zürcher Kantonalbank, möglich ist.
  \item Ergebnis der Analyse von Java Web Frameworks, ob eine Implementierung,
  der erkannten Swingkomponenten, möglich ist.
  \item Proof of concept. Es soll anhand eines Prototypen gezeigt werden, dass
  die Implementierung möglich ist.
  \item Eine Empfehlung für ein Java Web Framework.
\end{itemize}
  
  \section{Abgrenzung}
  
  Folgende Punkte werden formell abgegrenzt:

\begin{itemize}
  \item Die Analysen beschränken sich auf Recherchen im Internet, Büchern und
  interne Vorgaben der Zürcher Kantonalbank.
  \item Umfragen, Erhebungen sowie Feldstudien werden nicht durchgeführt.
\end{itemize}

\noindent  
Folgende Punkte werden inhaltlich abgegrenzt:
	
\begin{itemize}
  \item Die Auswahl, welche Java Swing Applikationen analysiert werden, soll
  wärend der Arbeit durchgeführt werden.
  \item Die Auswahl, welche Java Web Frameworks geprüft werden, soll wärend  der
  Arbeit durchgeführt werden.
\end{itemize}
  
  \section{Erwartete Resultate}
  
  Der Studierende soll dem Auftraggeber ein Dokument erstellen, das folgendes
beinhaltet:

\begin{itemize}    
  \item Ergebnis der Analyse von bestehenden Java Swing Applikationen der
  Zürcher Kantonalbank.
  \item Kategorisierung von verwendeten Swingkomponenten.
  \item Eine Auflistung von etablierten Java Web Frameworks.
  \item Ergebnis der Analyse, ob eine Integration der Java Web Frameworks, in
  der bestehenden IT Infrastruktur der Zürcher Kantonalbank, möglich ist.
  \item Ergebnis der Analyse von Java Web Frameworks, ob eine Implementierung,
  der erkannten Swingkomponenten, möglich ist.
  \item Proof of concept. Es soll anhand eines Prototypen gezeigt werden, dass
  die Implementierung möglich ist.
  \item Eine Empfehlung für ein Java Web Framework.
\end{itemize}
  
  \section{Abgrenzung}
  
  Folgende Punkte werden formell abgegrenzt:

\begin{itemize}
  \item Die Analysen beschränken sich auf Recherchen im Internet, Büchern und
  interne Vorgaben der Zürcher Kantonalbank.
  \item Umfragen, Erhebungen sowie Feldstudien werden nicht durchgeführt.
\end{itemize}

\noindent  
Folgende Punkte werden inhaltlich abgegrenzt:
	
\begin{itemize}
  \item Die Auswahl, welche Java Swing Applikationen analysiert werden, soll
  wärend der Arbeit durchgeführt werden.
  \item Die Auswahl, welche Java Web Frameworks geprüft werden, soll wärend  der
  Arbeit durchgeführt werden.
\end{itemize}
  
  \section{Erwartete Resultate}
  
  Der Studierende soll dem Auftraggeber ein Dokument erstellen, das folgendes
beinhaltet:

\begin{itemize}    
  \item Ergebnis der Analyse von bestehenden Java Swing Applikationen der
  Zürcher Kantonalbank.
  \item Kategorisierung von verwendeten Swingkomponenten.
  \item Eine Auflistung von etablierten Java Web Frameworks.
  \item Ergebnis der Analyse, ob eine Integration der Java Web Frameworks, in
  der bestehenden IT Infrastruktur der Zürcher Kantonalbank, möglich ist.
  \item Ergebnis der Analyse von Java Web Frameworks, ob eine Implementierung,
  der erkannten Swingkomponenten, möglich ist.
  \item Proof of concept. Es soll anhand eines Prototypen gezeigt werden, dass
  die Implementierung möglich ist.
  \item Eine Empfehlung für ein Java Web Framework.
\end{itemize}
  
  \section{Abgrenzung}
  
  Folgende Punkte werden formell abgegrenzt:

\begin{itemize}
  \item Die Analysen beschränken sich auf Recherchen im Internet, Büchern und
  interne Vorgaben der Zürcher Kantonalbank.
  \item Umfragen, Erhebungen sowie Feldstudien werden nicht durchgeführt.
\end{itemize}

\noindent  
Folgende Punkte werden inhaltlich abgegrenzt:
	
\begin{itemize}
  \item Die Auswahl, welche Java Swing Applikationen analysiert werden, soll
  wärend der Arbeit durchgeführt werden.
  \item Die Auswahl, welche Java Web Frameworks geprüft werden, soll wärend  der
  Arbeit durchgeführt werden.
\end{itemize}
  
  \newpage
  
  % ===========================================================================
  % Kapitel Erreichte Ziele beginns here
  % ===========================================================================  
  
  \chapter{Erreichte Ziele}
  
  Es wurden alle Ziele gemäss den erwarteten Resultaten der Aufgabenstellung
  erreicht. Die einzelnen Punkte sind hier analog der Aufgabenstellung
  aufgeführt:
  
  \begin{description}
    \item [Noch nicht Erreicht] Ergebnis der Analyse von bestehenden Java Swing
    Applikationen der Zürcher Kantonalbank.
    \item [Noch nicht Erreicht] Kategorisierung von verwendeten
    Swingkomponenten.
    \item [Noch nicht Erreicht] Eine Auflistung von etablierten Java Web
    Frameworks.
    \item [Noch nicht Erreicht] Ergebnis der Analyse, ob eine Integration der
    Java Web Frameworks, in der bestehenden IT Infrastruktur der Zürcher
    Kantonalbank, möglich ist.
    \item [Noch nicht Erreicht] Ergebnis der Analyse von Java Web Frameworks, ob
    eine Implementierung, der erkannten Swingkomponenten, möglich ist.
    \item [Noch nicht Erreicht] Proof of concept. Es soll anhand eines
    Prototypen gezeigt werden, dass die Implementierung möglich ist.
    \item [Noch nicht Erreicht] Eine Empfehlung für ein Java Web Framework.
  \end{description}
  
  % ===========================================================================
  % Kapitel Beschreibung der Ausgangslage beginns here
  % ===========================================================================
  
  \chapter{Beschreibung der Ausgangslage}
  
  Für das Informatik Diplomstudium an der Fachhochschule Zürich für Technik
  HSZ-T wird von den Studenten verlangt eine Diplomarbeit eigenständig zu
  verfassen.
    
  \section{Wahl des Themas}
  
  bla bla bla
  
  \section{Formell}
  
  Es wird auf die Sprache, die Richtlinien und die verwendeten
  Bewertungskriterien hingewiesen.
  
  \subsection{Sprache}
  
  Der Bericht wird in deutscher Sprache verfasst. Englische Ausdrücke werden im
  Kontext verwendet, wenn man davon ausgehen kann, dass es gängige Ausdrücke aus
  dem Gebiet der Informatik sind.
    
  \subsection{Richtlinien}
  Folgende Dokumente mit Richtlinien der Hochschule für Technik Zürich wurden
  für die Diplomarbeit berücksichtigt:

  \begin{itemize}
      \item Bestimmungen für die Diplomarbeit \cite{hsz_reglement}
      \item Ablauf Diplomarbeit \cite{hsz_ablauf}
      \item Bewertungskriterien Diplomarbeit \cite{hsz_bewertungskriterien}
  \end{itemize} 
    
  \subsection{Bewertungskriterien}
  
  Es werden die Bewertungskriterien für die Diplomarbeit gemäss
  \cite{hsz_bewertungskriterien} verwendet.
  
  \newpage
  
  \chapter{Projektadministration}
  
  Damit eine Nachvollziehbarkeit garantiert ist, werden hier die wichtigsten
  Termine, Meilensteine und Arbeitsschritte aufgelistet.
  
  \section{Projekt Termine}
  
  Die Projekt Termine wurden alle eingehalten, siehe Tabelle \ref{tab:termine}.
  \newline
  
  \begin{table}[h]
    \begin{center}
      \begin{tabular}{lp{7cm}ll}
        \toprule
        Termin & Datum & Ort \\
        \midrule
        21. 03. 2011 & Inhaltliches Kick-off Meeting & Panter llc\\
        13. 04. 2011 & Offizielles Kick-off Meeting & HSZ-T\\
        xx. xx. 2011 & Design-Review Meeting & HSZ-T\\
        xx. xx. 2011 & Abgabe der Dokumentation & HSZ-T\\
        xx. xx. 2011 & Schlusspräsentation & HSZ-T\\
        \bottomrule
      \end{tabular}
      \caption{Projekt Termine}
      \label{tab:termine}
    \end{center}
  \end{table}
  
  \section{Projekt Meilensteine}
  
  In der Projekt Historie sind die wichtigsten Meilensteine ersichtlich, siehe
  Tabelle \ref{tab:projekthistorie}.
  \newline
  
  \begin{table}[h]
    \begin{center}
      \begin{tabular}{lp{9cm}ll}
        \toprule
        Datum & Status & Wer \\
        \midrule
        14. 03. 2011 & Ein Dozierender hat die Arbeit inkl. Aufgabenstellung
        ausgeschrieben und wartet auf einen Studierenden der  diese Arbeit
        durchführt & Roman Würsch\\
        14. 03. 2011 & Eingabe Aufgabenstellung & Roman Würsch\\
        15. 03. 2011 & Die Arbeit ist freigegeben (Eine Semester- oder
        Bachelorarbeit kann nur durch die Studiengangsleitung freigegeben
        werden) & Olaf Stern\\
        16. 03. 2011 & Der Kickoff-Termin wurde reserviert & Roman Würsch\\
        \bottomrule
      \end{tabular}
      \caption{Projekt Historie}
      \label{tab:projekthistorie}
    \end{center}
  \end{table}
  
  \section{Arbeitsschritte}
  
  Alle vorgenommenen Arbeitsschritte werden in einem Wiki für die
  Nachvollziehbarkeit protokolliert. Das Wiki ist im Internet öffentlich
  zugänglich unter der URL:
  \url{https://github.com/sushicutta/Diplomarbeit/wiki/Arbeitsprotokoll}
  
  \newpage
  
  % ===========================================================================
  % Kapitel XXX beginns here
  % ===========================================================================
  
  \chapter{Java Swing Applikationen}

  \section{Grundlagen}
  
  \subsection{JAVA}
  
  \subsection{Programmiermodell}
  
  \subsubsection{Komponenten}
  
  \subsubsection{Layouts}
    
  \subsubsection{Events}
  
  \subsection{Look \& Feel}
  
  \subsection{Multithreading}
  
  \subsection{Konkurenz (SWT, \ldots)}
  
  \chapter{Rich Internet Applikationen}
  
  \section{Grundlagen}

  \subsection{Browser orientiert}
  
  \subsection{Plugin orientiert}
  
  \section{Technologien}
  
  \subsection{Eingesetzte Standards}
  
  \subsection{Programmiermodell}
  
  \section{Security}
  
  \section{Merkmale}
    
  \subsection{Suchmaschinenoptimierung}
  
  \subsection{Barrierefreiheit}
  
  \chapter{Analyse der Swing Applikationen}
  
  \section{Auswahl der Applikationen}
  
  \section{Kategorisierung von verwendeten Swing Komponenten}
  
  \chapter{Infrastruktur der Zürcher Kantonalbank}
  
  \section{Web Application Stack}
  
  \section{Business Layer}
  
  \section{Frontend}
  
  \chapter{Analyse der Web Frameworks}
  
  \section{Auswahl der Web Frameworks}
  
  \section{Pros und Contras}
  
  \section{Ist eine Umsetzung theoretisch möglich?}
  
  \section{Proof of Concept - Umsetzung durch einen Prototypen}
  
  \chapter{Integration in die ZKB Infrastruktur möglich?}
  
  \chapter{Empfehlung für ein Java Web Framework}
  
  \chapter{Reflektion}
  
  Rückblickend war es schade, dass ich diese Arbeit zu diesem Zeitpunkt
  schreiben musste, da mein Sohn Linus im Dezember zur Welt gekommen ist, und
  ich insgesamt über 100 Stunden vor meinen Rechnern verbracht habe, um alles
  zu vollenden. Natürlich liegt die Schuld voll und ganz bei mir selber, da ich
  viel früher mit der Semesterarbeit hätte beginnen können.
  
  Schlussendlich war die Arbeit in meinen Auge ein voller Erfolg, da ich alle
  Ziele erreicht und viel gelernt habe, und dank der unkomplizierten Art und Weise von
  Beat Seeliger, in der Betreuerrolle, mein Bestes zum Vorschein gebracht habe.
  Ich bin glücklich, dass es vorbei ist, und bereit für die Diplomarbeit.
  
  % ===========================================================================
  % Anhang beginns here
  % ===========================================================================
  
  \appendix
  
%  \chapter{Ein Testanhang}
%  
%  Im Anhang kann auf Implementierungsaspekte wie Datenbankschemata
%  oder Programmcode eingegangen werden.
  
  % ===========================================================================
  % Abkürzungsverzeichnis beginns here
  % ===========================================================================
  
  \chapter{Abkürzungsverzeichnis}
  \begin{acronym}
    \setlength{\itemsep}{-\parsep}
    \acro{ABC}{Alphabeth}
  \end{acronym}
  
  % ===========================================================================
  % Abbildungsverzeichnis beginns here
  % ===========================================================================
  
  % Abbildungsverzeichnis
  \listoffigures
  
  % ===========================================================================
  % Tabellenverzeichnis beginns here
  % ===========================================================================
  
  % Tabellenverzeichnis
  \listoftables
  
  % ===========================================================================
  % Literaturverzeichnis beginns here
  % ===========================================================================
  
  % verwendet alpha
  \bibliographystyle{alpha}
  % verwendet Literaturverzeichnis.bib
  % \renewcommand\bibname{Literaturverzeichnis} % Titel überschreiben
  \cleardoublepage
  \bibliography{Literaturverzeichnis}

\end{document}