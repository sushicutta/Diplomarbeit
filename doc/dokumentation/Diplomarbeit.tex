%
%   Technische Dokumentation für die Diplomarbeit
%
%   Created by Roman Wuersch on 2011-03-01.	
%
% =============================================================================
% Documentdefinition  beginns here
% =============================================================================

\documentclass[
11pt, % Schriftgrösse
a4paper, % A4 Papier
BCOR25mm, % Absoluter Wert der Bindekorrektur, z.B. BCOR1cm
DIV14, % Satzspiegel festlegen siehe
       % http://www.ctex.org/documents/packages/nonstd/koma-script.pdf
footsepline = false, % Trennlinie zwischen Textkörper und Fußzeile
                     % bei normalen Seiten
headsepline, % Trennlinie zwischen Kopfzeile und Textkörper
             % bei normalen Seiten
twoside, % Zweiseitig
openright,
%halfparskip, % Europäischer Satz mit Abstand zwischen den Absätzen
abstracton, % inkl. Abstract
listof=totocnumbered, % Abb.- und Tab.verzeichnis im Inhaltsverzeichnis
bibliography=totocnumbered % Lit.zeichnis in Inhaltsverzeichnis aufnehmen
]{scrreprt}

\usepackage[
automark,
%plainheadsepline, % Trennlinie zwischen Textkörper und Kopfzeile
                   % bei Chapter Seiten, siehe headsepline, scrreprt
%plainfootsepline  % Trennlinie zwischen Textkörper und Fußzeile
                   % bei Chapter Seiten, siehe footsepline, scrreprt
]{scrpage2} % Gestaltung von kopf- und Fußzeile

\usepackage[ngerman]{babel}

\usepackage[ngerman]{translator}

\usepackage{tocbasic}

% Use utf-8 encoding for foreign characters
\usepackage[utf8]{inputenc}

\usepackage{lmodern} % Latin Modern

% \usepackage[applemac]{inputenc}
\usepackage[T1]{fontenc}

% Setup for fullpage use
%\usepackage{fullpage}

% Silbentrennung kann unterdrückt werden
\usepackage{hyphenat}

%1.5 Zeilenabstand
\usepackage[onehalfspacing]{setspace}

% Schöne Schriften für PDF-Dateien
\usepackage{ae}

% Tradmark
\def\TTra{\textsuperscript{\texttrademark}}

% Festlegung des Seitenstils (scrpage2)
\pagestyle{scrheadings}
\clearscrheadfoot
\automark[chapter]{section}

\lehead{\sffamily\upshape\headmark}
\cehead{}
\rehead{}
\lefoot[\pagemark]{\upshape \pagemark}
\cefoot{}
\refoot{}

\lohead{}
\cohead{}
\rohead{\sffamily\upshape\headmark}
\lofoot{}
\cofoot{}
\rofoot[\pagemark]{\scshape \pagemark}

% Running Headers and footers
%\usepackage{fancyhdr}
%\setlength{\headheight}{15pt}

%\pagestyle{fancy}
%\renewcommand{\chaptermark}[1]{\markboth{\thechapter\ #1}{}}
%\renewcommand{\sectionmark}[1]{\markright{\thesection\ #1}{}}

%\fancyhf{}
%\fancyfoot[RE,LO]{\thepage}
%\fancyhead[LO]{\textit{\nouppercase{\leftmark}}}
%\fancyhead[RE]{\textit{\nouppercase{\rightmark}}}

%\renewcommand{\headrulewidth}{0.5pt}
%\renewcommand{\footrulewidth}{0pt}

%\fancypagestyle{plain}{%
%\fancyhf{}%
%\fancyfoot[RE,LO]{\thepage}
%\renewcommand{\headrulewidth}{0pt}
%\renewcommand{\footrulewidth}{0pt}
%}

%\setlength{\footskip}{1cm}

%\setlength{\headsep}{1.5cm}
%\addtolength{\textheight}{-1.5cm}

% Multipart figures
%\usepackage{subfigure}

% More symbols
%\usepackage{amsmath}
%\usepackage{amssymb}
%\usepackage{latexsym}

% Surround parts of graphics with box
\usepackage{boxedminipage}

% Package for including code in the document
\usepackage{listings}

% If you want to generate a toc for each chapter (use with book)
\usepackage{minitoc}

% Abkürzungsverzeichnis erstellen.
\usepackage[printonlyused]{acronym}

% schöne Tabelle zeichnen
\usepackage{booktabs}
\renewcommand{\arraystretch}{1.4} %Die Zeilenabstände in Tabllen angepasst.

% für variable Breiten
\usepackage{tabularx}

% Durchgestrichener Text
\usepackage[normalem]{ulem} %emphasize weiterhin kursiv

% This is now the recommended way for checking for PDFLaTeX:
\usepackage{ifpdf}

\usepackage[hyperfootnotes=false]{hyperref}
\hypersetup{
  bookmarks=true,         % show bookmarks bar?
  unicode=true,           % non-Latin characters in Acrobat’s bookmarks
  pdftoolbar=true,        % show Acrobat’s toolbar?
  pdfmenubar=true,        % show Acrobat’s menu?
  pdffitwindow=true,      % window fit to page when opened
  pdfstartview={FitH},    % fits the width of the page to the window
  plainpages=false,
  pdftitle={Diplomarbeit},   
  pdfauthor={Roman Würsch},
  pdfsubject={Evaluation eines Java Web Frameworks zur Ablösung bestehender Java
  Swing Applikationen},
  pdfcreator={TeX Live 2009},
  pdfproducer={pdfTeX, Version 3.1415926-1.40.10},
  pdfnewwindow=true,      % links in new window
  colorlinks=true,        % false: boxed links; true: colored links
  linkcolor=blue,         % color of internal links
  citecolor=green,        % color of links to bibliography
  filecolor=magenta,      % color of file links
  urlcolor=cyan           % color of external links
	% linkcolor=black,      % color of internal links
	% citecolor=black,      % color of links to bibliography
	% filecolor=black,      % color of file links
	% urlcolor=black        % color of external links
}

\ifpdf
    \usepackage[pdftex]{graphicx}
\else
    \usepackage{graphicx}
\fi

\makeatletter 
\let\orgdescriptionlabel\descriptionlabel 
\renewcommand*{\descriptionlabel}[1]{% 
  \let\orglabel\label 
  \let\label\@gobble 
  \phantomsection 
  \edef\@currentlabel{#1}% 
  %\edef\@currentlabelname{#1}% 
  \let\label\orglabel 
  \orgdescriptionlabel{#1}% 
} 
\makeatother

\title{Evaluation eines Java Web Frameworks zur Ablösung bestehender Java Swing
Applikationen}

\author{Diplomarbeit in Informatik\\
    \\
    Studierender - Roman Würsch\\
	Auftraggeber - Bernhard Mäder, ZKB\\
    Projektbetreuer - Beat Seeliger\\
    Experte - tbd\\
	\\
	HSZ-T - Technische Hochschule Zürich}

\date{März 2011 bis Juni 2011}

% =============================================================================
% Documenttext beginns here
% =============================================================================

\begin{document}

  \ifpdf
    \DeclareGraphicsExtensions{.pdf, .jpg, .tif}
  \else
    \DeclareGraphicsExtensions{.eps, .jpg}
  \fi
  
  % ===========================================================================
  % Titelblatt beginns here
  % ===========================================================================
  
  \maketitle
  
  \cleardoublepage
  
  % ===========================================================================
  % Abstract beginns here
  % ===========================================================================
  
  % \pagenumbering{Alph}
  
  \begin{abstract}

  Es sollen Methoden zur Analyse von Java Swing Applikationen und zur Evaluation
  von Java Web Frameworks ausgearbeitet werden.

  Mit diesen Methoden sollen bestehende Java Swing Applikationen der Zürcher
  Kantonalbank analysiert werden. In den Applikationen sollen die gemeinsamen
  Muster, genutzter Swingkomponenten, erkannt und kategorisiert werden. Java
  Web Frameworks, welche sich am Markt etabliert haben, sollen durch einen
  Evaluationsprozess auf deren Einsatz in der Zürcher Kantonalbank geprüft
  werden. Zusätzlich soll geprüft werden, ob mit den jeweiligen Frameworks
  die genutzten Swingkomponenten äquivalent umgesetzt werden können. Mit
  einer Analyse der IT Infrastruktur der Zürcher Kantonalbank, soll geprüft
  werden, ob eine Integration in die bestehende IT Infrastruktur der möglich
  ist.

  Durch die Umsetzung eines Prototypen soll gezeigt werden, dass das evaluierte
  Java Web Framework den geforderten Funktionalitäten entspricht.

  Mit den gewonnenen Erkenntnissen, soll eine Empfehlung, für den Einsatz
  eines Java Web Frameworks zur Ablösung bestehender Java Swing Applikationen,
  ausgesprochen werden.
  
\end{abstract}
  
  \cleardoublepage
  
  % ===========================================================================
  % Inahltsverzeichnis beginns here
  % ===========================================================================

  \pagenumbering{roman}
  
  \tableofcontents
  
  \cleardoublepage
  
  % ===========================================================================
  % Kapitel Personalienblatt beginns here
  % ===========================================================================
  
  \pagenumbering{arabic}
  
  \chapter{Personalienblatt}

  \begin{tabbing}
  \hspace*{6cm}\= \kill
  Name, Vorname: \> {\bf Roman Würsch} \\
  Adresse: \> {\bf Murhaldenweg 16} \\
  PLZ, Wohnort: \> {\bf 8057 Zürich} \\
  \\
  Geburtsdatum: \> {\bf 10.11.1980} \\
  Heimatort: \> {\bf Emmetten NW} \\
\end{tabbing}

\noindent
Ich bestätige, dass die vorliegende Diplomarbeit, ``Evaluation eines Java Web
Frameworks zur Ablösung bestehender Java Swing Applikationen'', in allen Teilen
selbständig erarbeitet und durchgeführt wurde.

\vspace*{3cm}

\begin{tabbing}
  \hspace*{8cm}\= \kill
  Ort und Datum \> {Unterschrift} \\
\end{tabbing}
  
  %\cleardoublepage
      
  % ===========================================================================
  % Kapitel Rahmenbedingungen beginns here
  % ===========================================================================
  
  \chapter{Rahmenbedingungen}
  
    Für das Informatik Diplomstudium an der Fachhochschule Zürich für Technik
  HSZ-T wird von den Studenten verlangt eine Diplomarbeit eigenständig zu
  verfassen.
  
  \section{Sprache}
  
  Der Bericht wird in deutscher Sprache verfasst. Englische Ausdrücke werden im
  Kontext verwendet, wenn man davon ausgehen kann, dass es gängige Ausdrücke aus
  dem Gebiet der Informatik sind. Um Schwerfälligkeiten im Text zu vermeiden,
  habe ich mich auf die männliche Form beschränkt. Selbstverständlich sind bei
  allen Formulierungen beide Geschlechter angesprochen.
    
  \section{Richtlinien}
  Folgende Dokumente mit Richtlinien der Hochschule für Technik Zürich wurden
  für die Diplomarbeit berücksichtigt:

  \begin{itemize}
      \item Bestimmungen für die Diplomarbeit \cite{hsz_reglement}
      \item Ablauf Diplomarbeit \cite{hsz_ablauf}
      \item Bewertungskriterien Diplomarbeit \cite{hsz_bewertungskriterien}
  \end{itemize} 
    
  \section{Bewertungskriterien}
  
  Es werden die Bewertungskriterien für die
  Diplomarbeit\cite{hsz_bewertungskriterien} verwendet.
  
%  \cleardoublepage
      
  % ===========================================================================
  % Kapitel Einführung beginns here
  % ===========================================================================
  
  \chapter{Einführung}
  
    Das Internet hat sich in den letzten Jahren zu einem strategisch wichtigen
  Vertriebskanal entwickelt. Viele Unternehmen haben das erkannt und setzen im
  Betriebsalltag die Möglichkeiten, die das Internet bietet, ein. In der
  Finanzindustrie wurde dieser Schritt mit dem Online-Banking gegen Ende der
  Neunzigerjahre gemacht. Das eine Onlinestrategie funktioniert und die
  Nachfrage dafür besteht hat sich in den letzten Jahren gezeigt.
  
  Die Informatik spielt in der Finanzindustrie im Allgemeinen eine Wichtige
  Rolle. Viele Finanzinstitute hier in der Schweiz zählen heute zu den grössten
  Arbeitgebern in der Informatik\cite{WoArbeitenInformatikfachleute}. Insgesammt
  gibt es in der Schweiz acht Unternehmen, welche selber nicht aus der
  Informatik Branche kommen, die mehr als 500 Informatikfachleute beschäftigen.
  Die Hälfte davon sind Finanzinstitute, namentlich sind das Credit Suisse,
  UBS, Zürich Versicherung und die Post. In der Zürcher Kantonalbank ist die
  Informatik ebenfalls stark vertreten.
  
  Durch die zunehmende Komplexität im Bankengeschäft, reichen manchmal käuflich
  erwerbliche Softwarelösungen nicht vollends. Um den Vorsprung zur Konkurenz
  auszubauen, entwickeln die Banken vielmals Computerprogramme selber, welche
  die Automatisierung ihrer Prozesse übernehmen, oder mit denen
  finanzmathematische Modelle berechnet werden können.
  
  Aus einer solchen Lösung heraus kann sich ein Geschäft für die Bank
  entwicklen. Eine Software, die von einem Finanzinstitut selber entwickelt
  wurde, könnte beispielsweise externen Vermögensverwaltern zur Verfügung
  gestellt werden. Dabei gibt es verschiedene Vertriebskanäle. Was sich mit dem
  Onlinebanking bewährt hat, kann sich auch für bestehende Lösungen bewähren.
  
  In der Zürcher Kantonalbank existieren viele solcher Lösungen bereits, sind
  aber nicht für eine Online-Strategie entwickelt worden, sonder für den
  internen Einsatz. Einige solcher Programme wurden als Desktop Applikationen
  entwickelt, das entsprach durchaus den Anforderungen für einen internen
  Einsatz. Um den Vertrieb zentral verwalten zu können, macht das Umrüsten
  auf eine Online-Lösung, durchaus Sinn.
  
  \section{Motivation}
  
  Die Zürcher Kantonalbank setzt bei der Entwicklung von Software als Inhouse
  Lösungen auf Java Swing Applikationen und auf Java Web Applikationen. Die
  Java Web Applikationen basieren auf dem Web Framework Apache Struts 1.3
  mit einer von der ZKB erstellten Erweiterung namens ZIP (ZKB Internet
  Plattform). Da eine Lösung als Java Web Applikationen zentral verwaltet
  werden kann gibt es erhebliche Einsparungen im Bereich Testing, Deployment
  und Patching. Somit sollen in Zukunft Java Swing Applikationen auf Java Web
  Applikationen umgerüstet werden.
  
  Das Framework Apache Struts 1.3 ist mitlerweilen in die Jahre gekommen. Es
  basiert auf dem Prinzip von ``request und response''. Wann immer eine Aktion
  von einem User in der Applikation gemacht wird, sei es das Ansteuern eines
  Links oder das Absenden eines Formulars, wird die Webseite, welche die
  Applikation repräsentiert, neu geladen. Heutzutage gibt es Frameworks, welche
  einem ermöglichen eine Web Applikation zu entwicklen, die sich bei der
  Bedienung wie eine klassische Desktop Applikation anfühlen. Dies wird durch
  die Technik von \ac{Ajax} realisiert.
  
  Bei \ac{Ajax} können Daten einer Web Applikation, ohne die komplette Webseite
  neu zu laden, verändert werden. Dies erlaubt es Web Applikationen, auf
  Benutzer Aktionen schneller zu reagieren, da vermieden wird, dass statische
  Daten, die sich unter Umständen nicht geändert haben, immer wieder über das
  Netzwerk übertragen werden müssen.
  
  Apache Struts 1.3 unterstützt \ac{Ajax} nicht direkt, das kann aber über
  Erweiterungen ermöglicht werden. Es gibt Java Web Frameworks, welche diese
  Technik out of the box mitbringen. Eine Evaluation soll zeigen welches dieser
  Java Web Frameworks am besten für einen möglichen Einsatz geeignet ist.
  
  \section{Zielsetzung}
  
  Gegenstand der vorliegenden Diplomarbeit ist die Analyse welche Java Web
  Frameworks für die Ablösung von bestehenden Java Swing Applikationen in Frage
  kommen. Es sollen anhand eines strukturierten Vorgehens eine Menge von
  bestehenden Java Swing Applikationen der Zürcher Kantonalbank auf deren
  Funktionalitäten der Benutzeroberfläche eruiert werden. Aufgrund der
  Anforderungen der IT-Architektur der Zürcher Kantonalbank sollen Java Web
  Frameworks auf die Validität der erhobenen Funktionalitäten verifiziert
  werden. Die Java Web Frameworks, welche diesen Ansprüchen genügen, sollen auf
  einen möglichen Einsatz in der Informatik Infrastruktur der Züricher
  Kantonalbank untersucht werden. Durch die Implementation eines Prototyen soll
  gezeigt werden, dass die Umsetzung möglich ist. Schlussendlich soll eine
  Empfehlung für ein Java Web Framework ausgesprochen werden, für den Fall,
  dass eine bestehende Java Swing Applikation in eine Web Applikation umgebaut
  werden soll. Ebenso sollen die gewonnen Erkenntnisse, für zukünftige
  Anpassungen im Bereich der Java Web Frameworks der IT-Architektur der Zürcher
  Kantonalbank, als Grundlage dienen.
  
  \section{Struktur}
  
  tbd
  
  \section{Danksagung}
  
  tbd
    

  \cleardoublepage
 
  % ===========================================================================
  % Kapitel Aufgabenstellung beginns here
  % ===========================================================================
  
  \chapter{Aufgabenstellung}
  
    Die Aufgabenstellung besteht aus den vier Teilen: Ausgangslage, Ziel der
  Arbeit, Aufgabenstellung und Erwartete Resultate. Zusätzlich wurde eine
  Abgrenzung hinzugefügt, damit der Rahmen der Diplomarbeit gesetzt ist.
  
  \section{Ausgangslage}
  
  Die Zürcher Kantonalbank hat viele Applikationen welche als Client Applikationen
in Swing implementiert sind. Dabei ist das Deployment der Clients und die
Ausbreitung entsprechender Patches, innerhalb der Zürcher Kantonalbank, mit
einem Mehraufwand verbunden. Client Applikationen, die einem Kunden zur
Verfügung gestellt werden, sind an eine spezifische Java Version gebunden. Bei
der Integration in die IT-Landschaft beim Kunden, kann das zur Verzögerung
durch einen erhöhten Testaufwand führen. In der Annahme, dass ein Webbrowser in
der Zürcher Kantonalbank und bei deren Kunden eingesetzt wird, macht der
Einsatz einer Weblösung Sinn.

  
  \section{Ziel der Arbeit}

  Es sollen bestehende Java Swing Applikationen der Zürcher Kantonalbank
analysiert werden. In diesen Applikationen sollen die gemeinsamen Muster,
genutzter Swingkomponenten, erkannt und kategorisiert werden. Java Web
Frameworks, welche sich am Markt etabliert haben, sollen auf einen möglichen
Einsatz geprüft werden. Es soll geprüft werden, ob mit den jeweiligen
Frameworks die genutzten Swingkomponenten äquivalent umgesetzt werden können.
Zudem soll geprüft werden, ob eine Integration in die bestehende IT
Infrastruktur der Zürcher Kantonalbank möglich ist.

  
  \section{Aufgabenstellung}
  
    Die Aufgabenstellung besteht aus den vier Teilen: Ausgangslage, Ziel der
  Arbeit, Aufgabenstellung und Erwartete Resultate. Zusätzlich wurde eine
  Abgrenzung hinzugefügt, damit der Rahmen der Diplomarbeit gesetzt ist.
  
  \section{Ausgangslage}
  
  Die Zürcher Kantonalbank hat viele Applikationen welche als Client Applikationen
in Swing implementiert sind. Dabei ist das Deployment der Clients und die
Ausbreitung entsprechender Patches, innerhalb der Zürcher Kantonalbank, mit
einem Mehraufwand verbunden. Client Applikationen, die einem Kunden zur
Verfügung gestellt werden, sind an eine spezifische Java Version gebunden. Bei
der Integration in die IT-Landschaft beim Kunden, kann das zur Verzögerung
durch einen erhöhten Testaufwand führen. In der Annahme, dass ein Webbrowser in
der Zürcher Kantonalbank und bei deren Kunden eingesetzt wird, macht der
Einsatz einer Weblösung Sinn.

  
  \section{Ziel der Arbeit}

  Es sollen bestehende Java Swing Applikationen der Zürcher Kantonalbank
analysiert werden. In diesen Applikationen sollen die gemeinsamen Muster,
genutzter Swingkomponenten, erkannt und kategorisiert werden. Java Web
Frameworks, welche sich am Markt etabliert haben, sollen auf einen möglichen
Einsatz geprüft werden. Es soll geprüft werden, ob mit den jeweiligen
Frameworks die genutzten Swingkomponenten äquivalent umgesetzt werden können.
Zudem soll geprüft werden, ob eine Integration in die bestehende IT
Infrastruktur der Zürcher Kantonalbank möglich ist.

  
  \section{Aufgabenstellung}
  
    Die Aufgabenstellung besteht aus den vier Teilen: Ausgangslage, Ziel der
  Arbeit, Aufgabenstellung und Erwartete Resultate. Zusätzlich wurde eine
  Abgrenzung hinzugefügt, damit der Rahmen der Diplomarbeit gesetzt ist.
  
  \section{Ausgangslage}
  
  \input{../aufgabenstellung/kapitel/ausgangslage.tex}
  
  \section{Ziel der Arbeit}

  \input{../aufgabenstellung/kapitel/zielDerArbeit.tex}
  
  \section{Aufgabenstellung}
  
  \input{../aufgabenstellung/kapitel/aufgabenstellung.tex}
  
  \section{Erwartete Resultate}
  
  \input{../aufgabenstellung/kapitel/erwarteteResultate.tex}
  
  \section{Abgrenzung}
  
  \input{../aufgabenstellung/kapitel/abgrenzung.tex}
  
  \section{Erwartete Resultate}
  
  Der Studierende soll dem Auftraggeber ein Dokument erstellen, das folgendes
beinhaltet:

\begin{itemize}    
  \item Ergebnis der Analyse von bestehenden Java Swing Applikationen der
  Zürcher Kantonalbank.
  \item Kategorisierung von verwendeten Swingkomponenten.
  \item Eine Auflistung von etablierten Java Web Frameworks.
  \item Ergebnis der Analyse, ob eine Integration der Java Web Frameworks, in
  der bestehenden IT Infrastruktur der Zürcher Kantonalbank, möglich ist.
  \item Ergebnis der Analyse von Java Web Frameworks, ob eine Implementierung,
  der erkannten Swingkomponenten, möglich ist.
  \item Proof of concept. Es soll anhand eines Prototypen gezeigt werden, dass
  die Implementierung möglich ist.
  \item Eine Empfehlung für ein Java Web Framework.
\end{itemize}
  
  \section{Abgrenzung}
  
  Folgende Punkte werden formell abgegrenzt:

\begin{itemize}
  \item Die Analysen beschränken sich auf Recherchen im Internet, Büchern und
  interne Vorgaben der Zürcher Kantonalbank.
  \item Umfragen, Erhebungen sowie Feldstudien werden nicht durchgeführt.
\end{itemize}

\noindent  
Folgende Punkte werden inhaltlich abgegrenzt:
	
\begin{itemize}
  \item Die Auswahl, welche Java Swing Applikationen analysiert werden, soll
  wärend der Arbeit durchgeführt werden.
  \item Die Auswahl, welche Java Web Frameworks geprüft werden, soll wärend  der
  Arbeit durchgeführt werden.
\end{itemize}
  
  \section{Erwartete Resultate}
  
  Der Studierende soll dem Auftraggeber ein Dokument erstellen, das folgendes
beinhaltet:

\begin{itemize}    
  \item Ergebnis der Analyse von bestehenden Java Swing Applikationen der
  Zürcher Kantonalbank.
  \item Kategorisierung von verwendeten Swingkomponenten.
  \item Eine Auflistung von etablierten Java Web Frameworks.
  \item Ergebnis der Analyse, ob eine Integration der Java Web Frameworks, in
  der bestehenden IT Infrastruktur der Zürcher Kantonalbank, möglich ist.
  \item Ergebnis der Analyse von Java Web Frameworks, ob eine Implementierung,
  der erkannten Swingkomponenten, möglich ist.
  \item Proof of concept. Es soll anhand eines Prototypen gezeigt werden, dass
  die Implementierung möglich ist.
  \item Eine Empfehlung für ein Java Web Framework.
\end{itemize}
  
  \section{Abgrenzung}
  
  Folgende Punkte werden formell abgegrenzt:

\begin{itemize}
  \item Die Analysen beschränken sich auf Recherchen im Internet, Büchern und
  interne Vorgaben der Zürcher Kantonalbank.
  \item Umfragen, Erhebungen sowie Feldstudien werden nicht durchgeführt.
\end{itemize}

\noindent  
Folgende Punkte werden inhaltlich abgegrenzt:
	
\begin{itemize}
  \item Die Auswahl, welche Java Swing Applikationen analysiert werden, soll
  wärend der Arbeit durchgeführt werden.
  \item Die Auswahl, welche Java Web Frameworks geprüft werden, soll wärend  der
  Arbeit durchgeführt werden.
\end{itemize}

  \cleardoublepage
   
  % ===========================================================================
  % Kapitel Erreichte Ziele beginns here
  % ===========================================================================  
  
  \chapter{Erreichte Ziele}
  
    Es wurden alle Ziele gemäss den erwarteten Resultaten der Aufgabenstellung
  erreicht. Die einzelnen Punkte sind hier analog der Aufgabenstellung
  aufgeführt:
  
  \begin{description}
    \item [Noch nicht Erreicht] Ergebnis der Analyse von bestehenden Java Swing
    Applikationen der Zürcher Kantonalbank.
    \item [Noch nicht Erreicht] Kategorisierung von verwendeten
    Swingkomponenten.
    \item [Noch nicht Erreicht] Eine Auflistung von etablierten Java Web
    Frameworks.
    \item [Noch nicht Erreicht] Ergebnis der Analyse, ob eine Integration der
    Java Web Frameworks, in der bestehenden IT Infrastruktur der Zürcher
    Kantonalbank, möglich ist.
    \item [Noch nicht Erreicht] Ergebnis der Analyse von Java Web Frameworks, ob
    eine Implementierung, der erkannten Swingkomponenten, möglich ist.
    \item [Noch nicht Erreicht] Proof of concept. Es soll anhand eines
    Prototypen gezeigt werden, dass die Implementierung möglich ist.
    \item [Noch nicht Erreicht] Eine Empfehlung für ein Java Web Framework.
  \end{description}

  \cleardoublepage
   
  % ===========================================================================
  % Kapitel Projektadministration beginns here
  % ===========================================================================  
    
  \chapter{Projektadministration}
 
    Durch eine Detailanalyse der Aufgabenstellung werden die wichtigsten
  Aufgaben und deren Resultate ausgearbeitet. Aufgrund dieser Erkenntnisse
  wird eine Grobplanung festgelegt. Damit die Nachvollziehbarkeit garantiert
  ist, werden die Termine, Meilensteine und Arbeitsschritte aufgelistet.
  
  \section{Detailanalyse der Aufgabenstelltung}
  
  \indent
  \indent
  ``\begin{itshape}Analyse bestehender Java Swing Applikationen der Zürcher
  Kantonalbank.\end{itshape}''
  \newline
  \newline
  \noindent
  Es sollen die gängigen Mechanismen von Java Swing Applikationen der Zürcher
  Kantonalbank untersucht werden, dass soll aus der Sicht des Anwenders
  passieren. In drei bestehenden Java Swing Applikationen soll die Existenz
  bekannter GUI Paradigmen untersucht werden.
  \newline
  \newline
  \noindent
  Resultat: Es soll eine Liste der erkannten Paradigmen vorliegen.
  \newline
  
  ``\begin{itshape}Erkennen und Kategorisieren der verwendeten
  Swingkomponenten.\end{itshape}''
  \newline
  \newline
  \noindent
  Die drei ausgewählten Java Swing Applikationen werden genauer betrachtet. Es
  wird das Augenmerk auf die Verwendung von Swingkomponenten gelegt. Diese
  sollen über die drei Applikationen hinweg konsolidiert und kategorisiert
  werden.
  \newline
  \newline
  \noindent
  Resultat: Es soll eine Liste der verwendeten Swingkomponenten vorliegen.
  \newline
  
  ``\begin{itshape}Evaluation von Java Web Frameworks, welche sich am Markt
  etabliert haben.\end{itshape}''
  \newline
  \newline
  \noindent
  Es soll ein Evaluationsverfahren gewählt werden, bei dem eine möglichst
  objektive Entscheidung gefällt werden kann. Welche Java Web Frameworks in die
  Evaluation miteinbezogen werden, soll über Recherchen im Internet und in
  Büchern geschehen. Es sollen fünf Java Web Frameworks gewählt werden, welche
  anhand der Recherchen für valable Optionen in Frage kommen. Über die
  Definition von Soll- und KO-Kriterien sollen die Rahmenbedigungen für das
  Evaluationsverfahren geschaffen werden. Anhand der ausgearbeiteten
  Evalautionsmethode soll gezeigt werden, wie geeignet die Java Web Frameworks
  wirklich sind.
  \newline
  \newline
  \noindent  
  Resultat: Es soll eine Rangliste der fünf Frameworks, in der Anordnung
  entsprechend ihrer Eignung, vorliegen.
  \newline

  ``\begin{itshape}Prüfen, ob eine Integration der evaluierten Java Web
  Frameworks, welche für eine Umsetzung geeignet sind, in der bestehenden IT
  Infrastruktur der Zürcher Kantonalbank möglich ist.\end{itshape}''
  \newline
  \newline
  \noindent
  Gemäss den Vorgaben der IT Infrastruktur der Zürcher Kantonalbank, soll ein
  mögliche Einsatz der evaluierten Java Web Frameworks geprüft werden. Die
  meisten Java Web Frameworks haben in ihrer Dokumentation die Anforderungen
  definiert, welche für einen möglichen Betrieb nötig sind. Aufgrund dieser
  Anforderungen, und der bestehenden IT Infrastruktur soll ein Vergleich gemacht
  werden.
  \newline
  \newline
  \noindent
  Resultat: Es sollen die Java Web Frameworks aufgelistet werden, welche für
  einen Einsatz in der IT Infrastruktur der Zürcher Kantonalbank in Frage kommen.
  \newline

  ``\begin{itshape}Prüfen, ob eine Implementierung der erkannten
  Swingkomponenten in den evaluierten Java Web Frameworks möglich
  ist.\end{itshape}''
  \newline
  \newline
  \noindent
  Die gewonnenen Erkenntnisse, aus der Analyse der Java Swing Applikationen,
  sollen nun mit der Liste, der in Frage kommenden Java Web Frameworks,
  zusammengeführt werden.
  \newline
  \newline
  \noindent
  Resultat: Es sollen die Java Web Frameworks aufgelistet werden, welche die
  notwendigen Swingkomponenten und GUI Paradigmen unterstützen.
  \newline

  ``\begin{itshape}Proof of concept. Erstellen eines Prototypen mit den
  evaluierten Java Web Frameworks und den erkannten
  Swingkomponenten.\end{itshape}''
  \newline
  \newline
  \noindent
  Das Java Web Framework, welches sich entsprechend der Evaluation am meisten
  für den Einsatz eignet und den Anforderungen der IT Infrastruktur und der
  notwendigen Swingkomponenten und GUI Paradigment genügt, soll sich anhand
  eines definierten Prototypen bewähren.
  \newline
  \newline
  \noindent
  Resultat: Es soll eine Empfehlung eines Java Web Frameworks, für den
  möglichen Einsatz in der Zürcher Kantonalbank, ausgesprochen werden.
  
  \section{Grobplanung}
  
  Die Grobplanung sieht man anhand der Grafik \ref{img:grobplanung}.
  
  \begin{figure}[ht]
    \begin{center}
      \includegraphics[width=0.7\textwidth]{./image/grobplanung.png}
      \caption{Grobplanung zur Diplomarbeit}
      \label{img:grobplanung}
    \end{center}
  \end{figure}
  
  \section{Termine}
  
  Die Projekt Termine wurden alle eingehalten, siehe Tabelle \ref{tab:termine}.
  \newline
  
  \begin{table}[ht]
    \begin{center}
      \begin{tabular}{lp{7cm}ll}
        \toprule
        Termin & Datum & Ort \\
        \midrule
        21. 03. 2011 & Inhaltliches Kick-off Meeting & Panter llc\\
        13. 04. 2011 & Offizielles Kick-off Meeting & HSZ-T\\
        xx. xx. 2011 & Design-Review Meeting & HSZ-T\\
        xx. xx. 2011 & Abgabe der Dokumentation & HSZ-T\\
        xx. xx. 2011 & Schlusspräsentation & HSZ-T\\
        \bottomrule
      \end{tabular}
      \caption{Projekt Termine}
      \label{tab:termine}
    \end{center}
  \end{table}
  
  \section{Meilensteine}
  
  In der Projekt Historie sind die wichtigsten Meilensteine ersichtlich, siehe
  Tabelle \ref{tab:projekthistorie}.
  \newline
  
  \begin{table}[ht]
    \begin{center}
      \begin{tabular}{lp{7cm}ll}
        \toprule
        Datum & Status & Wer \\
        \midrule
        14. 03. 2011 & Ein Dozierender hat die Arbeit inkl. Aufgabenstellung
        ausgeschrieben und wartet auf einen Studierenden der  diese Arbeit
        durchführt & Roman Würsch\\
        14. 03. 2011 & Eingabe Aufgabenstellung & Roman Würsch\\
        15. 03. 2011 & Die Arbeit ist freigegeben (Eine Semester- oder
        Bachelorarbeit kann nur durch die Studiengangsleitung freigegeben
        werden) & Olaf Stern\\
        16. 03. 2011 & Der Kickoff-Termin wurde reserviert & Roman Würsch\\
        \bottomrule
      \end{tabular}
      \caption{Projekt Historie}
      \label{tab:projekthistorie}
    \end{center}
  \end{table}
  
  \section{Arbeitsschritte}
  
  Alle vorgenommenen Arbeitsschritte werden in einem Wiki für die
  Nachvollziehbarkeit protokolliert. Das Wiki ist im Internet öffentlich
  zugänglich unter der \ac{URL}:\\
  \\
  \url{https://github.com/sushicutta/Diplomarbeit/wiki/Arbeitsprotokoll}.

  \cleardoublepage
   
  % ===========================================================================
  % Kapitel XXX beginns here
  % ===========================================================================
  
  \chapter{Java Swing Applikationen}

    Swing kommt aus dem Hause Oracle, ehemals Sun Microsystems, und ist ein
  Bestandteil der \ac{JFC}. Seit der Java Version 1.2 ist Swing Bestandteil der
  \ac{JRE}. Swing wurde in den letzten Jahren immer weiter ausgebaut und ist
  somit der Standard für die Entwicklung von Desktop- und Applet-Applikationen
  in Java.
  
  \section{Grundlagen}
  
  Unter Swing versteht man eine Reihe von leichtgewichtigen Komponenten zur
  Programmierung von grafischen Oberflächen. Mit leichtgewichtigen Komponenten
  meint man, dass alle Komponenten zu 100\% in Java geschrieben sind. Sie sind
  somit plattformübergreifend einsetzbar und sehen überall gleich aus. Die Swing
  Komponenten sind im \ac{JRE} unter dem Packet \(javax.swing\) eingeordnet,
  zusätzlich gibt es das Packet \(javax.accessibility\), welches zur
  Abstraktion des User Interfaces dient.
  
  \subsection{Komponenten}
  
  Gemäss \cite{SwingComponentsHighscore}, kennt Swing drei Hirarchien von
  Komponenten: Top-Level, Intermediate und Atomic Components.
  
  Unter Atomic Components versteht man einzelne Bausteine wie ein
  \(javax.swing.JButton\), ein einfacher Knopf, oder ein
  \(javax.swing.JTextField\), ein einfaches Textfeld.
  
  Intermediate Components bieten vielfältige Möglichkeiten an, um andere
  Intermediate Components zu unterteilen oder zu gruppieren. Zusätzlichen
  können sie auch eine beliebige Anzahl von Atomic Components enthalten. Gängige
  Vertreter sind \(javax.swing.JPanel\), eine Komponente zur Gruppierung anderer
  Komponenten, oder \(javax.swing.JSplitPane\), eine Komponente zur
  Unterteilung einer Komponente in zwei Teile.
  
  Die Top-Level Components sind \(javax.swing.JFrame\),
  zur Darstellung einer vollwertigen Fenster-Applikation,
  \(javax.swing.JDialog\), für Dialogfenster und \(javax.swing.JApplet\), zur
  Entwicklung von Java-Applets mit Swing.
  
  \subsection{Layouts}
  
  Viele Komponenten in Swing haben ein Layout, vorallem die Intermediate
  Components. Das Layout regelt die Anordnung der Komponenten und das Verhalten,
  falls sich die Grösse einer Komponente ändert. Swing definiert ein paar eigene
  Layouts, man kann aber auch die bestehenden Layouts aus dem Paket
  \(java.awt\), wie zum Beispiel \(java.awt.GridBagLayout\) verwenden. Swing
  bietet auch die Möglichkeit eigene Layouts zu definieren, was zum Beispiel
  JGoodies mit dem Freeware Projekt \(JGoodies Forms\) gemacht hat, siehe
  \cite{JGoodiesForms}.
    
  \subsection{Eventbasierte Kommunikation der Komponenten}
  
  Das Programmiermodell mit Java Swing ist Eventbasiert. Dabei können Events
  definiert werden, zum Beispiel ein Mausklick auf einen Knopf, bei welchen
  eine Aktion ausgeführt werden soll. Die Aktion könnte das Speichern eines
  Dokuments sein. Die eventbasierte Kommunikation zwischen den Swing
  Komponenten ist nach dem Observer Design Pattern implementiert, siehe
  \cite{ObserverDesignPattern}.
  
  \subsection{Hilfsmittel}
  
  Die Klasse \(javax.swing.SwingUtilities\) bietet eine vielzahl von
  Hilfsfunktionen, welche bei der Entwicklung von Swing Applikationen verwendet
  werden kann. Zusätzlich bietet Swing eine handvoll weiterer Klassen,
  welche einem das Leben als Programmierer erleichtert. Ein paar nennenswerte
  sind: \(javax.swing.UIManager\), um das aktuelle Look \& Feel zu managen,
  \(javax.swing.BorderFactory\), für das zeichnen von Rahmen um eine
  Komponente, oder auch \(java.swing.SwingWorkder<T,V>\), um asynchrone Tasks
  zu verarbeiten.
  
  \section{Pluggable Look \& Feel}
  
  Das Erscheinungsbild und das Verhalten der Swing Komponenten, auch genannt
  pluggable Look \& Feel, kann für alle Swing Komponenten separat definiert
  werden. Es lässt dem Programmiere die Möglichkeit offen, alle Komponenten
  individuell zu gestalten. Durch die Delegation an ein separates Objekt, kann
  das Look \& Feel zur Laufzeit ausgetauscht werden.
  
  \section{Multithreading}
  
  Swing ist zum grössten Teil nicht thread-save. Das heisst, auf ein Swing
  Objekt sollte nur mit einem Thread zugegriffen werden. Für den Zugriff und
  die Instanziirung von Java Swing Objekt steht der \ac{EDT} zur Verfügung,
  welcher alle Events, welche von einer Swing Komponente generiert wurde,
  abarbeitet.
  
  \subsection{Asynchrone Tasks}
  
  Damit der \ac{EDT} bei länger dauernden Tasks nicht blockert wird, stellt
  Swing das Konzept vom SwingWorker zur Verfügung, siehe \cite{SwingWorker}.
  Darüber kann der \ac{EDT} einen zusätzlichen Worker Tread starten. Durch das
  Abgeben von langandauernden Arbeiten an zusätzliche Threads, wird der
  \ac{EDT} nicht blockiert und kann sich so um das abarbeiten von Actions
  kümmern. Somit ist auch gewährleistet, das das \ac{GUI} dadurch nicht
  blockiert wird.


  \cleardoublepage
   
  % ===========================================================================
  % Kapitel XXX beginns here
  % ===========================================================================
   
  \chapter{Rich Internet Applikationen}
  
    \ac{RIA} ist kein Standard, sondern ein synonym für Applikationen, welche
  eine ``reichhaltige'' Benutzeroberfläche bieten und eine Verbindung mit dem
  Internet haben. Siehe \cite{RichInternetApplication} und
  \cite{RichInternetApplicationsWhitePaper} S. 2.
  
  \section{Grundlagen}
  
  Der Begriff \ac{RIA} ist mit der Entwicklung des Internets entstanden und
  wird heute oft verwendet. Für viele Leute ist \ac{RIA} ein synonmy für
  Webanwendungen, welche mit \ac{Ajax} realisiert werden. \ac{Ajax} bietet die
  Möglichkeit eine ``reichhaltige'' Benutzeroberfläche zu entwickeln, so wie
  man sich das von klassischen Desktopanwendungen gewöhnt ist. Die Grenze
  zwischen der klassischen Webanwendung und einer Desktopapplikation scheint
  damit zu verschwinden. Genauer betrachtet steht eine \ac{RIA} im
  Technologiespektrum aber zwischen dem Rich Client und dem Thin Client, siehe
  Grafik \ref{img:webanwendungen}.
  
  \begin{figure}[h]
    \begin{center}
      \includegraphics[width=0.7\textwidth]{./image/webanwendungen.png}
      \caption{Web Anwendungen (nach \cite{DiplomarbeitStephanSchuster} S. 6f.
      und \cite{WebApplicationSolutions} S. 5)}
      \label{img:webanwendungen}
    \end{center}
  \end{figure}
  
  Rich Client steht für kompilierte Desktopapplikationen und Thin Client für
  Webapplikationen welche im Webbrowser laufen, siehe
  \cite{WebApplicationSolutions} S. 4f. Da die Grenze zwischen \ac{RIA} und
  Thin Client nicht klar definiert ist, kommt ab und zu auch der Begriff von
  Thin Client zum Einsatz.
  
  \section{Browser orientiert}
  
  Browser orientierte \ac{RIA} kommen aus der evolution der Internettechnologie
  heraus, da sich deren Grundkonzepte in den letzten Jahren verändert haben. Um
  die Jahrtausenwende wurden klassische Webanwendung nach dem Prinzip von
  Request - Response aufgebaut. Der Benutzer konnte mit einer Interaktion einen
  Statuswechsel von einer Seite zur Nächsten auslösen, zum Beispiel durch das
  anklicken eines Links oder mit dem Versenden eines HTML-Formulars, siehe
  Grafik \ref{img:classicPageReload}. Dabei wurde immer der gesamte
  Seiteninhalt neu geladen, was zu längeren wartezeiten beim Laden der Seite
  und zu einem ungewohnten Anwendergefühl, im Vergleich zu Desktopanwengungen,
  geführt hat. Zudem werden immer alle Daten vom Server an den Browser
  übermittelt, welche für die Darstellung der Webanwendung von Nöten war, siehe
  \cite{AjaxInAction} S. 44ff.
  
  \begin{figure}[hbt]
    \begin{center}
      \includegraphics[width=0.9\textwidth]{./image/classicPageReload.png}
      \caption{Klassische Webanwendung aus der Usersicht (nach
      \cite{DiplomarbeitStephanSchuster} S. 10)}
      \label{img:classicPageReload}
    \end{center}
  \end{figure}
  
  In einem \ac{UML} Sequenzdiagramm sieht das wie folgt aus, siehe Grafik
  \ref{img:sequenzdiagrammClassicPageReload}.
  
  \begin{figure}[hbt]
    \begin{center}
      \includegraphics[width=0.7\textwidth]{./image/sequenzdiagrammClassicPageReload.png}
      \caption{HTTP Request als \ac{UML} Sequenzdiagramm (nach
      \cite{HttpBasics} S. 10)}
      \label{img:sequenzdiagrammClassicPageReload}
    \end{center}
  \end{figure}
  
  %\newpage
  
  Mit dem Konzept von \ac{Ajax}, bei dem der Browser asynchron Daten vom Server
  nachladen kann, wurde die Möglichkeit geschaffen, Anwendungen zu entwickeln,
  welche sich in der Bedienung wie Desktopanwendungen anfühlen, siehe Grafik
  \ref{img:ajaxPageReload}. Das Prinzip funktioniert dadruch, dass eine
  zusätzliche Schicht zwischen dem Browser und Server eingerichtet wird. Diese
  Schicht, ich nenne sie hier Ajax-Engine, übernimmt die Kontrolle über die
  Datenkommunikation zum Server. Die Ajax-Engine bietet die Möglichkeit
  asynchron zur Clientinteraktion Daten vom Server anzufordern und bei Erhalt
  dynamisch in die bestehende Seite einzuflechten. Das Ergebnis ist, dass der
  Browser vom Server entkoppelt wird, wobei der Benutzer die Seite weiterhin
  verwenden kann, auch der Server im Hintergrund getätigte Interaktionen
  verarbeitet.
  
  \begin{figure}[hbt]
    \begin{center}
      \includegraphics[width=0.63\textwidth]{./image/ajaxPageReload.png}
      \caption{Webanwendung mit Ajax aus der Usersicht (nach
      \cite{DiplomarbeitStephanSchuster} S.12)}
      \label{img:ajaxPageReload}
    \end{center}
  \end{figure}
  
  In einem \ac{UML} Sequenzdiagramm sieht das wie folgt aus, siehe Grafik
  \ref{img:sequenzdiagrammAjaxPageReload}.
  
  \begin{figure}[hbt]
    \begin{center}
      \includegraphics[width=0.9\textwidth]{./image/sequenzdiagrammAjaxPageReload.png}
      \caption{Ajax Request als \ac{UML} Sequenzdiagramm}
      \label{img:sequenzdiagrammAjaxPageReload}
    \end{center}
  \end{figure}
  
  \subsection{Security}
  
  Nach \cite{RichInternetApplication} gelten Browser orientierte \ac{RIA},
  welche auf Webstandards\footnote{Webstandards werden durch das Gremium W3C
  definiert, siehe \url{http://www.w3.org/}} basieren, als relativ sicher.
  Mögliche Sicherheitslöcher stellen vorallem der jeweils verwendete Browser,
  in der die Applikation dargestellt wird, und Attacken nach dem Prizip von
  Social Engineering\footnote{Bei Social Engineering geht es darum, durch
  zwischenmenschliche Beeinflussung, unberechtigt an Daten oder Dinge zu
  gelangen, siehe \cite{SocialEngineering}}.
  
  \subsection{Suchmaschinenoptimierung}
  
  Suchmaschinenoptimierung dient dazu im Ranking einer Suchmaschine besser
  abzuschliessen. Dies ist vorallem bei Unternehmen von Bedeutung, welche das
  Internet als Vertriebkanal sehen. Bei statischen Webseiten ist das Indexieren
  ein Prozess der gut funktioniert. Bei einer \ac{RIA}, welche über \ac{Ajax}
  Daten zur Laufzeit einer Webapplikation asynchron nachlädt, wird das für eine
  Suchmaschine ein schwierigeres Unterfangen, die Daten Sinnvoll abzugreifen.
  
  \section{Plugin orientiert}
  
  Plugin orientierte \ac{RIA} sind Applikationen, welche in einem Browser
  Plugin laufen. Dabei sind Adobe Flash, Java und Microsoft Silverlight die drei
  grössten Vertreter in dieser Sparte, siehe \cite{RichInternetApplications}
  und \cite{RichInternetApplicationMarketShare}.

  \section{Client orientiert}
  
  Unter ``Client orientiert'' versteht man lediglich, dass alle
  Desktopanwendungen, welche entweder mit den Internet kommunizieren oder
  zumindes über dieses ausgeliefert werden, zu dieser Kategorie gehören. Dies
  trifft nur zu, in der Annahme, dass Desktopanwendungen eine ``reichhaltige''
  Bedienoberfläche anbieten, siehe \cite{RichInternetApplication}. Eine Java
  Swing Applikation könnte also auch als Client orientierte \ac{RIA}
  klassifiziert werden.


  \cleardoublepage
   
  % ===========================================================================
  % Kapitel XXX beginns here
  % ===========================================================================
   
  \chapter{Analyse der Swing Applikationen}
  
  \section{Auswahl der Applikationen}
  
  \section{Kategorisierung von verwendeten Swing Komponenten}

  \cleardoublepage
   
  % ===========================================================================
  % Kapitel XXX beginns here
  % ===========================================================================
   
  \chapter{Infrastruktur der Zürcher Kantonalbank}
  
  \section{Web Application Stack}
  
  \section{Business Layer}
  
  \section{Frontend}

  \cleardoublepage
  
  % ===========================================================================
  % Kapitel XXX beginns here
  % ===========================================================================
  
  \chapter{Analyse der Web Frameworks}
  
    In der Analyse der Webframeworks sollen fünf Alternativen untersucht werden.
  Es soll eine objektive Methode zur Evaluation dafür gewählt werden. Mit der
  definition von sinnvollen Rahmenbedingungen soll die Evaluation durchgeführt
  und die Resultate vorgelegt werden.
  
  \section{Methoden zur Entscheidungsfindung bei einer Evaluation}
  
  Es gibt verschiedene Methoden wie man bei der Auswahl einer Softwarelösung
  vorgehen kann. Um eine möglichst objektive Betrachtung zu gewährleisten, wurde
  die Methode der gewichteten \ac{NWA} gewählt, siehe \cite{Nutzwertanalyse}.
  Diese Methode stammt aus dem Bereich der quantitativen Analysemethoden der
  Eintscheidungstheorie. Um eine möglichst präzise objektive Gewichtung der
  einzelnen Faktoren zu erhalten wurde die Methode \ac{AHP} gewählt, siehe
  \cite{AnalyticHierarchyProcess}. Diese Methode stammt aus dem Bereich der
  präskriptiven Entscheidungstheorie. Der \ac{AHP} wurde von Thomas L. Saaty in
  den 70er Jahren des 20. Jahrhunderts entwickelt und im Buch ``The Analytic
  Hierarchy Process: Planning, Priority Setting, Resource
  Allocation''\cite{AnalyticHierarchyProcessBook} veröffentlicht.
  
  \subsection{Gewichtete Nutzwertanalyse}
  
  Bei der gewichteten \ac{NWA} wird eine Menge von Kandidaten, auf deren
  Nutzen, miteinander verglichen. Der Vergleich wird über \(n\) vergleichbare
  Alternativen geführt. Dabei werden die einzelnen Alternativen mit einem
  Erfüllungsgrad \(e_i\) bewertet. Die Skala der Erfüllungsgrade ist in der
  Tabelle \ref{tab:erfuellungsgrade} ersichtlich.
  \newline
  
  \begin{table}[h]
    \begin{center}
      \begin{tabular}{lc}
        \toprule
        Erfüllungsgrad & Skala\\
        \midrule
        nicht erfüllt & 0\\
        schlecht & 1, 2\\
        mittel & 3 - 5\\
        gut & 6 - 8\\
        sehr gut & 9\\
        \bottomrule
      \end{tabular}
      \caption{Skala der Erfüllungsgrade}
      \label{tab:erfuellungsgrade}
    \end{center}
  \end{table}
  
  Jede Alternative wird durch einen Gewichtungsfaktor \(g_i\) versehen, was
  die Präferenz der Alternative wiederspiegelt. Dabei gilt: Die Gewichte gi werden
  so gewählt, dass ihre Summe 1 (100\%) ergibt, siehe Formel \ref{eq:gewicht}.

  \begin{equation}
    \label{eq:gewicht}
    \(Gewicht:= \sum \limits_{i=1}^n g_i = 1\)  
  \end{equation}

  \newpage

  Der Nutzwert ergibt sich durch die Formel \ref{eq:nutzwert}:

  \begin{equation}
    \label{eq:nutzwert}
    \(Nutzwert:= \sum \limits_{i=1}^n e_i \cdot g_i\)  
  \end{equation}
  
  Für jeden zu evaluierenden Kandidaten soll geprüft werden, ob eines der
  KO-Kriterien erfüllt ist, was zu einem Ausschluss des Kandidaten führen würde.
  Falls das nicht der Fall ist, wird der Nutzwert des Kandidaten berechnet.
  Derjenige Kandidat mit dem grössten Nutzwert entspricht am meisten den
  Anforderungen.
  
  \subsubsection{Anschauliches Beispiel}
  
  Als anschauliches Beispiel sollen zwei Kandidaten - Auto \(A\) und \(B\) -
  miteinander verglichen werden. Sie sollen auf deren Alternativen - Leistung,
  Aussehen und Alltagstauglichkeit - verglichen werden. In der Tabelle
  \ref{tab:beispielNwa} ist das Beispiel ersichtlich. Das Resultat zeigt, dass
  das Auto \(A\) dem Auto \(B\)  gegenüber bevorzugt werden soll, da der
  Nutzwert \(5.0 > 4.7\) ist. 
  
  \begin{table}[h]
    \begin{center}
      \begin{tabular}{lrrrrr}
        \toprule
        Alternativen & Gewichtung \(g\) & \(e_A\) & Wertigkeit \(A\) & \(e_B\)
        & Wertigkeit \(B\)\\
        \midrule
        Leistung            & 0.3 & 7 & 2.1 & 9 & 2.7 \\
        Aussehen            & 0.2 & 2 & 0.4 & 5 & 1.0 \\
        Alltagstauglichkeit & 0.5 & 5 & 2.5 & 2 & 1.0 \\
        \midrule
        \midrule
        Ergebnis            & 1.0 &   & 5.0 &   & 4.7 \\
        \bottomrule
      \end{tabular}
      \caption{Beispiel einer Nutzwertanalyse}
      \label{tab:beispielNwa}
    \end{center}
  \end{table}
 
  \subsection{Analytic Hierarchy Process}
  
  Der Analytic Hierarchy Process stamt aus der Feder eines Mathematikers. Aus
  diesem Grund ist das Verfahren auch einiges Anspruchsvoller als eine
  gewichtete Nutzwertanalyse. Ich gehe hier nicht auf die ganzen Details ein, da
  es den Rahmen der Diplomarbeit übersteigen würde. Totzdem soll eine grober
  Überblick über den \ac{AHP} gegeben werden.
  
  Der \ac{AHP} besteht aus drei Phasen, siehe \cite{AnalyticHierarchyProcess}: 
  
  \begin{enumerate}
    \item Sammeln der Daten
    \item Daten vergleichen und gewichten
    \item Daten verarbeiten
  \end{enumerate}
  
  \subsubsection{1. Sammeln der Daten}
  
  In der ersten Phase sollen alle Daten, die für eine Entscheidungsfindung
  erheblich sind, gesammelt werden.
  
  \begin{itemize}
    \item Zuerst soll eine konkrete Frage formuliert werden, für welche die
    beste Antwort gesucht wird.
    \item Danach sollen zu der gestellten Frage alle Kriterien  gesucht werden,
    welche die Lösung beeinflussen können.
    \item Als letztes sollen alle Alternativen gesucht werden, welche als
    mögliche Lösung infrage kommen.
  \end{itemize}
  
  \subsubsection{2. Daten vergleichen und gewichten}
  
  In der zweiten Phase folgt nun die Gegenüberstellung, Vergleich und Bewertung
  aller Kriterien beziehungsweise Alternativen in zwei Unterschritten.
  
  \begin{itemize}
    \item Jedes Kriterium wird jedem anderen gegenübergestellt und darauf
    verlgichen, was eine grössere Bedeutung in der gestellten Frage hat. Die
    Skala geht von 1 bis 9, siehe Tabelle \ref{tab:vergleichsgrade}.
    \item Für jedes Kriterium wird jede mögliche Alternativen mit jeder anderen
    gegenübergestellt und auf ihre Eignung hin untersuchen, welche Alternative
    am besten zur Erfüllung des jeweiligen Kriteriums passt. Die Skala geht von
    1 bis 9, siehe Tabelle \ref{tab:vergleichsgrade}.
  \end{itemize}
  
  \begin{table}[h]
    \begin{center}
      \begin{tabular}{lc}
        \toprule
        Bedeutung & Skala\\
        \midrule
        gleiche Bedeutung & 1\\
        leicht grössere Bedeutung & 2 - 3\\
        viel grössere Bedeutung & 4 - 6\\
        erheblich grössere Bedeutung & 7 - 8\\
        absolut dominierend & 9\\
        \bottomrule
      \end{tabular}
      \caption{Skala der Vergleichsgrade}
      \label{tab:vergleichsgrade}
    \end{center}
  \end{table}
    
  \subsubsection{3. Daten verarbeiten}
  
  Mit einem mathematischen Modell, kann der \ac{AHP} nun eine präzise Gewichtung
  aller Kriterien errechnen. Mit der Gewichtung der Kriterien und dem Vergleich
  der Alternativen, kann der \ac{AHP} nun berechnen, welches die beste Lösung
  (Alternative) für die gestellt Frage ist.
  Diese Berechnungen werden meistens mit der Unterstütztung einer Software
  gemacht. Als Beispiel gibt es JAHP 2.1, was ein Java Programm ist, mit 
  welchem der gesammte Prozess des \ac{AHP} abgebildet und berechnet werden
  kann. Das Programm wird unter den Bedingungen der GNU General Public License
  vertrieben.
    
  \subsection{Kombination beider Methoden}
  
  Diese beiden Methoden können auch kombiniert eingesetzt
  werden\cite{AhpNwaKombination}, da der \ac{AHP} relativ komplex in der
  Umsetztung ist. Als Kombination kann die Nutzwertanalyse als Methode
  verwendet werden, und der \ac{AHP} wird für die präzise berechnung der
  Gewichtung einzelner Entscheidungskriterien verwendet. Aus der Kombination
  entsteht somit eine neue Methode, welche durch die Verständlichkeit der
  \ac{NWA} und der objektiven Gewichtung des \ac{AHP} eine plausable Evaluation
  ermöglicht.
    
  \section{Rahmenbedingungen zur Auswahl der Web Frameworks}
  
  Die Auswahl der zu verwendenden Web Frameworks wird durch eine Menge von
  Soll-Kriterien bestimmt. Diese Menge bestimmt die Anforderungen, welche
  erfüllt werden sollen. Die Anforderungen stammen von einer Projektgruppe der
  Hochschule für Technik und Wirtschaft Berlin.
  
  Anhand einer Menge von KO-Kriterien wird die Auswahl eingeschränkt. Es werden
  dabei die Vorgaben aus der IT-Architektur der Zürcher Kantonalbank verwendet,
  welche klare Regeln definiert.

  \subsection{Soll-Kriterien}
  
  Soll-Kriterien sind Vorgaben, die möglichst weitgehend Erfüllt werden sollen.
  Wenn ein solches Kriterium nicht erfüllt werden kann, schliesst das die
  Alternative nicht aus. Jedem Soll-Kriterium wird für die Identifikation eine
  eindeutige ID vergeben. Die ID setzt sich folgendermassen zusammen.
  \{Soll\}-\{Laufnummer\}.
  
  \subsubsection{Anforderungen an Webframeworks nach AgileLearn}
  
  Eine Projektgruppe namens AgileLearn von der Hochschule für Technik und
  Wirtschaft Berlin befasst sich mit dem Thema Web Frameworks. Die
  Projektgruppe hat sich die Frage gestellt: ``\begin{itshape}Welche
  Anforderungen müssen bei der Wahl eine Webframeworks berücksichtigt
  werden?\end{itshape}'' Aus der Fragestellung heraus, haben sie 18
  Anforderungen an ein Web Framework ausgearbeitet, welche öffentlich in einem
  Google-Doc\footnote{Ein Service von Google um Dokumente öffentlich zu
  bearbeiten.} ersichtlich sind.
  
  Folgende Anforderungen stammen aus dem Dokument ``\begin{itshape}18
  Anforderungen an Webframeworks -
  OpenDoc\end{itshape}''\cite{AnforderungenAnWebframeworks} und werden hier
  zusammengefasst. Da auf zwei Anforderungen verzichtet wurde, sind 16
  Anforderungen übrig geblieben, welche in den Katalog der Soll-Kriterien
  aufgenommen werden. Es wurde auf folgende Punkte als Anforderung an ein
  Webframework verzichtet:
  \newline
  
  ``MVC-Entwurfsmuster''
  \newline
  \newline
  \noindent
  Das MVC-Konzept ist sicher gut, sollte aber nicht eine Anforderung an ein Web
  Framework sein, da es viele ebenbürdige Alternativen gibt.
  \newline
  
  ``Object Relational Mapping (ORM)''
  \newline
  \newline
  \noindent
  In der Java Welt gibt es viele ORM Lösungen, welche sich in den Jahren
  durchgesetzt haben, ein paar nennenswerte sind Hibernate, TopLink und
  EclipseLink. Diese Anforderung entfällt also.
  \newline
  
  Folgende Punkte wurden in den Katalog der Soll-Kriterien aufgenommen und somit
  mit einer eindeutigen ID versehen:

  \begin{description}

    % Kapitel 2.2.2, N-Tier Applikationen, Seite 20
    \item[Soll-01 - Zugriffskontrolle
    (Authentifizierung/Authorisation/Rollenverwaltung)\label{itm:Soll-01}]

    Ein Webframework sollte EntwicklerInnen verschiedene Mechanismen
    bereitstellen, um die Anwendung vor fremden und unerlaubten Zugriff schützen
    zu können.

    Authentifizierung/Autorisierung: Üblicherweise werden im Vorfeld Rollen für
    verschiedenen Gruppen festgelegt. Das Ziel einer sicheren Webanwendung ist
    es, bestimmte Bereiche einer Seite abzusichern und die Rechte aller
    Benutzer je nach Rolle einzuschränken. Anhand der Rolle wird deren
    Benutzer für die festgelegten Bereiche autorisiert.

    Vertraulichkeit / Verschlüsselung: Sensible Daten, wie Passwörter und
    Personendaten, müssen vor dem Zugriff und der Kenntnisnahme von Dritten
    geschützt werden. Hierfür werden die Daten während der Übertragung
    verschlüsselt. Für eine sichere Datenübertragung werden meist
    Verschlüsselungsprotokolle wie SSL (Secure Sockets Layer), sowie dessen
    Nachfolger TLS (Transport Layer Security) eingesetzt. Diese gelten als
    relativ sicher und sind bei Transaktionen bei einer Bank unverzichtbar.

    \item[Soll-02 - Form-Validierung\label{itm:Soll-02}]
    Das Verarbeiten von Formularen bzw. die Handhabung von Benutzereingaben und
    -aktionen gehört zu den täglichen Aufgaben der Webentwicklung. Die Logik
    für server- und clientseitiges Validieren ist im Idealfall nur einmal
    implementiert.

    Server-Side Validation: Das Webframework soll die Mögichkeit bieten,
    eingegebene Daten einfach zu überprüfen. Dabei soll für jede Eigenschaft
    eines Datenmodells (Model) ein Wertebereich definierbar sein, zusätzlich
    soll geprüft werden, ob die Eingabe erforderlich ist. Die Programmierlogik
    soll minimal sein. Nach dem Senden der Daten wird alles überprüft und ggf.
    entsprechende Fehlermeldungen zurückgegeben.

    Client-Side Validation: Das Webframework soll wenn möglich schon auf dem
    Client validieren. Wenn möglich, sollen die Daten gleich bei oder kurz nach
    der Eingabe überprüft werden, wobei die Logik auf dem Server implementiert
    ist. Beispiel: Ist der Anmeldename schon vergeben. Dadurch werden
    Serverressourcen gespart und der Benutzer hat ein direktes Feedback.

    \item[Soll-03 - Modulare Architektur\label{itm:Soll-03}]

    Eine Webanwendung stellt ein Zusammenspiel verschiedenster
    Internettechnologien dar, die wiederum hinsichtlich ihrer Entwicklung einem
    stetigen Wandel unterliegen. Die Herausforderung für die Entwickler ist es,
    die Entwicklung der Technologien im Auge zu behalten um gegebenenfalls
    Neuheiten oder Änderungen im System anzupassen. Eine Webanwendung sollte
    daher in all ihren Bestandteilen möglichst wartbar bleiben.

    \item[Soll-04 - Schnittstellen und Webservices\label{itm:Soll-04}]

    Interoperabilität beschreibt den Austausch von Informationen verschiedener
    Softwaresysteme. Es gibt verschiedene Technologien, mit denen ein
    Informationsaustausch umgesetzt werden kann - REST, SOAP und RPC sind am
    weitesten verbreitet. Das Webframework sollte Mechanismen und Funktionen
    zur Umsetzung von Schnittstellen bereitstellen.

    \item[Soll-05 - Testing\label{itm:Soll-05}]

    Testing ist mit test-driven development (TDD), vor allem in der agilen
    Softwareentwicklung, ein fester Bestandteil während der Projektentwicklung.
    Vor der Implementierung überprüft der Programmierer, mittels Unit-Tests, 
    konsequent das Verhalten jeglicher Komponenten. Gerade kritische Prozesse
    und Transaktionen (zum Beispiel eine Banküberweisung) sollten ausgiebig
    getestet werden. Das Webframework soll die Möglichkeit von Unit-Tests
    bieten.

    \item[Soll-06 - Internationalisierung und Lokalisierung\label{itm:Soll-06}]

    Viele Webanwendungen richten sich mittlerweile an ein internationales
    Publikum. Dank der Offenheit und der weiten Verbreitung des Internets lassen
    sich dadurch sehr einfach neue Zielgruppen (BenutzerInnen) erschließen.
    Jedoch gilt es, einige Vorraussetzungen und Besonderheiten bei der
    Internationalisierung von Webanwendungen zu beachten. Zunächst müssen
    sämtliche Texte übersetzt und möglicherweise in neue Datenbanken ausgelagert
    werden. Hierbei ist zu berücksichtigen, dass es länderspezifische
    Zeichensätze, Zahlen, Datum und Währungswerte gibt. Hinzu kommt, dass unter
    Umständen auch spezielle Grafiken neu erstellt werden müssen.

    \item[Soll-07 - Scaffolding / Rapid Prototyping\label{itm:Soll-07}]

    Mit Scaffolding ist das automatische Generien von den sogenannten CRUD-Pages
    (Create, Read, Update, Delete) gemeint.

    Scaffolding und das dadurch verstandene Rapid Prototyping ist ein
    wesentlicher Aspekt der agilen Softwareentwicklung (zum Beispiel in
    Verwendung mit der Scrum-Methodik). Gerade zu Beginn eines Projekts eignet
    sich das Rapid Prototyping, da Änderungen in den Models umgehend in die
    Views eingebunden werden können.

    \item[Soll-08 - Caching\label{itm:Soll-08}]

    Neben der Übertragunsgeschwindigkeit gibt es eine Menge anderer Faktoren,
    die für eine leistungsstarke Webanwendung entscheidend sind. Ein Aspekt ist
    das Caching - das Zwischenspeichern von Daten, die häufig verwendet bzw.
    aufgerufen werden.

    Das Caching kann meist auf verschiedenen Ebenen implementiert werden: Auf
    dem Client, auf dem Webserver und auf der Datenbankebene. Dabei soll das
    Webframework diese Mechanismen unterstützen und es ermöglichen, diese
    einfach zu aktivieren und zu kalibrieren.  

    \item[Soll-09 - View-Engine\label{itm:Soll-09}]

    View Engines werden eingesetzt, um das Arbeiten mit den Views zu
    erleichtern. Sie unterstützen vor allem das Templating - das Erstellen von
    Vorlagen. Durch typisierte Views wird eine Webanwendung robuster, weil
    weniger fehleranfällig; durch partial Views (oft auch nur als partials
    bezeichnet) werden einzelne Elemente oder Bereiche der Benutzeroberfläche
    wiederverwendbar gemacht. Diese Vorlagen können von mehreren Views genutzt
    werden. Dadurch wird deutlich weniger redundanter Code erstellt. Dies ist
    ein wichtiger Aspekt in Bezug auf das Don’t repeat yourself (DRY) Prinzip.

    \item[Soll-10 - Dokumentation\label{itm:Soll-10}]

    Eine Webanwendung ist aus Entwicklersicht eine Zusammenfassung
    verschiedenster Technologien (Webframeworks, Bibliotheken, Schnittstellen).
    Über das Application Programming Interface (API) haben Programmierer Zugriff
    auf die verfügbaren Funktionen. Nicht selten erstreckt sich die
    Dokumentation einer API über mehrere hundert Seiten. Für Programmierer ist
    es daher von enormer Bedeutung, dass die Bibliotheken gut strukturiert und
    verständlich beschrieben sind. Schlecht oder gar nicht dokumentierte
    Technologien erhöhen die Fehlerquote und sind oft ausschlaggebend für einen
    nicht flüssigen Workflow.

    \item[Soll-11 - Community\label{itm:Soll-11}]

    Die beteiligten Personen in den Foren, Mailing-Listen oder Wikis bilden im
    Zusammenhang mit Webframeworks die Community. Es findet dabei ein
    Wissensaustausch statt, der weit über die standard-Dokumentation hinaus
    geht. Die Nutzer helfen sich gegenseitig bei Problemen und Fehlern und oft
    hinterlassen sie mit ihren Einträgen wiederum einen Lösungsansatz, der
    zukünftig von anderen wieder aufgegriffen werden kann. Es ist daher 
    wichtig, dass den Anwendern eine Möglichkeit geboten wird, sich
    auszutauschen, gemeinsam Fehlermeldungen zu deuten und Lösungsansätze zu
    entwickeln.

    \item[Soll-12 - IDE-Unterstützung\label{itm:Soll-12}]

    In der Softwareentwicklung ist die IDE das Basiswerkzeug für die
    Programmierer. Die Entwicklungsumgebung stellt den Entwicklern verschiedene
    Komponenten zur Verfügung, wie Editor, Compiler, Linker oder Debugger.
    Hinzu kommen mit Syntaxhighlighting, Refactoring und Code-Formatierung
    weitere wichtige Funktionen, die die Entwickler in vielerlei Hinsicht enorm
    unterstützen.

    \item[Soll-13 - Kosten für Entwicklungswerkzeuge\label{itm:Soll-13}]

    Je nach verfügbaren finanziellen Mitteln spielen die Kosten von
    Entwicklungswerkzeugen und Technologien durchaus eine Rolle. Bei beiden
    Faktoren hat man meist die Wahl zwischen kostenlosen (OpenSource) und
    kommerziellen Produkten. Je nach Webanwendung müssen somit Lizenzgebühren
    für die Entwicklungswerkzeuge, Server zum Ausführen der Anwendung und
    Datenbankserver berücksichtigt werden. Dabei ist es wichtig die richtige
    Mischung verschiedener Komponenten zu finden.
  
    \item[Soll-14 - Eignung für agile Entwicklung\label{itm:Soll-14}]

    Die herkömmlichen Methoden der Software-Entwicklung werden heute oft durch
    neue agile Methoden, wie Extreme Programming oder Scrum abgelöst. Sie
    fokussieren auf das Wesentliche und stehen für deutlich mehr Flexibilität in
    der Entwicklungsphase als konventionelle Methoden. Verschiedene Technologien
    unterstützen die agilen Methoden. Refactoring, Testing spielt dabei eine
    wichtige Rolle und soll unterstützt werden.

    \item[Soll-15 - Lernkurve für EntwicklerInnen\label{itm:Soll-15}]

    Zur Umsetzung einer komplexen Webanwendung wird den Entwicklern ein Wissen
    über verschiedenste Bereiche der Softwareentwicklung abverlangt.
    Glücklicherweise gibt es für die Webentwicklung keinen einheitlichen
    Standard, der festlegt, wie eine Webanwendung entwickelt werden muss. Das
    Internet bietet für jeden Bereich eine Auswahl unterschiedlicher
    Technologien an. Daher müssen vor der Entwicklung mehrere Entscheidungen
    getroffen werden - hinsichtlich Programmiersprache, Javascript-Framework
    oder Datenbankserver. Es werden daher oft Technologien gewählt, die leicht
    zu erlernen und verstehen sind.

    Wichtig ist, dass man einen schnellen Einstieg bekommt und Erfolge bald
    sichtbar werden, um die Motivation der Entwickler zu erhöhen.

    \item[Soll-16 - AJAX-Unterstützung\label{itm:Soll-16}]

    Durch das Aufkommen von Javascript-Bibliotheken wie jQuery und Prototype
    haben sich vollkommen neue Möglichkeiten eröffnet mit Javascript auf dem
    Client zu arbeiten und mit AJAX wurde die Kommunikation zwischen Server und
    Client im Web revolutioniert. Die direkte Unterstützung eines
    Javascript-Frameworks ist für ein Webframework sinnvoll und erwünscht.

    Darüber hinaus sollte die unterstützte Bibliothek lose gekoppelt sein, und
    damit austauschbar. Sogenanntes unobtrusive Javascript, bei dem auch bei
    abgeschaltetem Javascript die Anwendung funktioniert, ist auch eine Anforderung.
  \end{description}    
    
  \subsection{KO-Kriterien}
  
  KO-Kriterien sind Vorgaben, welche zwingend erfüllt sein müssen. Falls ein
  Kriterium nicht erfüllt werden kann, fällt die Entscheidung auf diese
  Alternative negativ aus. Jedem KO-Kriterium wird für die Identifikation eine
  eindeutige ID vergeben. Die ID setzt sich folgendermassen zusammen.
  \{KO\}-\{Laufnummer\}.
  
  \subsubsection{Grundsätze aus der IT-Architektur der Zürcher Kantonalbank}
  
  Folgende KO-Kriterien sind aus dem \begin{itshape}Handbuch der
  IT-Architektur\end{itshape}\cite{ZkbHandbuchDerItArchitektur} der Züricher
  Kantonalbank entnommen. Die Namensgebung unterscheidet sich leicht, es wird
  nicht von KO-Kriterien, sondern von Grundsätzen gesprochen. Ein Grundsatz
  wird wie folgt definiert:\\
  
  ``\begin{itshape}Es sind Grundsätze definiert, nach denen sich die Baupläne
  der IT-Systeme zu richten haben. Die Grundsätze sind ein Regelwerk mit
  Weisungscharakter.\end{itshape}''
  \footnote{\cite{ZkbHandbuchDerItArchitektur} Kapitel 1.3 - \begin{itshape}Was
  ist die IT-Architektur der ZKB\end{itshape}, Seite 11}
  \\
  \\
  \noindent
  Dabei gibt es eine Hintertür:\\

  ``\begin{itshape}Grundsätze sind verbindliche Vorgaben (Konventionen), von
  denen nur in begründeten Ausnahmen abgewichen werden kann.\end{itshape}''
  \footnote{\cite{ZkbHandbuchDerItArchitektur} Kapitel 1.8 -
  \begin{itshape}Leseanleitung\end{itshape}, Seite 14}
  \\
  \\
  \noindent
  Das Dokument wurde analysiert und die Grundsätze, welche für diese
  Diplomarbeit relevanten sind, werden hier aufgelistet:
  
  \begin{description}

    % Kapitel 2.2.2, N-Tier Applikationen, Seite 20
    \item[KO-01\label{itm:KO-01}]
    \footnote{\cite{ZkbHandbuchDerItArchitektur} Kapitel 2.2.2 -
    \begin{itshape}N-Tier Applikationen\end{itshape}, Seite 20}
    Applikationen sollen als N-Tier Applikationen designed und
    implementiert werden.

    % Kapitel 2.2.10, Objektorientierung, Seite 22
    \item[KO-02\label{itm:KO-02}]
    \footnote{\cite{ZkbHandbuchDerItArchitektur} Kapitel 2.2.10 -
    \begin{itshape}Objektorientierung\end{itshape}, Seite 22}
    \ac{OO} soll innerhalb der Informatik für Neuentwicklungen
    durchgängig angewandt werden.

    % Kapitel 3.3, Mehrsprachigkeit, Seite 36
    \item[KO-03\label{itm:KO-03}]
    \footnote{\cite{ZkbHandbuchDerItArchitektur} Kapitel 3.3 -
    \begin{itshape}Mehrsprachigkeit\end{itshape}, Seite 36}
    Eine neue Applikation (oder eine neue Komponente einer
    bestehenden Applikation) ist mehrsprachfähig zu realisieren.

    % Kapitel 3.3, Mehrsprachigkeit, Seite 37
    \item[KO-04\label{itm:KO-04}]
    \footnote{\cite{ZkbHandbuchDerItArchitektur} Kapitel 3.3 -
    \begin{itshape}Mehrsprachigkeit\end{itshape}, Seite 37}
    Neue Applikationen sind Unicode-fähig zu realisieren.

    % Kapitel 3.9.1, Skalierbarkeit / Ausfallsicherheit / Perfomance
    % Seite 49
    \item[KO-05\label{itm:KO-05}]
    \footnote{\cite{ZkbHandbuchDerItArchitektur} Kapitel 3.9.1 -
    \begin{itshape}Skalierbarkeit / Ausfallsicherheit / Perfomance\end{itshape},
    Seite 49}
    Eine Applikation muss in mehreren Instanzen lauffähig sein.

    % Kapitel 4.4.1, User Interface, Seite 55
    \item[KO-06\label{itm:KO-06}]
    \footnote{\cite{ZkbHandbuchDerItArchitektur} Kapitel 4.4.1 -
    \begin{itshape}User Interface\end{itshape}, Seite 55}
    Der ZKB GUI Style Guide ist in allen ZKB IT-Projekten anzuwenden.
    
    % Kapitel 5.2, Verwendung der Zentralen Server Infrastruktur ZSI, Seite 57
    \item[KO-07\label{itm:KO-07}]
    \footnote{\cite{ZkbHandbuchDerItArchitektur} Kapitel 5.2 -
    \begin{itshape}Verwendung der Zentralen Server Infrastruktur
    ZSI\end{itshape}, Seite 57}
    Die \ac{ZSI} ist als Server-Plattform für Applikationen, welche Windows-,
    Unix-, Linux-basierte Server einsetzen, zu verwenden.
    
    % Kapitel 9.2, Java RMI, Seite 71
    \item[KO-08\label{itm:KO-08}]
    \footnote{\cite{ZkbHandbuchDerItArchitektur} Kapitel 9.2 -
    \begin{itshape}Java RMI\end{itshape}, Seite 71}
    RMI kann für reine Java-Anwendungen eingesetzt werden.
    
    % Kapitel 12.2.1, Einsatz von Frameworks, Seite 140
    \item[KO-09\label{itm:KO-09}]
    \footnote{\cite{ZkbHandbuchDerItArchitektur} Kapitel 12.2.1 -
    \begin{itshape}Einsatz von Frameworks\end{itshape}, Seite 140}
    Für Java-Applikationen (Internet, Extranet und Intranet) wird das
    ZIP-Framework eingesetzt.
    
    % Kapitel 12.3.5, Client-/Server-Schemata von Internet-Applikationen,
    % Seite 141
    \item[KO-10\label{itm:KO-10}]
    \footnote{\cite{ZkbHandbuchDerItArchitektur} Kapitel 12.3.5 -
    \begin{itshape}Client-/Server-Schemata von Internet-Applikationen\end{itshape}, Seite 141}
    Die Validierung und Plausibilisierung der Eingaben erfolgt immer
    abschliessend auf dem bankseitigen Applikations-Server. Es ist aber
    durchaus möglich, dass sich auf der Client-Seite eine Logik zur Überprüfung
    und Validierung der Eingaben für den Benutzerkomfort befindet.
    
    % Kapitel 12.3.5, Client-/Server-Schemata von Internet-Applikationen,
    % Seite 141
    \item[KO-11\label{itm:KO-11}]
    \footnote{\cite{ZkbHandbuchDerItArchitektur} Kapitel 12.3.5 -
    \begin{itshape}Client-/Server-Schemata von Internet-Applikationen\end{itshape}, Seite 141}
    Die Business-Logik in einer Internet-Applikation ist so auszulegen, dass
    diese von verschiedenen Präsentations-Logiken im Rahmen von Ultra-Thin- und
    Thin-Client-Applikationen genutzt werden kann.
    
    % Kapitel 12.3.6.1, Browser-Abhängigkeiten, Seite 142
    \item[KO-12\label{itm:KO-12}]
    \footnote{\cite{ZkbHandbuchDerItArchitektur} Kapitel 12.3.6.1 -
    \begin{itshape}Browser-Abhängigkeiten\end{itshape}, Seite 142}
    Die Internet-Applikationen der ZKB werden nicht mit Browser-Abhängigkeiten
    versehen und orientieren sich an den neutralen Standards der W3C-Komission.
    
    % Kapitel 12.3.6.2, Session-Mechanismus, Seite 142
    \item[KO-13\label{itm:KO-13}]
    \footnote{\cite{ZkbHandbuchDerItArchitektur} Kapitel 12.3.6.2 -
    \begin{itshape}Session-Mechanismus\end{itshape}, Seite 142}
    Für Ultra-Thin-Client-Applikationen wird als Session-Mechanismus die
    Cookie- oder die URL-Rewriting-Methode angewendet.
    
    % Kapitel 12.3.6.4, Einfache Internet-Applikationen, Seite 143
    \item[KO-14\label{itm:KO-14}]
    \footnote{\cite{ZkbHandbuchDerItArchitektur} Kapitel 12.3.6.4 -
    \begin{itshape}Einfache Internet-Applikationen\end{itshape}, Seite 143}
    Einfache Internet-Applikationen mit dem Schwerpunkt Information können
    ausschliesslich Java Server Pages verwenden.
    
    % Kapitel 12.3.6.5, Komplexe Internet-Applikationen, Seite 143
    \item[KO-15\label{itm:KO-15}]
    \footnote{\cite{ZkbHandbuchDerItArchitektur} Kapitel 12.3.6.5 -
    \begin{itshape}Komplexe Internet-Applikationen\end{itshape}, Seite 143}
    Komplexe Internet-Applikationen verwenden eine Kombination von Java Server
    Pages und mindestens einem Servlet als Dispatcher-Mechanismus.
    
    % Kapitel 12.3.6.6, Verwendung von JavaScript beziehungsweise ECMAScript
    % Seite 143
    \item[KO-16\label{itm:KO-16}]
    \footnote{\cite{ZkbHandbuchDerItArchitektur} Kapitel 12.3.6.6 -
    \begin{itshape}Verwendung von JavaScript beziehungsweise
    ECMAScript\end{itshape}, Seite 143}
    Die Internet-Applikationen funktionieren auch eingeschränkt, ohne dass die
    Skript-Funktion im Browser aktiviert ist.
    
    % Kapitel 12.3.6.7, Einstz von ActiveX und Cookies, Seite 143
    \item[KO-17\label{itm:KO-17}]
    \footnote{\cite{ZkbHandbuchDerItArchitektur} Kapitel 12.3.6.6 -
    \begin{itshape}Verwendung von JavaScript beziehungsweise
    ECMAScript\end{itshape}, Seite 143}
    ActiveX wird wegen der Möglichkeit für direkte Zugriffe auf das
    Betriebssystem nicht eingesetzt.
    
    % Kapitel 12.3.6.8, Applets, Seite 144
    \item[KO-18\label{itm:KO-18}]
    \footnote{\cite{ZkbHandbuchDerItArchitektur} Kapitel 12.3.6.8 -
    \begin{itshape}Applets\end{itshape}, Seite 144}
    Es werden keine neuen Applikationen als Java Applets entwickelt. Der
    Einsatz von Applets beschränkt sich auf einfache Funktionen wie Börsen-
    oder News-Ticker.
    
    % Kapitel 12.3.6.9, Browser-Plugins, Seite 144
    \item[KO-19\label{itm:KO-19}]
    \footnote{\cite{ZkbHandbuchDerItArchitektur} Kapitel 12.3.6.9 -
    \begin{itshape}Browser-Plugins\end{itshape}, Seite 144}
    Internet-Applikation werden ohne Plugins entwickelt.
    
    % Kapitel 12.3.7, Client-Technologien, Seite 144
    \item[KO-20\label{itm:KO-20}]
    \footnote{\cite{ZkbHandbuchDerItArchitektur} Kapitel 12.3.7 -
    \begin{itshape}Client-Technologien\end{itshape}, Seite 144}
    Für Ultra Thin Clients bzw. Browser-basierende Applikationen muss das
    aktuelle, Struts-basierende HTML-Client-Framework der ZKB Internet
    Plattform verwendet werden.
    
    % Kapitel 12.3.8, Layering von Internet-Applikationen, Seite 145
    \item[KO-21\label{itm:KO-21}]
    \footnote{\cite{ZkbHandbuchDerItArchitektur} Kapitel 12.3.8 -
    \begin{itshape}Layering von Internet-Applikationen\end{itshape}, Seite 145}
    Neue Internet- und Extranet-Applikationen müssen sich an das Layering
    gemäss nachfolgender Grafik @@@@@@@@@ HIER NOCH DIE GRAFIK REFERENZIEREN
    @@@@@@@ halten. Für Intranet-Applikationen ist die Validator-Komponente
    fakultativ.
    
    % Kapitel 12.3.8, Layering von Internet-Applikationen, Seite 146
    \item[KO-22\label{itm:KO-22}]
    \footnote{\cite{ZkbHandbuchDerItArchitektur} Kapitel 12.3.8 -
    \begin{itshape}Layering von Internet-Applikationen\end{itshape}, Seite 146}
    Eine Applikation muss ohne Änderung von der Intranet-Anwendung zur Extra-
    oder Internet-Applikation gemacht werden können. Es geschieht dies
    lediglich durch das Vorschalten der Validator-Komponente.
    
    % Kapitel 12.3.10.2, Web Server, Seite 146
    \item[KO-23\label{itm:KO-23}]
    \footnote{\cite{ZkbHandbuchDerItArchitektur} Kapitel 12.3.10.2 -
    \begin{itshape}Web Server\end{itshape}, Seite 146}
    Der ZKB Standard-Web-Server ist der Apache HTTP-Server.
    
    % Kapitel 12.4.2, Architektur beim Einsatz von Application-Servern,
    % Seite 147
    \item[KO-24\label{itm:KO-24}]
    \footnote{\cite{ZkbHandbuchDerItArchitektur} Kapitel 12.4.2 -
    \begin{itshape}Architektur beim Einsatz von
    Application-Servern\end{itshape}, Seite 147}
    Der Application-Server wird als die integrierte technische Middleware für
    die Unterstützung von Java Server Pages (JSP), Servlets, Enterprise Java
    Beans (EJB) und der sicheren Kommunikation zwischen Client und Server
    eingesetzt.
    
    % Kapitel 12.4.2, Architektur beim Einsatz von Application-Servern,
    % Seite 148
    \item[KO-25\label{itm:KO-25}]
    \footnote{\cite{ZkbHandbuchDerItArchitektur} Kapitel 12.4.2 -
    \begin{itshape}Architektur beim Einsatz von
    Application-Servern\end{itshape}, Seite 148}
    Der ZKB Standard-J2EE-Application-Server ist der JBoss Application Server.
    
    % Kapitel 12.4.2, Architektur beim Einsatz von Application-Servern,
    % Seite 148
    \item[KO-26\label{itm:KO-26}]
    \footnote{\cite{ZkbHandbuchDerItArchitektur} Kapitel 12.4.2 -
    \begin{itshape}Architektur beim Einsatz von
    Application-Servern\end{itshape}, Seite 148}
    Der J2EE-Application-Server wird in der \ac{ZSI} eingesetzt.
    
    % Kapitel 12.4.2, Architektur beim Einsatz von Application-Servern,
    % Seite 148
    \item[KO-27\label{itm:KO-27}]
    \footnote{\cite{ZkbHandbuchDerItArchitektur} Kapitel 12.4.2 -
    \begin{itshape}Architektur beim Einsatz von
    Application-Servern\end{itshape}, Seite 148}
    Die Serverplattform für den Einsatz von Web-Application-Servern für
    Internet-/Intranet-/Extranet-Applikationen ist Linux.
    
    % Kapitel 12.4.2, Architektur beim Einsatz von Application-Servern,
    % Seite 150
    \item[KO-28\label{itm:KO-28}]
    \footnote{\cite{ZkbHandbuchDerItArchitektur} Kapitel 12.4.2 -
    \begin{itshape}Architektur beim Einsatz von
    Application-Servern\end{itshape}, Seite 150}
    Die technischen Services wie Session Management, Load Balancing,
    Transaction Management und Instance Pooling werden vom J2EE Application
    Server zur Verfügung gestellt.
    
    % Kapitel 12.4.3, Einsatz von Enterprise Java Beans, Seite 150
    \item[KO-29\label{itm:KO-29}]
    \footnote{\cite{ZkbHandbuchDerItArchitektur} Kapitel 12.4.3 -
    \begin{itshape}Einsatz von Enterprise Java Beans\end{itshape}, Seite 150}
    Das Muster JSP/Servlets mit EJBs ist anzuwenden, wenn die Applikation eine
    umfangreiche, komplexe Business-Logik aufweist, die Business-Logik
    wiederverwendbar sein soll, mehrere unterschiedliche Clients (Browser
    (Ultra-Thin-)(HTML), Thin-(Java), Mobile, …) mit einer Business Logik
    bedient werden müssen, hohe Anforderungen an die Skalierbarkeit gestellt
    werden und ein lange Lebenszyklus der Applikation erwartet wird.
    
    % Kapitel 12.4.3, Einsatz von Enterprise Java Beans, Seite 150
    \item[KO-30\label{itm:KO-30}]
    \footnote{\cite{ZkbHandbuchDerItArchitektur} Kapitel 12.4.3 -
    \begin{itshape}Einsatz von Enterprise Java Beans\end{itshape}, Seite 150}
    Das Muster JSP/Servlets ohne EJBs ist anzuwenden, wenn die Applikation eine
    einfache Business-Logik aufweist, nur einen Client, zum Beispiel ein
    Browser (Ultra-Thin-)-Interface unterstützt, niedrige Anforderungen an die
    Skalierbarkeit stellt und nur eine vergleichsweise kurzer Lebenszyklus der
    Applikation erwartet wird.
    
    % Kapitel 12.9.2, Standards, Seite 171
    \item[KO-31\label{itm:KO-31}]
    \footnote{\cite{ZkbHandbuchDerItArchitektur} Kapitel 12.9.2 -
    \begin{itshape}Standards\end{itshape}, Seite 171}
    In der ZKB werden folgende Standards für Web Services eingesetzt: SOAP,
    WSDL, W3C X Schema (XSD, WXS).
    
    % Kapitel 13.1, Client/Server-Konzept, Seite 175
    \item[KO-32\label{itm:KO-32}]
    \footnote{\cite{ZkbHandbuchDerItArchitektur} Kapitel 13.1 -
    \begin{itshape}Client/Server-Konzept\end{itshape}, Seite 175}
    Alle Neuentwicklungen sind konsequent in Client-/Server-Komponenten
    aufzuteilen.
    
    % Kapitel 13.2, Anwendungsstruktur (MVC/Model, View, Controller), Seite 175
    \item[KO-33\label{itm:KO-33}]
    \footnote{\cite{ZkbHandbuchDerItArchitektur} Kapitel 13.2 -
    \begin{itshape}Anwendungsstruktur (MVC/Model, View,
    Controller)\end{itshape}, Seite 175}
    Die gewünschte Isolation der Systemteile wird durch eine Anwendungsstruktur
    mit Trennung in Model (Verarbeitung, Datenhaltung), View
    (Benutzeroberfläche) und Controller (Organisation, Ereignisvermittlung)
    erreicht.
    
    % Kapitel 13.9.7.2, Allgemeine Grundsätze, Seite 189
    \item[KO-34\label{itm:KO-34}]
    \footnote{\cite{ZkbHandbuchDerItArchitektur} Kapitel 13.9.7.2 -
    \begin{itshape}Allgemeine Grundsätze\end{itshape}, Seite 189}
    Jede Applikation ist dafür verantwortlich, dass diejenigen Daten, für die
    sie den Lead hat, validiert sind.
    
    % Kapitel 13.9.7.2, Allgemeine Grundsätze, Seite 189
    \item[KO-35\label{itm:KO-35}]
    \footnote{\cite{ZkbHandbuchDerItArchitektur} Kapitel 13.9.7.2 -
    \begin{itshape}Allgemeine Grundsätze\end{itshape}, Seite 189}
    Jede Applikation benutzt Datenvalidierung um sicherzustellen, dass sie die
    erhaltenen Daten verarbeiten kann und dass ihr Betrieb nicht gefährdet ist
    (Stabilität / Verfügbarkeit).
    
    % Kapitel 13.9.7.3.1, Datenvalidierung an der Benutzersczhnittstelle
    % Seite 191
    \item[KO-36\label{itm:KO-36}]
    \footnote{\cite{ZkbHandbuchDerItArchitektur} Kapitel 13.9.7.3.1 -
    \begin{itshape}Datenvalidierung an der Benutzersczhnittstelle\end{itshape},
    Seite 191}
    Benutzereingaben werden so früh wie möglich validiert.
    
    % Kapitel 13.10, Programmiersprachen und Entwicklungsumgebungen
    % Seite 192
    \item[KO-37\label{itm:KO-37}]
    \footnote{\cite{ZkbHandbuchDerItArchitektur} Kapitel 13.10 -
    \begin{itshape}Programmiersprachen und Entwicklungsumgebungen\end{itshape},
    Seite 192}
    Für die Entwicklung neuer Applikationen und beim neuen Design bestehender
    Applikationen wird Java eingesetzt.
    
    % Kapitel 13.10.4.8, Zusätzliche Java Klassenbibliotheken, Seite 195
    \item[KO-38\label{itm:KO-38}]
    \footnote{\cite{ZkbHandbuchDerItArchitektur} Kapitel 13.10.4.8 -
    \begin{itshape}Zusätzliche Java Klassenbibliotheken\end{itshape}, Seite 195}
    Zusätzliche Java-Klassenbibliotheken, also solche, die nicht im JDK
    enthalten sind, werden nur in begründeten Ausnahmen eingesetzt.
    Normalerweise müssen solche Bibliotheken dem 100\%-Pure-Java-Grundsatz
    entsprechen. Ausnahmen können systemnahe Funktionen für Security und
    dergleichen sein.
    
    % Kapitel 13.11, Einsatz von .NET-basierten Applikationen, Seite 196
    \item[KO-39\label{itm:KO-39}]
    \footnote{\cite{ZkbHandbuchDerItArchitektur} Kapitel 13.11 -
    \begin{itshape}Einsatz von .NET-basierten Applikationen\end{itshape}, Seite
    196}
    .NET-basierte Applikationen werden in der ZKB Informatik nicht entwickelt
    oder zur Entwicklung in Auftrag gegeben ausser Kleinapplikationen der
    Kategorie „Office“.
    
    % Kapitel 13.13, Einsatz von Open Source Software, Seite 206
    \item[KO-40\label{itm:KO-40}]
    \footnote{\cite{ZkbHandbuchDerItArchitektur} Kapitel 13.13 -
    \begin{itshape}Einsatz von Open Source Software\end{itshape}, Seite 206}
    Die Nutzung von Open Source Software ist erlaubt.
    
    % Kapitel 13.13, Einsatz von Open Source Software, Seite 206
    \item[KO-41\label{itm:KO-41}]
    \footnote{\cite{ZkbHandbuchDerItArchitektur} Kapitel 13.13 -
    \begin{itshape}Einsatz von Open Source Software\end{itshape}, Seite 206}
    Open Source Software unterliegt denselben Kriterien wie kommerzielle
    Software. Sie muss evaluiert, registriert, homologiert, intern supported
    und gepflegt werden.    
    
    % Kapitel 13.13.1, Kriterien für die Evaluation von Open Source
    % Software, Seite 207
    \item[KO-42\label{itm:KO-42}]
    \footnote{\cite{ZkbHandbuchDerItArchitektur} Kapitel 13.13.1 -
    \begin{itshape}Kriterien für die Evaluation von Open Source
    Software\end{itshape}, Seite 207}
    Produktiv eingesetzte Open Source Software muss durch angemessene
    Informationsquellen unterstützt sein (Newsgroups, Mailinglists,
    FAQ-Listen, WebSites, User Groups).    
    
    % Kapitel 13.13.1, Kriterien für die Evaluation von Open Source
    % Software, Seite 207
    \item[KO-43\label{itm:KO-43}]
    \footnote{\cite{ZkbHandbuchDerItArchitektur} Kapitel 14.14.1 -
    \begin{itshape}Kriterien für die Evaluation von Open Source
    Software\end{itshape}, Seite 207}
    Die Weiterentwicklung von produktiv eingesetzter Open Source Software
    ausserhalb der ZKB muss öffentlich einsehbar sein und aktiv verfolgt werden
    können.
    
    % Kapitel 13.13.1, Kriterien für die Evaluation von Open Source
    % Software, Seite 207
    \item[KO-44\label{itm:KO-44}]
    \footnote{\cite{ZkbHandbuchDerItArchitektur} Kapitel 13.13.1 -
    \begin{itshape}Kriterien für die Evaluation von Open Source
    Software\end{itshape}, Seite 207}
    Die Lizenzbedingungen einer Open Source Software müssen vor dem Einsatz
    geprüft werden und von der Zürcher Kantonalbank akzeptiert werden können.
  \end{description}

  \section{Hirarchie der Anforderungen}
  
  \section{Auswahl der Frameworks}
  
  \section{Gewichtete Nutzwertanalyse mit Analytic Hirarchy Process}
  
  \section{Resultat}
 
  
  \cleardoublepage
  
  % ===========================================================================
  % Kapitel XXX beginns here
  % ===========================================================================
   
  \chapter{Integration in die ZKB Infrastruktur möglich?}

  \cleardoublepage
  
  % ===========================================================================
  % Kapitel XXX beginns here
  % ===========================================================================  
  
  \chapter{Proof of Concept - Umsetzung durch einen Prototypen}

  \cleardoublepage  
  
  % ===========================================================================
  % Kapitel XXX beginns here
  % ===========================================================================
   
  \chapter{Empfehlung für ein Java Web Framework}

  \cleardoublepage
  
  % ===========================================================================
  % Kapitel XXX beginns here
  % ===========================================================================
   
  \chapter{Fazit und Ausblick} 

  \cleardoublepage
  
  % ===========================================================================
  % Kapitel Reflektion beginns here
  % ===========================================================================
  
  \chapter{Reflektion}
  
  Die Arbeit war in meinen Auge ein voller Erfolg, da ich alle Ziele erreicht
  und viel gelernt habe. Ich bin bereit für einen neuen
  Lebensabschnitt.

  \cleardoublepage
  
  % ===========================================================================
  % Anhang beginns here
  % ===========================================================================
  
  \appendix
  
%  \chapter{Ein Testanhang}
%  
%  Im Anhang kann auf Implementierungsaspekte wie Datenbankschemata
%  oder Programmcode eingegangen werden.
  
  % ===========================================================================
  % Abkürzungsverzeichnis beginns here
  % ===========================================================================
  
  \chapter{Abkürzungsverzeichnis}

  \begin{acronym}[NWA]
  \setlength{\itemsep}{-\parsep}
  \acro{OO}{Objektorientierung}
  \acro{URL}{Uniform Resource Locator}
  \acro{RIA}{Rich Internet Application}   
  \acro{Ajax}{Asynchronous JavaScript and XML}
  \acro{ZSI}{Zentrale Server Infrastruktur}
  \acro{AHP}{Analytic Hierarchy Process}
  \acro{NWA}{Nutzwertanalyse}
  \acro{JFC}{Java Foundation Classes}
  \acro{JRE}{Java Runtime Environment}    
  \acro{EDT}{Event Dispatch Thread}
  \acro{UML}{Unified Modeling Language}
  \acro{IDE}{Integrierte Entwicklungsumgebung}
  \acro{GUI}{Grafische Benutzeroberfläche}
  \acro{EBS}{Einschreibe- und Bewertungssystem der HSZ-T}
\end{acronym}

  
  \cleardoublepage
  
  % ===========================================================================
  % Abbildungsverzeichnis beginns here
  % ===========================================================================
  
  % Abbildungsverzeichnis
  \listoffigures
  
  \cleardoublepage
  
  % ===========================================================================
  % Tabellenverzeichnis beginns here
  % ===========================================================================
  
  % Tabellenverzeichnis
  \listoftables
  
  \cleardoublepage
  
  % ===========================================================================
  % Literaturverzeichnis beginns here
  % ===========================================================================
  
  % verwendet alpha
  \bibliographystyle{alpha}
  % verwendet Literaturverzeichnis.bib
  % \renewcommand\bibname{Literaturverzeichnis} % Titel überschreiben
  \bibliography{Literaturverzeichnis}
  \cleardoublepage

  % ===========================================================================
  % Protokolle beginns here
  % ===========================================================================
 
  \chapter{Kick-off Protokoll}

    \section{Anwesende Personen}
    
  \begin{table}[ht]
    \begin{center}
      \begin{tabular}{rlc}
        \toprule
        Funktion & Person & Anwesend\\
        \midrule
        Student & Roman Würsch & Ja\\
        Auftraggeber & Bernhard Mäder, ZKB & Nein\\
        Projektbetreuer & Beat Seeliger & Ja\\
        Experte & --- & Nein\\
        Vertreter Studiengang Informatik & Olaf Stern & Ja\\
        \bottomrule
      \end{tabular}
      %\captionsetup{list=no}
      \caption{Anwesende Personen}
      \label{tab:anwesendePersonen}
    \end{center}
  \end{table}

  \section{Beschlüsse}
  \begin{itemize}
      \item Gemäss Email von Herr Stern vom 17. März 2011
      \begin{itemize}
        \item ``Sie führen in Absprache mit Ihrem Betreuer ein inhaltliches
        Kick-Off durch. Gibt Ihr Betreuer sein OK anschliessend, fahren Sie
        mit der Bearbeitung der Arbeit fort (kein Zeitverlust für Sie).''
        \item ``An dem von Ihnen gebuchten Termin am 13. April führen wir
        ein verkürztes Kick-Off durch, dieses wird auch als das formale
        Kick-Off in EBS eingetragen und protokolliert.''
      \end{itemize} 
      \item Gemäss Email von Herr Stern vom 15. März 2011: Es wurden
      die geforderten Änderungen an der Aufgabenstellung angepasst.
      \item Es soll ein RIA - Framework (z.B. ULC) mit in die Evaluation
      genommen werden.
      \item Es soll ein MVC - Framework (z.B. Struts) mit einbezogen
      werden.
      \item Es werden die Bewertungskriterien für die Bachelor Arbeit verwendet.
      \item Der Zeitplan ist straff, sportlich, sollte aber machbar sein.
  \end{itemize}
  
  \cleardoublepage
   
  \chapter{Design-review Protokoll}
  
    \section{Anwesende Personen}
    
  \begin{table}[ht]
    \begin{center}
      \begin{tabular}{rlc}
        \toprule
        Funktion & Person & Anwesend\\
        \midrule
        Student & Roman Würsch & Ja\\
        Auftraggeber & Bernhard Mäder, ZKB & Nein\\
        Projektbetreuer & Beat Seeliger & Ja\\
        Experte & --- & Nein\\
        Vertreter Studiengang Informatik & Olaf Stern & Ja\\
        \bottomrule
      \end{tabular}
      %\captionsetup{list=no}
      \caption{Anwesende Personen}
      \label{tab:anwesendePersonen}
    \end{center}
  \end{table}

  \section{Beschlüsse}
  \begin{itemize}
      \item Gemäss Email von Herr Stern vom 17. März 2011
      \begin{itemize}
        \item ``Sie führen in Absprache mit Ihrem Betreuer ein inhaltliches
        Kick-Off durch. Gibt Ihr Betreuer sein OK anschliessend, fahren Sie
        mit der Bearbeitung der Arbeit fort (kein Zeitverlust für Sie).''
        \item ``An dem von Ihnen gebuchten Termin am 13. April führen wir
        ein verkürztes Kick-Off durch, dieses wird auch als das formale
        Kick-Off in EBS eingetragen und protokolliert.''
      \end{itemize} 
      \item Gemäss Email von Herr Stern vom 15. März 2011: Es wurden
      die geforderten Änderungen an der Aufgabenstellung angepasst.
      \item Es soll ein RIA - Framework (z.B. ULC) mit in die Evaluation
      genommen werden.
      \item Es soll ein MVC - Framework (z.B. Struts) mit einbezogen
      werden.
      \item Es werden die Bewertungskriterien für die Bachelor Arbeit verwendet.
      \item Der Zeitplan ist straff, sportlich, sollte aber machbar sein.
  \end{itemize}

\end{document}