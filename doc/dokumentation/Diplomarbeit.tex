%
%   Technische Dokumentation für die Diplomarbeit
%
%   Created by Roman Wuersch on 2011-03-01.	
%
% =============================================================================
% Documentdefinition  beginns here
% =============================================================================

\documentclass[
11pt, % Schriftgrösse
a4paper, % A4 Papier
BCOR25mm, % Absoluter Wert der Bindekorrektur, z.B. BCOR1cm
DIV14, % Satzspiegel festlegen
footsepline = false, % Trennlinie zwischen Textkörper und Fußzeile
headsepline, % Trennlinie zwischen Kopfzeile und Textkörper
twoside, % Zweiseitig
openright,
%halfparskip, % Europäischer Satz mit Abstand zwischen den Absätzen
abstracton, % inkl. Abstract
listof=totocnumbered, % Abb.- und Tab.verzeichnis im Inhaltsverzeichnis
bibliography=totocnumbered % Lit.zeichnis in Inhaltsverzeichnis aufnehmen
]{scrreprt}

\usepackage[
automark,
%plainheadsepline,
%plainfootsepline
]{scrpage2} % Gestaltung von kopf- und Fußzeile

\usepackage[ngerman]{babel}

\usepackage{tocbasic}

% Use utf-8 encoding for foreign characters
\usepackage[utf8]{inputenc}
% \usepackage[applemac]{inputenc}
\usepackage[T1]{fontenc}

% Setup for fullpage use
%\usepackage{fullpage}

% Silbentrennung kann unterdrückt werden
\usepackage{hyphenat}

% Tradmark
\def\TTra{\textsuperscript{\texttrademark}}

%1.5 Zeilenabstand
\usepackage[onehalfspacing]{setspace}


% Festlegung des Seitenstils (scrpage2)
\pagestyle{scrheadings}
\clearscrheadfoot
\automark[chapter]{section}

\lehead{\sffamily\upshape\headmark}
\cehead{}
\rehead{}
\lefoot[\pagemark]{\upshape \pagemark}
\cefoot{}
\refoot{}

\lohead{}
\cohead{}
\rohead{\sffamily\upshape\headmark}
\lofoot{}
\cofoot{}
\rofoot[\pagemark]{\scshape \pagemark}

% Running Headers and footers
%\usepackage{fancyhdr}
%\setlength{\headheight}{15pt}

%\pagestyle{fancy}
%\renewcommand{\chaptermark}[1]{\markboth{\thechapter\ #1}{}}
%\renewcommand{\sectionmark}[1]{\markright{\thesection\ #1}{}}

%\fancyhf{}
%\fancyfoot[RE,LO]{\thepage}
%\fancyhead[LO]{\textit{\nouppercase{\leftmark}}}
%\fancyhead[RE]{\textit{\nouppercase{\rightmark}}}

%\renewcommand{\headrulewidth}{0.5pt}
%\renewcommand{\footrulewidth}{0pt}

%\fancypagestyle{plain}{%
%\fancyhf{}%
%\fancyfoot[RE,LO]{\thepage}
%\renewcommand{\headrulewidth}{0pt}
%\renewcommand{\footrulewidth}{0pt}
%}

%\setlength{\footskip}{1cm}

%\setlength{\headsep}{1.5cm}
%\addtolength{\textheight}{-1.5cm}

% Multipart figures
%\usepackage{subfigure}

% More symbols
%\usepackage{amsmath}
%\usepackage{amssymb}
%\usepackage{latexsym}

% Surround parts of graphics with box
\usepackage{boxedminipage}

% Package for including code in the document
\usepackage{listings}

% If you want to generate a toc for each chapter (use with book)
\usepackage{minitoc}

% Abkürzungsverzeichnis erstellen.
\usepackage[printonlyused]{acronym}

% schöne Tabelle zeichnen
\usepackage{booktabs}
\renewcommand{\arraystretch}{1.2} %Die Zeilenabstände in Tabllen angepasst.

% für variable Breiten
\usepackage{tabularx}

% Durchgestrichener Text
\usepackage[normalem]{ulem} %emphasize weiterhin kursiv

% This is now the recommended way for checking for PDFLaTeX:
\usepackage{ifpdf}

\usepackage[hyperfootnotes=false]{hyperref}
\hypersetup{
  bookmarks=true,         % show bookmarks bar?
  unicode=true,           % non-Latin characters in Acrobat’s bookmarks
  pdftoolbar=true,        % show Acrobat’s toolbar?
  pdfmenubar=true,        % show Acrobat’s menu?
  pdffitwindow=true,      % window fit to page when opened
  pdfstartview={FitH},    % fits the width of the page to the window
  pdftitle={Diplomarbeit},   
  pdfauthor={Roman Würsch},
  pdfsubject={Evaluation eines Java Web Frameworks zur Ablösung bestehender Java
  Swing Applikationen},
  pdfcreator={TeX Live 2009},
  pdfproducer={pdfTeX, Version 3.1415926-1.40.10},
  pdfnewwindow=true,      % links in new window
  colorlinks=false,       % false: boxed links; true: colored links
  linkcolor=red,          % color of internal links
  citecolor=green,        % color of links to bibliography
  filecolor=magenta,      % color of file links
  urlcolor=cyan          % color of external links
}

\ifpdf
    \usepackage[pdftex]{graphicx}
\else
    \usepackage{graphicx}
\fi

\title{Evaluation eines Java Web Frameworks zur Ablösung bestehender Java Swing
Applikationen}

\author{Diplomarbeit in Informatik\\
    \\
    Studierender - Roman Würsch\\
	Auftraggeber - Bernhard Mäder, ZKB\\
    Projektbetreuer - Beat Seeliger\\
    Experte - tbd\\
	\\
	HSZ-T - Technische Hochschule Zürich}

\date{März 2011 bis Juni 2011}

% =============================================================================
% Documenttext beginns here
% =============================================================================

\begin{document}

  \ifpdf
    \DeclareGraphicsExtensions{.pdf, .jpg, .tif}
  \else
    \DeclareGraphicsExtensions{.eps, .jpg}
  \fi
  
  % ===========================================================================
  % Titelblatt beginns here
  % ===========================================================================
  
  \maketitle
  
  % ===========================================================================
  % Abstract beginns here
  % ===========================================================================
  
  \pagenumbering{Alph}
  
  \begin{abstract}

  Es werden Methoden zur Analyse von Java Swing Applikationen und zur
  Evaluation von Java Web Frameworks ausgearbeitet werden.

  Mit der Methode zur Analyse werden bestehende Java Swing Applikationen der
  Zürcher Kantonalbank analysiert. Der Fokus liegt darauf in den Applikationen
  die gemeinsamen Muster, genutzter Swingkomponenten, zu erkennen und zu
  kategorisieren. Java Web Frameworks, welche sich am Markt etabliert haben,
  werden durch einen Evaluationsprozess auf deren Einsatz in der Zürcher
  Kantonalbank geprüft. Zusätzlich wird geprüft, ob mit den jeweiligen
  Frameworks die genutzten Swingkomponenten äquivalent umgesetzt werden können.
  Mit einer Analyse der IT Infrastruktur der Zürcher Kantonalbank, wird
  geprüft, ob eine Integration in die bestehende IT Infrastruktur möglich ist.

  Durch die Umsetzung eines Prototypen soll gezeigt werden, dass das evaluierte
  Java Web Framework den geforderten Funktionalitäten entspricht.

  Mit den gewonnenen Erkenntnissen, wird eine Empfehlung, für den Einsatz eines
  Java Web Frameworks zur Ablösung bestehender Java Swing Applikationen
  ausgesprochen.
  
\end{abstract}
  
  \cleardoublepage
  
  % ===========================================================================
  % Inahltsverzeichnis beginns here
  % ===========================================================================

  \pagenumbering{Roman}
  
  \tableofcontents
  
  \cleardoublepage
  
  % ===========================================================================
  % Kapitel Administratives beginns here
  % ===========================================================================
  
  \pagenumbering{arabic}
  
  \chapter{Personalienblatt}

  \begin{tabbing}
  \hspace*{6cm}\= \kill
  Name, Vorname: \> {\bf Roman Würsch} \\
  Adresse: \> {\bf Murhaldenweg 16} \\
  PLZ, Wohnort: \> {\bf 8057 Zürich} \\
  \\
  Geburtsdatum: \> {\bf 10.11.1980} \\
  Heimatort: \> {\bf Emmetten NW} \\
\end{tabbing}

\noindent
Ich bestätige, dass die vorliegende Diplomarbeit, ``Evaluation eines Java Web
Frameworks zur Ablösung bestehender Java Swing Applikationen'', in allen Teilen
selbständig erarbeitet und durchgeführt wurde.

\vspace*{3cm}

\begin{tabbing}
  \hspace*{8cm}\= \kill
  Ort und Datum \> {Unterschrift} \\
\end{tabbing}
  
  % ===========================================================================
  % Kapitel Aufgabenstellung beginns here
  % ===========================================================================
  
  %\newpage
  
  \chapter{Aufgabenstellung}
  
  \section{Ausgangslage}
  
  Die Zürcher Kantonalbank hat viele Applikationen welche als Client Applikationen
in Swing implementiert sind. Dabei ist das Deployment der Clients und die
Ausbreitung entsprechender Patches, innerhalb der Zürcher Kantonalbank, mit
einem Mehraufwand verbunden. Client Applikationen, die einem Kunden zur
Verfügung gestellt werden, sind an eine spezifische Java Version gebunden. Bei
der Integration in die IT-Landschaft beim Kunden, kann das zur Verzögerung
durch einen erhöhten Testaufwand führen. In der Annahme, dass ein Webbrowser in
der Zürcher Kantonalbank und bei deren Kunden eingesetzt wird, macht der
Einsatz einer Weblösung Sinn.

  
  \section{Ziel der Arbeit}

  Es sollen bestehende Java Swing Applikationen der Zürcher Kantonalbank
analysiert werden. In diesen Applikationen sollen die gemeinsamen Muster,
genutzter Swingkomponenten, erkannt und kategorisiert werden. Java Web
Frameworks, welche sich am Markt etabliert haben, sollen auf einen möglichen
Einsatz geprüft werden. Es soll geprüft werden, ob mit den jeweiligen
Frameworks die genutzten Swingkomponenten äquivalent umgesetzt werden können.
Zudem soll geprüft werden, ob eine Integration in die bestehende IT
Infrastruktur der Zürcher Kantonalbank möglich ist.

  
  %\newpage
  
  \section{Aufgabenstellung}
  
    Die Aufgabenstellung besteht aus den vier Teilen: Ausgangslage, Ziel der
  Arbeit, Aufgabenstellung und Erwartete Resultate. Zusätzlich wurde eine
  Abgrenzung hinzugefügt, damit der Rahmen der Diplomarbeit gesetzt ist.
  
  \section{Ausgangslage}
  
  Die Zürcher Kantonalbank hat viele Applikationen welche als Client Applikationen
in Swing implementiert sind. Dabei ist das Deployment der Clients und die
Ausbreitung entsprechender Patches, innerhalb der Zürcher Kantonalbank, mit
einem Mehraufwand verbunden. Client Applikationen, die einem Kunden zur
Verfügung gestellt werden, sind an eine spezifische Java Version gebunden. Bei
der Integration in die IT-Landschaft beim Kunden, kann das zur Verzögerung
durch einen erhöhten Testaufwand führen. In der Annahme, dass ein Webbrowser in
der Zürcher Kantonalbank und bei deren Kunden eingesetzt wird, macht der
Einsatz einer Weblösung Sinn.

  
  \section{Ziel der Arbeit}

  Es sollen bestehende Java Swing Applikationen der Zürcher Kantonalbank
analysiert werden. In diesen Applikationen sollen die gemeinsamen Muster,
genutzter Swingkomponenten, erkannt und kategorisiert werden. Java Web
Frameworks, welche sich am Markt etabliert haben, sollen auf einen möglichen
Einsatz geprüft werden. Es soll geprüft werden, ob mit den jeweiligen
Frameworks die genutzten Swingkomponenten äquivalent umgesetzt werden können.
Zudem soll geprüft werden, ob eine Integration in die bestehende IT
Infrastruktur der Zürcher Kantonalbank möglich ist.

  
  \section{Aufgabenstellung}
  
    Die Aufgabenstellung besteht aus den vier Teilen: Ausgangslage, Ziel der
  Arbeit, Aufgabenstellung und Erwartete Resultate. Zusätzlich wurde eine
  Abgrenzung hinzugefügt, damit der Rahmen der Diplomarbeit gesetzt ist.
  
  \section{Ausgangslage}
  
  Die Zürcher Kantonalbank hat viele Applikationen welche als Client Applikationen
in Swing implementiert sind. Dabei ist das Deployment der Clients und die
Ausbreitung entsprechender Patches, innerhalb der Zürcher Kantonalbank, mit
einem Mehraufwand verbunden. Client Applikationen, die einem Kunden zur
Verfügung gestellt werden, sind an eine spezifische Java Version gebunden. Bei
der Integration in die IT-Landschaft beim Kunden, kann das zur Verzögerung
durch einen erhöhten Testaufwand führen. In der Annahme, dass ein Webbrowser in
der Zürcher Kantonalbank und bei deren Kunden eingesetzt wird, macht der
Einsatz einer Weblösung Sinn.

  
  \section{Ziel der Arbeit}

  Es sollen bestehende Java Swing Applikationen der Zürcher Kantonalbank
analysiert werden. In diesen Applikationen sollen die gemeinsamen Muster,
genutzter Swingkomponenten, erkannt und kategorisiert werden. Java Web
Frameworks, welche sich am Markt etabliert haben, sollen auf einen möglichen
Einsatz geprüft werden. Es soll geprüft werden, ob mit den jeweiligen
Frameworks die genutzten Swingkomponenten äquivalent umgesetzt werden können.
Zudem soll geprüft werden, ob eine Integration in die bestehende IT
Infrastruktur der Zürcher Kantonalbank möglich ist.

  
  \section{Aufgabenstellung}
  
    Die Aufgabenstellung besteht aus den vier Teilen: Ausgangslage, Ziel der
  Arbeit, Aufgabenstellung und Erwartete Resultate. Zusätzlich wurde eine
  Abgrenzung hinzugefügt, damit der Rahmen der Diplomarbeit gesetzt ist.
  
  \section{Ausgangslage}
  
  \input{../aufgabenstellung/kapitel/ausgangslage.tex}
  
  \section{Ziel der Arbeit}

  \input{../aufgabenstellung/kapitel/zielDerArbeit.tex}
  
  \section{Aufgabenstellung}
  
  \input{../aufgabenstellung/kapitel/aufgabenstellung.tex}
  
  \section{Erwartete Resultate}
  
  \input{../aufgabenstellung/kapitel/erwarteteResultate.tex}
  
  \section{Abgrenzung}
  
  \input{../aufgabenstellung/kapitel/abgrenzung.tex}
  
  \section{Erwartete Resultate}
  
  Der Studierende soll dem Auftraggeber ein Dokument erstellen, das folgendes
beinhaltet:

\begin{itemize}    
  \item Ergebnis der Analyse von bestehenden Java Swing Applikationen der
  Zürcher Kantonalbank.
  \item Kategorisierung von verwendeten Swingkomponenten.
  \item Eine Auflistung von etablierten Java Web Frameworks.
  \item Ergebnis der Analyse, ob eine Integration der Java Web Frameworks, in
  der bestehenden IT Infrastruktur der Zürcher Kantonalbank, möglich ist.
  \item Ergebnis der Analyse von Java Web Frameworks, ob eine Implementierung,
  der erkannten Swingkomponenten, möglich ist.
  \item Proof of concept. Es soll anhand eines Prototypen gezeigt werden, dass
  die Implementierung möglich ist.
  \item Eine Empfehlung für ein Java Web Framework.
\end{itemize}
  
  \section{Abgrenzung}
  
  Folgende Punkte werden formell abgegrenzt:

\begin{itemize}
  \item Die Analysen beschränken sich auf Recherchen im Internet, Büchern und
  interne Vorgaben der Zürcher Kantonalbank.
  \item Umfragen, Erhebungen sowie Feldstudien werden nicht durchgeführt.
\end{itemize}

\noindent  
Folgende Punkte werden inhaltlich abgegrenzt:
	
\begin{itemize}
  \item Die Auswahl, welche Java Swing Applikationen analysiert werden, soll
  wärend der Arbeit durchgeführt werden.
  \item Die Auswahl, welche Java Web Frameworks geprüft werden, soll wärend  der
  Arbeit durchgeführt werden.
\end{itemize}
  
  \section{Erwartete Resultate}
  
  Der Studierende soll dem Auftraggeber ein Dokument erstellen, das folgendes
beinhaltet:

\begin{itemize}    
  \item Ergebnis der Analyse von bestehenden Java Swing Applikationen der
  Zürcher Kantonalbank.
  \item Kategorisierung von verwendeten Swingkomponenten.
  \item Eine Auflistung von etablierten Java Web Frameworks.
  \item Ergebnis der Analyse, ob eine Integration der Java Web Frameworks, in
  der bestehenden IT Infrastruktur der Zürcher Kantonalbank, möglich ist.
  \item Ergebnis der Analyse von Java Web Frameworks, ob eine Implementierung,
  der erkannten Swingkomponenten, möglich ist.
  \item Proof of concept. Es soll anhand eines Prototypen gezeigt werden, dass
  die Implementierung möglich ist.
  \item Eine Empfehlung für ein Java Web Framework.
\end{itemize}
  
  \section{Abgrenzung}
  
  Folgende Punkte werden formell abgegrenzt:

\begin{itemize}
  \item Die Analysen beschränken sich auf Recherchen im Internet, Büchern und
  interne Vorgaben der Zürcher Kantonalbank.
  \item Umfragen, Erhebungen sowie Feldstudien werden nicht durchgeführt.
\end{itemize}

\noindent  
Folgende Punkte werden inhaltlich abgegrenzt:
	
\begin{itemize}
  \item Die Auswahl, welche Java Swing Applikationen analysiert werden, soll
  wärend der Arbeit durchgeführt werden.
  \item Die Auswahl, welche Java Web Frameworks geprüft werden, soll wärend  der
  Arbeit durchgeführt werden.
\end{itemize}
  
  \section{Erwartete Resultate}
  
  Der Studierende soll dem Auftraggeber ein Dokument erstellen, das folgendes
beinhaltet:

\begin{itemize}    
  \item Ergebnis der Analyse von bestehenden Java Swing Applikationen der
  Zürcher Kantonalbank.
  \item Kategorisierung von verwendeten Swingkomponenten.
  \item Eine Auflistung von etablierten Java Web Frameworks.
  \item Ergebnis der Analyse, ob eine Integration der Java Web Frameworks, in
  der bestehenden IT Infrastruktur der Zürcher Kantonalbank, möglich ist.
  \item Ergebnis der Analyse von Java Web Frameworks, ob eine Implementierung,
  der erkannten Swingkomponenten, möglich ist.
  \item Proof of concept. Es soll anhand eines Prototypen gezeigt werden, dass
  die Implementierung möglich ist.
  \item Eine Empfehlung für ein Java Web Framework.
\end{itemize}
  
  \section{Abgrenzung}
  
  Folgende Punkte werden formell abgegrenzt:

\begin{itemize}
  \item Die Analysen beschränken sich auf Recherchen im Internet, Büchern und
  interne Vorgaben der Zürcher Kantonalbank.
  \item Umfragen, Erhebungen sowie Feldstudien werden nicht durchgeführt.
\end{itemize}

\noindent  
Folgende Punkte werden inhaltlich abgegrenzt:
	
\begin{itemize}
  \item Die Auswahl, welche Java Swing Applikationen analysiert werden, soll
  wärend der Arbeit durchgeführt werden.
  \item Die Auswahl, welche Java Web Frameworks geprüft werden, soll wärend  der
  Arbeit durchgeführt werden.
\end{itemize}
  
  \newpage
  
  % ===========================================================================
  % Kapitel Erreichte Ziele beginns here
  % ===========================================================================  
  
  \chapter{Erreichte Ziele}
  
  Es wurden alle Ziele gemäss den erwarteten Resultaten der Aufgabenstellung
  erreicht. Die einzelnen Punkte sind hier analog der Aufgabenstellung
  aufgeführt:
  
  \begin{description}
    \item [Noch nicht Erreicht] Ergebnis der Analyse von bestehenden Java Swing
    Applikationen der Zürcher Kantonalbank.
    \item [Noch nicht Erreicht] Kategorisierung von verwendeten
    Swingkomponenten.
    \item [Noch nicht Erreicht] Eine Auflistung von etablierten Java Web
    Frameworks.
    \item [Noch nicht Erreicht] Ergebnis der Analyse, ob eine Integration der
    Java Web Frameworks, in der bestehenden IT Infrastruktur der Zürcher
    Kantonalbank, möglich ist.
    \item [Noch nicht Erreicht] Ergebnis der Analyse von Java Web Frameworks, ob
    eine Implementierung, der erkannten Swingkomponenten, möglich ist.
    \item [Noch nicht Erreicht] Proof of concept. Es soll anhand eines
    Prototypen gezeigt werden, dass die Implementierung möglich ist.
    \item [Noch nicht Erreicht] Eine Empfehlung für ein Java Web Framework.
  \end{description}
  
  % ===========================================================================
  % Kapitel Beschreibung der Ausgangslage beginns here
  % ===========================================================================
  
  \chapter{Beschreibung der Ausgangslage}
  
  Für das Informatik Diplomstudium an der Fachhochschule Zürich für Technik
  HSZ-T wird von den Studenten verlangt eine Diplomarbeit eigenständig zu
  verfassen.
  
  \section{Formell}
  
  Es wird auf die Sprache, die Richtlinien und die verwendeten
  Bewertungskriterien hingewiesen.
  
  \subsection{Sprache}
  
  Der Bericht wird in deutscher Sprache verfasst. Englische Ausdrücke werden im
  Kontext verwendet, wenn man davon ausgehen kann, dass es gängige Ausdrücke aus
  dem Gebiet der Informatik sind.
    
  \subsection{Richtlinien}
  Folgende Dokumente mit Richtlinien der Hochschule für Technik Zürich wurden
  für die Diplomarbeit berücksichtigt:

  \begin{itemize}
      \item Bestimmungen für die Diplomarbeit \cite{hsz_reglement}
      \item Ablauf Diplomarbeit \cite{hsz_ablauf}
      \item Bewertungskriterien Diplomarbeit \cite{hsz_bewertungskriterien}
  \end{itemize} 
    
  \subsection{Bewertungskriterien}
  
  Es werden die Bewertungskriterien für die Diplomarbeit gemäss
  \cite{hsz_bewertungskriterien} verwendet.
  
  \section{Inhaltlich}
  
  \subsection{Wahl des Themas}
  
  Die Zürcher Kantonalbank setzt bei der Entwicklung von Software als Inhouse
  Lösungen auf Java Swing Applikationen und auf Java Web Applikationen. Die
  Java Web Applikationen basieren auf dem Web Framework Apache Struts 1.3
  mit einer von der ZKB erstellten Erweiterung namens ZIP (ZKB Internet
  Plattform). Da eine Lösung als Java Web Applikationen zentral verwaltet
  werden kann gibt es erhebliche Einsparungen im Bereich Testing, Deployment
  und Patching.
  
  Das Framework Apache Struts 1.3 ist mitlerweilen in die Jahre gekommen. Es
  basiert auf dem Prinzip von ``request und response''. Wann immer eine Aktion
  von einem User in der Applikation gemacht wird, sei es das Ansteuern eines
  Links oder das Absenden eines Formulars, wird die Webseite, welche die
  Applikation repräsentiert, neu geladen. Heutzutage gibt es Frameworks, welche
  einem ermöglichen eine Web Applikation zu entwicklen, die sich bei der
  Bedienung wie eine klassische Desktop Applikation anfühlen. Dies wird durch
  die Technik von \ac{Ajax} realisiert.
  
  Bei \ac{Ajax} können Daten einer Web Applikation, ohne die komplette Webseite
  neu zu laden, verändert werden. Dies erlaubt es Web Applikationen, auf
  Benutzer Aktionen schneller zu reagieren, da vermieden wird, dass statische
  Daten, die sich unter Umständen nicht geändert haben, immer wieder über das
  Netzwerk übertragen werden müssen.
  
  Apache Struts 1.3 unterstützt \ac{Ajax} nicht direkt, kann aber über
  Erweiterungen ermöglicht werden. Es gibt Java Web Frameworks, welche diese
  Technik out of the box mitbringen. Eine Evaluation soll zeigen welches dieser
  Java Web Frameworks am besten für einen möglichen Einsatz geeignet ist.
  
  \subsection{Detailanalyse der Aufgabenstelltung}
  
  \newpage
  
  \chapter{Projektadministration}
  
  Damit eine Nachvollziehbarkeit garantiert ist, werden hier die wichtigsten
  Termine, Meilensteine und Arbeitsschritte aufgelistet.
  
  \section{Projekt Termine}
  
  Die Projekt Termine wurden alle eingehalten, siehe Tabelle \ref{tab:termine}.
  \newline
  
  \begin{table}[h]
    \begin{center}
      \begin{tabular}{lp{7cm}ll}
        \toprule
        Termin & Datum & Ort \\
        \midrule
        21. 03. 2011 & Inhaltliches Kick-off Meeting & Panter llc\\
        13. 04. 2011 & Offizielles Kick-off Meeting & HSZ-T\\
        xx. xx. 2011 & Design-Review Meeting & HSZ-T\\
        xx. xx. 2011 & Abgabe der Dokumentation & HSZ-T\\
        xx. xx. 2011 & Schlusspräsentation & HSZ-T\\
        \bottomrule
      \end{tabular}
      \caption{Projekt Termine}
      \label{tab:termine}
    \end{center}
  \end{table}
  
  \section{Projekt Meilensteine}
  
  In der Projekt Historie sind die wichtigsten Meilensteine ersichtlich, siehe
  Tabelle \ref{tab:projekthistorie}.
  \newline
  
  \begin{table}[h]
    \begin{center}
      \begin{tabular}{lp{9cm}ll}
        \toprule
        Datum & Status & Wer \\
        \midrule
        14. 03. 2011 & Ein Dozierender hat die Arbeit inkl. Aufgabenstellung
        ausgeschrieben und wartet auf einen Studierenden der  diese Arbeit
        durchführt & Roman Würsch\\
        14. 03. 2011 & Eingabe Aufgabenstellung & Roman Würsch\\
        15. 03. 2011 & Die Arbeit ist freigegeben (Eine Semester- oder
        Bachelorarbeit kann nur durch die Studiengangsleitung freigegeben
        werden) & Olaf Stern\\
        16. 03. 2011 & Der Kickoff-Termin wurde reserviert & Roman Würsch\\
        \bottomrule
      \end{tabular}
      \caption{Projekt Historie}
      \label{tab:projekthistorie}
    \end{center}
  \end{table}
  
  \section{Arbeitsschritte}
  
  Alle vorgenommenen Arbeitsschritte werden in einem Wiki für die
  Nachvollziehbarkeit protokolliert. Das Wiki ist im Internet öffentlich
  zugänglich unter der \ac{URL}:\\
  \\
  \url{https://github.com/sushicutta/Diplomarbeit/wiki/Arbeitsprotokoll}. 
  
  \newpage
  
  % ===========================================================================
  % Kapitel XXX beginns here
  % ===========================================================================
  
  \chapter{Java Swing Applikationen}

  \section{Grundlagen}
  
  \subsection{JAVA}
  
  \subsection{Programmiermodell}
  
  \subsubsection{Komponenten}
  
  \subsubsection{Layouts}
    
  \subsubsection{Events}
  
  \subsection{Look \& Feel}
  
  \subsection{Multithreading}
  
  \subsection{Konkurenz (SWT, \ldots)}
  
  \chapter{Rich Internet Applikationen}
  
  \section{Grundlagen}

  \subsection{Browser orientiert}
  
  \subsection{Plugin orientiert}
  
  \section{Technologien}
  
  \subsection{Eingesetzte Standards}
  
  \subsection{Programmiermodell}
  
  \section{Security}
  
  \section{Merkmale}
    
  \subsection{Suchmaschinenoptimierung}
  
  \subsection{Barrierefreiheit}
  
  \chapter{Analyse der Swing Applikationen}
  
  \section{Auswahl der Applikationen}
  
  \section{Kategorisierung von verwendeten Swing Komponenten}
  
  \chapter{Infrastruktur der Zürcher Kantonalbank}
  
  \section{Web Application Stack}
  
  \section{Business Layer}
  
  \section{Frontend}
  
  \chapter{Analyse der Web Frameworks}
  
  \section{Methoden zur Entscheidungsfindung bei einer Evaluation}
  
  \subsection{Gewichtete Nutzwertanalyse}
    
  \subsection{Analytic Hierarchy Process}
    
  \section{Auswahl der Web Frameworks}
    
  \subsection{KO-Kriterien}
  
  Die Anforderungen an die Auswahl von Web Frameworks wird anhand verschiedener
  KO-Kriterien eingeschränkt. Es wird dabei unterschieden zwischen allgemeinen
  Kriterien und Vorgaben aus der IT-Architektur der Zürcher Kantonalbank. Jedem
  KO-Kriterium wird für die Identifikation eine eindeutige ID vergeben. Die ID
  setzt sich folgendermassen zusammen. \{KO\}-\{ZKB|ALL\}-\{Laufnummer\}.
  
  \subsubsection{Allgemeine Kriterien}
  
  \subsubsection{Vorgaben aus der IT-Architektur der Zürcher Kantonalbank}
  
  Folgende KO-Kriterien sind aus dem \begin{itshape}Handbuch der
  IT-Architektur\end{itshape}\cite{ZkbHandbuchDerItArchitektur} der Züricher
  Kantonalbank entnommen. Die Namensgebung unterscheidet sich leicht, es wird
  nicht von KO-Kriterien, sondern von Grundsätzen gesprochen. Ein Grundsatz
  wird wie folgt definiert:\\
  
  ``\begin{itshape}Es sind Grundsätze definiert, nach denen sich die Baupläne
  der IT-Systeme zu richten haben. Die Grundsätze sind ein Regelwerk mit
  Weisungscharakter.\end{itshape}''
  \footnote{\cite{ZkbHandbuchDerItArchitektur} Kapitel 1.3 - \begin{itshape}Was ist die IT-Architektur der
  ZKB\end{itshape}, Seite 11}
  \\
  \\
  \noindent
  Dabei gibt es eine Hintertür:\\

  ``\begin{itshape}Grundsätze sind verbindliche Vorgaben (Konventionen), von
  denen nur in begründeten Ausnahmen abgewichen werden kann.\end{itshape}''
  \footnote{\cite{ZkbHandbuchDerItArchitektur} Kapitel 1.8 - \begin{itshape}Leseanleitung\end{itshape},
  Seite 14}
  \\
  \\
  \noindent
  Das Dokument wurde analysiert und die Liste der relevanten Grundsätze wird
  hier aufgelistet:
  
  \begin{description}

    % Kapitel 2.2.2, N-Tier Applikationen, Seite 20
    \item[KO-ZKB-01]
    \footnote{\cite{ZkbHandbuchDerItArchitektur} Kapitel 2.2.2 - \begin{itshape}N-Tier
    Applikationen\end{itshape}, Seite 20}
    Applikationen sollen als N-Tier Applikationen designed und
    implementiert werden.

    % Kapitel 2.2.10, Objektorientierung, Seite 22
    \item[KO-ZKB-02] \ac{OO} soll innerhalb der Informatik für Neuentwicklungen
    durchgängig angewandt werden.
    \footnote{\cite{ZkbHandbuchDerItArchitektur} Kapitel 2.2.10 -
    \begin{itshape}Objektorientierung\end{itshape}, Seite 22}

    % Kapitel 3.3, Mehrsprachigkeit, Seite 36
    \item[KO-ZKB-03] Eine neue Applikation (oder eine neue Komponente einer
    bestehenden Applikation) ist mehrsprachfähig zu realisieren.
    \footnote{\cite{ZkbHandbuchDerItArchitektur} Kapitel 3.3 -
    \begin{itshape}Mehrsprachigkeit\end{itshape}, Seite 36}

    % Kapitel 3.3, Mehrsprachigkeit, Seite 37
    \item[KO-ZKB-04] Neue Applikationen sind Unicode-fähig zu realisieren
    \footnote{\cite{ZkbHandbuchDerItArchitektur} Kapitel 3.3 -
    \begin{itshape}Mehrsprachigkeit\end{itshape}, Seite 37}

    % Kapitel 3.9.1, Skalierbarkeit / Ausfallsicherheit / Perfomance
    % Seite 49
    \item[KO-ZKB-05] Eine Applikation muss in mehreren Instanzen lauffähig sein.
    \footnote{\cite{ZkbHandbuchDerItArchitektur} Kapitel 3.9.1 -
    \begin{itshape}Skalierbarkeit / Ausfallsicherheit / Perfomance\end{itshape},
    Seite 49}

    % Kapitel 4.4.1, User Interface, Seite 55
    \item[KO-ZKB-06] Der ZKB GUI Style Guide ist in allen ZKB IT-Projekten
    anzuwenden
    \footnote{\cite{ZkbHandbuchDerItArchitektur} Kapitel 4.4.1 -
    \begin{itshape}User Interface\end{itshape}, Seite 55}
    
    % Kapitel 5.2, Verwendung der Zentralen Server Infrastruktur ZSI, Seite 57
    \item[KO-ZKB-07] Die ZSI (Zentrale Server Infrastruktur) ist als
    Server-Plattform für Applikationen, welche Windows-, Unix-, Linux-basierte
    Server einsetzen, zu verwenden.
    \footnote{\cite{ZkbHandbuchDerItArchitektur} Kapitel 5.2 -
    \begin{itshape}Verwendung der Zentralen Server Infrastruktur
    ZSI\end{itshape}, Seite 57}
    
    % Kapitel 9.2, Java RMI, Seite 71
    \item[KO-ZKB-08] RMI kann für reine Java-Anwendungen eingesetzt werden.
    \footnote{\cite{ZkbHandbuchDerItArchitektur} Kapitel 9.2 -
    \begin{itshape}Java RMI\end{itshape}, Seite 71}
    
    % Kapitel 12.2.1, Einsatz von Frameworks, Seite 140
    \item[KO-ZKB-09] Für Java-Applikationen (Internet, Extranet und Intranet)
    wird das ZIP-Framework eingesetzt.
    \footnote{\cite{ZkbHandbuchDerItArchitektur} Kapitel 12.2.1 -
    \begin{itshape}Einsatz von Frameworks\end{itshape}, Seite 140}
    
    % Kapitel 12.3.5, Client-/Server-Schemata von Internet-Applikationen,
    % Seite 141
    \item[KO-ZKB-10] Die Validierung und Plausibilisierung der Eingaben erfolgt
    immer abschliessend auf dem bankseitigen Applikations-Server. Es ist aber
    durchaus möglich, dass sich auf der Client-Seite eine Logik zur
    Überprüfung und Validierung der Eingaben für den Benutzerkomfort
    befindet.
    \footnote{\cite{ZkbHandbuchDerItArchitektur} Kapitel 12.3.5 -
    \begin{itshape}Client-/Server-Schemata von Internet-Applikationen\end{itshape}, Seite 141}
    
    % Kapitel 12.3.5, Client-/Server-Schemata von Internet-Applikationen,
    % Seite 141
    \item[KO-ZKB-11] Die Business-Logik in einer Internet-Applikation ist so
    auszulegen, dass diese von verschiedenen Präsentations-Logiken im Rahmen
    von Ultra-Thin- und Thin-Client-Applikationen genutzt werden kann.
    \footnote{\cite{ZkbHandbuchDerItArchitektur} Kapitel 12.3.5 -
    \begin{itshape}Client-/Server-Schemata von Internet-Applikationen\end{itshape}, Seite 141}
    
    % Kapitel 12.3.6.1, Browser-Abhängigkeiten, Seite 142
    \item[KO-ZKB-12] Die Internet-Applikationen der ZKB werden nicht mit
    Browser-Abhängigkeiten versehen und orientieren sich an den neutralen
    Standards der W3C-Komission.
    \footnote{\cite{ZkbHandbuchDerItArchitektur} Kapitel 12.3.6.1 -
    \begin{itshape}Browser-Abhängigkeiten\end{itshape}, Seite 142}
    
    % Kapitel 12.3.6.2, Session-Mechanismus, Seite 142
    \item[KO-ZKB-13] Für Ultra-Thin-Client-Applikationen wird als
    Session-Mechanismus die Cookie- oder die URL-Rewriting-Methode angewendet.
    \footnote{\cite{ZkbHandbuchDerItArchitektur} Kapitel 12.3.6.2 -
    \begin{itshape}Session-Mechanismus\end{itshape}, Seite 142}
    
    % Kapitel 12.3.6.4, Einfache Internet-Applikationen, Seite 143
    \item[KO-ZKB-14] Einfache Internet-Applikationen mit dem Schwerpunkt
    Information können ausschliesslich Java Server Pages verwenden.
    \footnote{\cite{ZkbHandbuchDerItArchitektur} Kapitel 12.3.6.4 -
    \begin{itshape}Einfache Internet-Applikationen\end{itshape}, Seite 143}
    
    % Kapitel 12.3.6.5, Komplexe Internet-Applikationen, Seite 143
    \item[KO-ZKB-15] Komplexe Internet-Applikationen verwenden eine Kombination
    von Java Server Pages und mindestens einem Servlet als Dispatcher-Mechanismus.
    \footnote{\cite{ZkbHandbuchDerItArchitektur} Kapitel 12.3.6.5 -
    \begin{itshape}Komplexe Internet-Applikationen\end{itshape}, Seite 143}
    
    % Kapitel 12.3.6.6, Verwendung von JavaScript beziehungsweise ECMAScript
    % Seite 143
    \item[KO-ZKB-16] Die Internet-Applikationen funktionieren auch
    eingeschränkt, ohne dass die Skript-Funktion im Browser aktiviert ist.
    \footnote{\cite{ZkbHandbuchDerItArchitektur} Kapitel 12.3.6.6 -
    \begin{itshape}Verwendung von JavaScript beziehungsweise ECMAScript\end{itshape}, Seite 143}
    
    % Kapitel 12.3.6.7, Einstz von ActiveX und Cookies, Seite 143
    \item[KO-ZKB-17] ActiveX wird wegen der Möglichkeit für direkte Zugriffe auf
    das Be-triebssystem nicht eingesetzt.
    \footnote{\cite{ZkbHandbuchDerItArchitektur} Kapitel 12.3.6.6 -
    \begin{itshape}Verwendung von JavaScript beziehungsweise ECMAScript\end{itshape}, Seite 143}
    
    % Kapitel 12.3.6.8, Applets, Seite 144
    \item[KO-ZKB-18] Es werden keine neuen Applikationen als Java Applets
    entwickelt. Der Einsatz von Applets beschränkt sich auf einfache Funktionen wie Börsen-
    oder News-Ticker.
    \footnote{\cite{ZkbHandbuchDerItArchitektur} Kapitel 12.3.6.8 -
    \begin{itshape}Applets\end{itshape}, Seite 144}
    
    % Kapitel 12.3.6.9, Browser-Plugins, Seite 144
    \item[KO-ZKB-19] Internet-Applikation werden ohne Plugins entwickelt.
    \footnote{\cite{ZkbHandbuchDerItArchitektur} Kapitel 12.3.6.9 -
    \begin{itshape}Browser-Plugins\end{itshape}, Seite 144}
    
    % Kapitel 12.3.7, Client-Technologien, Seite 144
    \item[KO-ZKB-20] Für Ultra Thin Clients bzw. Browser-basierende
    Applikationen muss das aktuelle, Struts-basierende HTML-Client-Framework der ZKB Internet
    Platt-form verwendet werden.
    \footnote{\cite{ZkbHandbuchDerItArchitektur} Kapitel 12.3.7 -
    \begin{itshape}Client-Technologien\end{itshape}, Seite 144}
    
    % Kapitel 12.3.8, Layering von Internet-Applikationen, Seite 145
    \item[KO-ZKB-21] Neue Internet- und Extranet-Applikationen müssen sich an
    das Laye-ring gemäss nachfolgender Grafik @@@@@@@@@ HIER NOCH DIE GRAFIK
    REFERENZIEREN halten. Für Intranet-Applikationen ist die
    Validator-Komponente fakultativ.
    \footnote{\cite{ZkbHandbuchDerItArchitektur} Kapitel 12.3.8 -
    \begin{itshape}Layering von Internet-Applikationen\end{itshape}, Seite 145}
    
    % Kapitel 12.3.8, Layering von Internet-Applikationen, Seite 146
    \item[KO-ZKB-22] Eine Applikation muss ohne Änderung von der
    Intranet-Anwendung zur Extra- oder Internet-Applikation gemacht werden können. Es geschieht
    dies lediglich durch das Vorschalten der Validator-Komponente.
    \footnote{\cite{ZkbHandbuchDerItArchitektur} Kapitel 12.3.8 -
    \begin{itshape}Layering von Internet-Applikationen\end{itshape}, Seite 146}
    
    % Kapitel 12.3.10.2, Web Server, Seite 146
    \item[KO-ZKB-23] Der ZKB Standard-Web-Server ist der Apache HTTP-Server.
    \footnote{\cite{ZkbHandbuchDerItArchitektur} Kapitel 12.3.10.2 -
    \begin{itshape}Web Server\end{itshape}, Seite 146}
    
    % Kapitel 12.4.2, Architektur beim Einsatz von Application-Servern,
    % Seite 147
    \item[KO-ZKB-24] Der Application-Server wird als die integrierte technische
    Middleware für die Unterstützung von Java Server Pages (JSP), Servlets,
    Enterprise Java Beans (EJB) und der sicheren Kommunikation zwischen Client
    und Server einge-setzt.
    \footnote{\cite{ZkbHandbuchDerItArchitektur} Kapitel 12.4.2 -
    \begin{itshape}Architektur beim Einsatz von
    Application-Servern\end{itshape}, Seite 147}
    
    % Kapitel 12.4.2, Architektur beim Einsatz von Application-Servern,
    % Seite 148
    \item[KO-ZKB-25] Der ZKB Standard-J2EE-Application-Server ist der JBoss
    Application Server.
    \footnote{\cite{ZkbHandbuchDerItArchitektur} Kapitel 12.4.2 -
    \begin{itshape}Architektur beim Einsatz von
    Application-Servern\end{itshape}, Seite 148}
    
    % Kapitel 12.4.2, Architektur beim Einsatz von Application-Servern,
    % Seite 148
    \item[KO-ZKB-26] Der J2EE-Application-Server wird in der ZSI eingesetzt.
    \footnote{\cite{ZkbHandbuchDerItArchitektur} Kapitel 12.4.2 -
    \begin{itshape}Architektur beim Einsatz von
    Application-Servern\end{itshape}, Seite 148}
    
    % Kapitel 12.4.2, Architektur beim Einsatz von Application-Servern,
    % Seite 148
    \item[KO-ZKB-27] Die Serverplattform für den Einsatz von
    Web-Application-Servern für Internet-/Intranet-/Extranet-Applikationen ist Linux.
    \footnote{\cite{ZkbHandbuchDerItArchitektur} Kapitel 12.4.2 -
    \begin{itshape}Architektur beim Einsatz von
    Application-Servern\end{itshape}, Seite 148}
    
    % Kapitel 12.4.2, Architektur beim Einsatz von Application-Servern,
    % Seite 150
    \item[KO-ZKB-28] Die technischen Services wie Session Management, Load
    Balancing, Transaction Management und Instance Pooling werden vom J2EE Application
    Server zur Verfügung gestellt.
    \footnote{\cite{ZkbHandbuchDerItArchitektur} Kapitel 12.4.2 -
    \begin{itshape}Architektur beim Einsatz von
    Application-Servern\end{itshape}, Seite 150}
    
    % Kapitel 12.4.3, Einsatz von Enterprise Java Beans, Seite 150
    \item[KO-ZKB-29] Das Muster JSP/Servlets mit EJBs ist anzuwenden, wenn die
    Applika-tion eine umfangreiche, komplexe Business-Logik aufweist, die
    Business-Logik wiederverwendbar sein soll, mehrere unterschiedliche Clients
    (Browser (Ultra-Thin-)(HTML), Thin-(Java), Mobile, …) mit einer Business
    Logik bedient werden müssen, hohe Anforderungen an die Skalierbarkeit
    gestellt werden und ein lange Lebenszyklus der Applikation erwartet wird.
    \footnote{\cite{ZkbHandbuchDerItArchitektur} Kapitel 12.4.3 -
    \begin{itshape}Einsatz von Enterprise Java Beans\end{itshape}, Seite 150}
    
    % Kapitel 12.4.3, Einsatz von Enterprise Java Beans, Seite 150
    \item[KO-ZKB-30] Das Muster JSP/Servlets ohne EJBs ist anzuwenden, wenn die
    Applikation eine einfache Business-Logik aufweist, nur einen Client, zum
    Beispiel ein Browser (Ultra-Thin-)-Interface unterstützt, niedrige
    Anforderungen an die Skalierbarkeit stellt und nur eine vergleichsweise
    kurzer Lebenszyklus der Applikation erwartet wird.
    \footnote{\cite{ZkbHandbuchDerItArchitektur} Kapitel 12.4.3 -
    \begin{itshape}Einsatz von Enterprise Java Beans\end{itshape}, Seite 150}
    
    % Kapitel 12.9.2, Standards, Seite 171
    \item[KO-ZKB-31] In der ZKB werden folgende Standards für Web Services
    eingesetzt: SOAP, WSDL, W3C X Schema (XSD, WXS)
    \footnote{\cite{ZkbHandbuchDerItArchitektur} Kapitel 12.9.2 -
    \begin{itshape}Standards\end{itshape}, Seite 171}
    
    % Kapitel 13.1, Client/Server-Konzept, Seite 175
    \item[KO-ZKB-32] Alle Neuentwicklungen sind konsequent in
    Client-/Server-Komponen-ten aufzuteilen.
    \footnote{\cite{ZkbHandbuchDerItArchitektur} Kapitel 13.1 -
    \begin{itshape}Client/Server-Konzept\end{itshape}, Seite 175}
    
    % Kapitel 13.2, Anwendungsstruktur (MVC/Model, View, Controller), Seite 175
    \item[KO-ZKB-33] Die gewünschte Isolation der Systemteile wird durch
    eine Anwendungsstruktur mit Trennung in Model (Verarbeitung,
    Datenhaltung), View (Be-nutzeroberfläche) und Controller (Organisation,
    Ereignisvermittlung) erreicht.
    \footnote{\cite{ZkbHandbuchDerItArchitektur} Kapitel 13.2 -
    \begin{itshape}Anwendungsstruktur (MVC/Model, View,
    Controller)\end{itshape}, Seite 175}
    
    % Kapitel 13.9.7.2, Allgemeine Grundsätze, Seite 189
    \item[KO-ZKB-34] Jede Applikation ist dafür verantwortlich, dass diejenigen
    Daten, für die sie den Lead hat, validiert sind.
    \footnote{\cite{ZkbHandbuchDerItArchitektur} Kapitel 13.9.7.2 -
    \begin{itshape}Allgemeine Grundsätze\end{itshape}, Seite 189}
    
    % Kapitel 13.9.7.2, Allgemeine Grundsätze, Seite 189
    \item[KO-ZKB-35] Jede Applikation benutzt Datenvalidierung um
    sicherzustellen, dass sie die erhaltenen Daten verarbeiten kann und dass ihr Betrieb nicht
    gefährdet ist (Stabilität / Verfügbarkeit).
    \footnote{\cite{ZkbHandbuchDerItArchitektur} Kapitel 13.9.7.2 -
    \begin{itshape}Allgemeine Grundsätze\end{itshape}, Seite 189}
    
    % Kapitel 13.9.7.3.1, Datenvalidierung an der Benutzersczhnittstelle
    % Seite 191
    \item[KO-ZKB-36] Benutzereingaben werden so früh wie möglich validiert.
    \footnote{\cite{ZkbHandbuchDerItArchitektur} Kapitel 13.9.7.3.1 -
    \begin{itshape}Datenvalidierung an der Benutzersczhnittstelle\end{itshape},
    Seite 191}
    
    % Kapitel 13.10, Programmiersprachen und Entwicklungsumgebungen
    % Seite 192
    \item[KO-ZKB-37] Für die Entwicklung neuer Applikationen und beim neuen
    Design bestehender Applikationen wird Java eingesetzt.
    \footnote{\cite{ZkbHandbuchDerItArchitektur} Kapitel 13.10 -
    \begin{itshape}Programmiersprachen und Entwicklungsumgebungen\end{itshape},
    Seite 192}
    
    % Kapitel 13.10.4.8, Zusätzliche Java Klassenbibliotheken, Seite 195
    \item[KO-ZKB-38] Zusätzliche Java-Klassenbibliotheken, also solche, die
    nicht im JDK enthalten sind, werden nur in begründeten Ausnahmen eingesetzt.
    Normalerweise müssen solche Bibliotheken dem 100\%-Pure-Java-Grundsatz
    entsprechen. Ausnahmen können systemnahe Funktionen für Security und
    der-gleichen sein.
    \footnote{\cite{ZkbHandbuchDerItArchitektur} Kapitel 13.10.4.8 -
    \begin{itshape}Zusätzliche Java Klassenbibliotheken\end{itshape}, Seite 195}
    
    % Kapitel 13.11, Einsatz von .NET-basierten Applikationen, Seite 196
    \item[KO-ZKB-39] .NET-basierte Applikationen werden in der ZKB Informatik
    nicht ent-wickelt oder zur Entwicklung in Auftrag gegeben ausser
    Kleinapplikationen der Kategorie „Office“.
    \footnote{\cite{ZkbHandbuchDerItArchitektur} Kapitel 13.11 -
    \begin{itshape}Einsatz von .NET-basierten Applikationen\end{itshape}, Seite
    196}
    
    % Kapitel 13.13, Einsatz von Open Source Software, Seite 206
    \item[KO-ZKB-40] Die Nutzung von Open Source Software ist erlaubt.
    \footnote{\cite{ZkbHandbuchDerItArchitektur} Kapitel 13.13 -
    \begin{itshape}Einsatz von Open Source Software\end{itshape}, Seite 206}
    
    % Kapitel 13.13, Einsatz von Open Source Software, Seite 206
    \item[KO-ZKB-41] Open Source Software unterliegt denselben Kriterien wie
    kommer-zielle Software. Sie muss evaluiert, registriert, homologiert, intern
    supported und gepflegt werden.
    \footnote{\cite{ZkbHandbuchDerItArchitektur} Kapitel 13.13 -
    \begin{itshape}Einsatz von Open Source Software\end{itshape}, Seite 206}
    
    % Kapitel 13.13.1, Kriterien für die Evaluation von Open Source
    % Software, Seite 207
    \item[KO-ZKB-42] Produktiv eingesetzte Open Source Software muss durch
    angemessene Informationsquellen unterstützt sein. (Newsgroups,
    Mailinglists, FAQ-Listen, WebSites, User Groups)
    \footnote{\cite{ZkbHandbuchDerItArchitektur} Kapitel 13.13.1 -
    \begin{itshape}Kriterien für die Evaluation von Open Source
    Software\end{itshape}, Seite 207}
    
    % Kapitel 13.13.1, Kriterien für die Evaluation von Open Source
    % Software, Seite 207
    \item[KO-ZKB-43] Die Weiterentwicklung von produktiv eingesetzter Open
    Source Soft-ware ausserhalb der ZKB muss öffentlich einsehbar sein und aktiv verfolgt werden können.
    \footnote{\cite{ZkbHandbuchDerItArchitektur} Kapitel 14.14.1 -
    \begin{itshape}Kriterien für die Evaluation von Open Source
    Software\end{itshape}, Seite 207}
    
    % Kapitel 13.13.1, Kriterien für die Evaluation von Open Source
    % Software, Seite 207
    \item[KO-ZKB-44] Die Lizenzbedingungen einer Open Source Software müssen vor
    dem Einsatz geprüft werden und von der Zürcher Kantonalbank akzeptiert werden
    können.
    \footnote{\cite{ZkbHandbuchDerItArchitektur} Kapitel 13.13.1 -
    \begin{itshape}Kriterien für die Evaluation von Open Source
    Software\end{itshape}, Seite 207}
  \end{description}
  
  \subsubsection{Umsetzung theoretisch möglich}
    
  \subsection{Hirarchie der Anforderungen}
  
  \begin{itemize}
    \item Zugriffskontrolle (Authentifizierung/Authorisation/Rollenverwaltung)

Ein Webframework sollte EntwicklerInnen verschiedene Mechanismen bereitstellen,
um die Anwendung vor fremden und unerlaubten Zugriff schützen zu können.

Authentifizierung/Autorisierung: Das Ziel einer sicheren Webanwendung ist es,
bestimmte Bereiche einer Seite abzusichern und die Rechte aller BenutzerInnen
je nach Rolle einzuschränken. Üblicherweise werden im Vorfeld Rollen für die
verschiedenen Gruppen festgelegt. Anhand der Rolle wird der/die BenutzerIn für
die festgelegten Bereiche autorisiert.

Vertraulichkeit / Verschlüsselung: Sensible Daten, wie Passwörter und
Personendaten, müssen vor dem Zugriff und der Kenntnisnahme von Dritten
geschützt werden. Hierfür werden die Daten während der Übertragung und
teilweise in der Datenbank verschlüsselt. Für eine sichere Datenübertragung
werden meist Verschlüsselungsprotokolle wie SSL (Secure Sockets Layer), sowie
dessen Nachfolger TLS (Transport Layer Security) eingesetzt. Diese gelten als
relativ sicher und sind bei Transaktionen in Online Shops oder bei Banken
unverzichtbar. Nach der Datenübertragung muss der Betreiber dafür sorgen, dass
die sensiblen Informationen vertraulich gespeichert werden. Für die
Verschlüsselung von Daten innerhalb einer Datenbank, eignet sich zum Beispiel
der MD5-Algorithmus. Falls nun Dritte an diese Daten kommen, sind sie im ersten
Augenblick unbrauchbar, da an der Stelle des Passworts (‘tony0815’) eine
verschlüsselte Zeichenkette (‘7e716d0e702df05’) steht. Das Webframework soll
also für Kryptographie und SSL/HTTPS nützliche Funktionen bereitstellen, um die
Arbeit zu erleichtern.

    \item Form-Validierung

Server-Side Validation: Das Verarbeiten von Formularen bzw. die Handhabung von
BenutzerInneneingaben und -aktionen gehört zu den täglichen Aufgaben der
Webentwicklung. Das Webframework soll die Mögichkeit bieten, eingegebene Daten
einfach zu überprüfen. Dabei soll für jede Eigenschaft eines Datenmodells
(Model) ein Wertebereich definierbar sein, ob die Eingabe erforderlich ist und
eigene Erweiterungen unterstützen. Diese Aufgabe soll möglichst transparent
gelöst sein, d.h. die Programmierlogik ist minimal, es wird nur definiert, was
für die Eingabe erlaubt ist. Nach dem Senden der Daten wird alles überprüft und
ggf. entsprechende Fehlermeldungen zurückgegeben.

Client-Side Validation: Das Webframework soll neben dem server-seitigen
Validieren auch gleichzeitig einzelne Überprüfungen auf den Client auslagern.
D.h. dass gleich bei oder kurz nach der Eingabe Eingaben überprüft werden,
wobei die Logik auf dem Server implementiert ist. Bsp.: Ist der Anmeldename
schon vergeben. Dadurch werden Serverressourcen gespart und der/die BenutzerIn
hat ein direktes Feedback. Die Logik für server- und client-seitiges Validieren
ist im Idealfall nur einmal implementiert.

  \item Modulare Architektur

Eine Webanwendung stellt ein Zusammenspiel verschiedenster Internettechnologien
dar, die wiederum hinsichtlich ihrer Entwicklung einem stetigen Wandel
unterliegen. Die Herausforderung für die EntwicklerInnen ist es, die
Entwicklung der Technologien im Auge zu behalten um ggfs. Neuheiten oder
Änderungen im System anzupassen. Eine Webanwendung sollte daher in all ihren
Bestandteilen möglichst wartbar bleiben.

Kaum eine Software kommt in seinem Lebenszyklus ohne Optimierungen und
Anpassungen aus. Die Anforderungen können sich ändern (seitens der
AuftraggeberInnen oder der BenutzerInnen) oder die Software muss aus anderen
Gründen erweitert werden (neue Funktionen, Schnittstellen). Für die Umsetzung
ist es von großer Bedeutung, bestehende Softwarebausteine in den
Änderungsprozessen wiederzuverwenden und dann ggfs. zu erweitern. Gerade bei
größeren Projekten sollte die Anwendung modularisiert werden, damit einzelne
Komponenten wiederverwendbar und zudem austauschbar bleiben
(komponentenbasierte Entwicklung).

  \item Schnittstellen und Webservices

Interoperabilität beschreibt den Austausch von Informationen verschiedener
(Software-) Systeme und/oder die Bereitstellung von Ressourcen über Dienste
(Webservices). Die Grundlage für einen universellen Informationsaustausch
bietet meist XML, das zum Beschreiben der Informationen dient. Es gibt
verschiedene Technologien, mit denen ein Webservice umgesetzt werden kann -
REST, SOAP und RPC sind am weitesten verbreitet. Das Webframework sollte in
diesen Fällen Mechanismen und Funktionen zur Umsetzung von Schnittstellen und
Webservices bereitstellen.

  \item MVC-Entwurfsmuster

Besonders im Web hat sich das Model-View-Controller Entwurfsmuster als quasi
Standard-Architekturmuster für Webanwendungen etabliert und hält daher Einzug
bei den meisten Webframeworks. Es dient zur Strukturierung der Software in drei
Einheiten. Das Model (Datenmodell) enthält die Geschäftslogik - die
Informationen die dargestellt werden. Der Controller (Steuerung) ist die
Schnittstelle zwischen der View und dem Datenmodell. Er nimmt Benutzeraktionen
entgegen (z.B. Formulardaten) und leitet sie an ein bestimmtes Datenmodell
weiter. Der Controller führt dann Operationen wie speichern, ändern und löschen
auf dem Datenmodell (besser gesagt dem Objekt) aus. Die View ist die
Präsentationsschicht und stellt die Daten dar, die es vom Controller
entgegennimmt. Jedoch sollte eine View nicht ohne einen Controller neue Objekte
erzeugen oder speichern (Trennung von Logik und Darstellung). Mit dem
MVC-Entwurfsmuster können u.a. ProgrammiererInnen und DesignerInnen während der
Entwicklung unabhängig voneinander arbeiten.

  \item Testing

Wichtige Bestandteile einer Software sollten stets getestet werden, um jegliche
(Programmier-)Fehler vor dem Live-Betrieb auszuschließen. All zu oft ist für
Testing aber keine Zeit oder es stehen nicht genügend finanzielle Mittel zur
Verfügung. Allerdings sind Fehler im Live-Betrieb sowohl für die BenutzerInnen,
als auch für die EntwicklerInnen äußert ärgerlich. Testing ist mit test-driven
development (TDD), vor allem in der agilen Softwareentwicklung, ein fester
Bestandteil während der Projektentwicklung. Vor der Implementierung überprüft
der/die ProgrammiererInnen - mittels Unit-Tests - konsequent das Verhalten
jeglicher Komponenten. Gerade kritische Prozesse und Transaktionen (z.B. eine
Banküberweisung) sollten ausgiebig getestet werden.

  \item Internationalisierung und Lokalisierung

Viele Webanwendungen richten sich mittlerweile an ein internationales Publikum.
Dank der Offenheit und der weiten Verbreitung des Internets lassen sich dadurch
sehr einfach neue Zielgruppen (BenutzerInnen) erschließen. Jedoch gilt es,
einige Vorraussetzungen und Besonderheiten bei der Internationalisierung von
Webanwendungen zu beachten. Zunächst müssen sämtliche Texte übersetzt und
möglicherweise in neue Datenbanken ausgelagert werden. Hierbei ist zu
berücksichtigen, dass es länderspezifische Zeichensätze, Zahlen, Datum und
Währungswerte gibt. Hinzu kommt, dass unter Umständen auch spezielle Grafiken
neu erstellt werden müssen.

  \item Object Relational Mapping (ORM)

Die meisten Webframeworks unterstützen das objektorientierte
Programmierparadigma. Im Zusammenhang mit relationalen Datenbanken kommt es
allerdings zu grundlegenden Problemen, weil der Zustand und das Verhalten eines
Objekts nicht in einer relationalen Datenbank gespeichert werden kann. Dies ist
auf die beiden widersprüchlichen Konzepte von objektorientierter Programmierung
(OOP) und relationalen Datenbankmanagementsystemen (RDMS) zurückzuführen. Im
Gegensatz zu objektorientierten Datenbanken, werden beim ORM die Tabellen aus
der Datenbank als Klassen abgebildet (gemappt). Durch die Klassenabbildung wird
für den/die ProgrammiererIn wieder die gewohnte OOP-Umgebung geschaffen. Wenn
ein Objekt erstellt oder geändert wird, ist der Mapper verantwortlich, um diese
Informationen in der Datenbank zu speichern und auch ggfs. wieder zu löschen.
ORM bildet somit eine Schnittstelle zwischen OOP und relationalen Datenbanken.

Unterstützung von Transaktionen: Bei der Speicherung von Informationen in
mehreren Datenbanktabellen muss sicher gestellt sein, dass entweder alle oder
keine Informationen gespeichert werden.

  \item Scaffolding / Rapid Prototyping

Mit Scaffolding ist das automatische Generien von den sogenannten CRUD-Pages
(Create, Read, Update, Delete) gemeint. In Verbindung mit dem
MVC-Entwurfsmuster, werden sowohl Controller und die dazugehörigen Views
erzeugt. Die ProgrammiererInnen können somit ohne großen Aufwand Models in den
Views testen.

Scaffolding und das dadurch verstandene Rapid Prototyping ist ein wesentlicher
Aspekt der agilen Softwareentwicklung (zum Beispiel in Verwendung mit der
Scrum-Methodik). Gerade zu Beginn eines Projekts eignet sich das Rapid
Prototyping, da Änderungen in den Models umgehend in die Views eingebunden
werden können.

  \item Caching

Neben der Übertragunsgeschwindigkeit gibt es eine Menge anderer Faktoren, die
für eine leistungsstarke Webanwendung entscheidend sind. Ein Aspekt ist das
Caching - das Zwischenspeichern von Daten, die häufig verwendet bzw. aufgerufen
werden.

Das Caching kann meist auf verschiedenen Ebenen implementiert werden: Auf dem
Client, auf dem Webserver und auf der Datenbankebene. Dabei soll das
Webframework diese Mechanismen unterstützen und es ermöglichen, diese einfach
zu aktivieren und zu kalibrieren.  

  \item View-Engine

View Engines werden eingesetzt, um das Arbeiten mit den Views zu erleichtern.
Sie unterstützen vor allem das Templating - das Erstellen von Vorlagen. Durch
typisierte Views wird eine Webanwendung robuster, weil weniger fehleranfällig;
durch partial Views (oft auch nur als partials bezeichnet) werden einzelne
Elemente oder Bereiche der Benutzeroberfläche wiederverwendbar gemacht. Diese
Vorlagen können von mehreren Views genutzt werden. Dadurch wird deutlich
weniger redundanter Code erstellt. Dies ist ein wichtiger Aspekt in Bezug auf
das Don’t repeat yourself (DRY) Prinzip.

  \item Dokumentation

Die Umsetzung einer Webanwendung erfolgt meist über ein Webframework, das die
EntwicklerInnen nach ihrem Ermessen aussuchen. Eine Webanwendung ist aus
EntwicklerInnensicht eine Zusammenfassung verschiedenster Technologien
(Webframeworks, Bibliotheken, Schnittstellen). Über das Application Programming
Interface (API) haben ProgrammiererInnen Zugriff auf die verfügbaren
Funktionen. Nicht selten erstreckt sich die Dokumentation einer API über
mehrere hundert Seiten. Für ProgrammiererInnen ist es daher von enormer
Bedeutung, dass die Bibliotheken gut strukturiert und verständlich beschrieben
sind. Schlecht oder gar nicht dokumentierte Technologien erhöhen die
Fehlerquote und sind oft ausschlaggebend für einen nicht flüssigen Workflow.

  \item Community

Die beteiligten Personen in den Foren, Mailing-Listen oder Wikis bilden im
Zusammenhang mit Webframeworks die Community. Es findet dabei ein
Wissensaustausch statt, der weit über die standard-Dokumentation hinaus geht.
Die NutzerInnen helfen sich gegenseitig bei Problemen und Fehlern und oft
hinterlassen sie mit ihren Einträgen wiederum einen Lösungsansatz, der
zukünftig von anderen wieder aufgegriffen werden kann. Es ist daher wichtig,
dass den AnwenderInnen eine Möglichkeit geboten wird, sich auszutauschen -
gemeinsam Fehlermeldungen zu deuten und Lösungsansätze zu entwickeln.

  \item IDE-Unterstützung

In der Softwareentwicklung ist die IDE das Basiswerkzeug für die
ProgrammiererInnen. Die Entwicklungsumgebung stellt den EntwicklerInnen
verschiedene Komponenten zur Verfügung, wie Editor, Compiler, Linker oder
Debugger. Hinzu kommen mit Syntaxhighlighting, Refactoring und
Code-Formatierung weitere wichtige Funktionen, die die EntwicklerInnen in
vielerlei Hinsicht enorm unterstützen.

  \item Kosten für Entwicklungswerkzeuge

Je nach verfügbaren finanziellen Mitteln spielen die Kosten von
Entwicklungswerkzeugen und Technologien durchaus eine Rolle. Bei beiden
Faktoren hat man meist die Wahl zwischen kostenlosen (OpenSource) und
kommerziellen Produkten. Je nach Webanwendung müssen somit Lizenzgebühren für
die Entwicklungswerkzeuge, Server zum Ausführen der Anwendung und
Datenbankserver berücksichtigt werden. Dabei ist es wichtig die richtige
Mischung verschiedener Komponenten zu finden.
  
  \item Eignung für agile Entwicklung

Die herkömmlichen Methoden der Software-Entwicklung werden heute oft durch neue
agile Methoden, wie Extreme Programming oder Scrum abgelöst. Sie fokussieren
auf das Wesentliche und stehen für deutlich mehr Flexibilität in der
Entwicklungsphase als konventionelle Methoden. Verschiedene Technologien
unterstützen die agilen Methoden. Refactoring, Testing spielt dabei eine
wichtige Rolle und soll unterstützt werden.

  \item Lernkurve für EntwicklerInnen

Zur Umsetzung einer komplexen Webanwendung wird den EntwicklerInnen ein Wissen
über verschiedenste Bereiche der Softwareentwicklung abverlangt.
Glücklicherweise gibt es für die Webentwicklung keinen einheitlichen Standard,
der festlegt, wie eine Webanwendung entwickelt werden muss. Das Internet bietet
für jeden Bereich eine Auswahl unterschiedlicher Technologien an. Daher müssen
vor der Entwicklung mehrere Entscheidungen getroffen werden - hinsichtlich
Programmiersprache, Javascript-Framework oder Datenbankserver. Es werden daher
oft Technologien gewählt, die leicht zu erlernen und verstehen sind.

Wichtig ist, dass man einen schnellen Einstieg bekommt und Erfolge bald
sichtbar werden, um die Motivation der EntwicklerInnen zu erhöhen.

  \item AJAX-Unterstützung

Durch das Aufkommen von Javascript-Bibliotheken wie jQuery und Prototype haben
sich vollkommen neue Möglichkeiten eröffnet mit Javascript auf dem Client zu
arbeiten und mit AJAX wurde die Kommunikation zwischen Server und Client im Web
revolutioniert. Die direkte Unterstützung eines Javascript-Frameworks ist für
ein Webframework sinnvoll und erwünscht.

Darüber hinaus sollte die unterstützte Bibliothek lose gekoppelt sein, und
damit austauschbar. Sog. unobtrusive Javascript, bei dem auch bei
abgeschaltetem Javascript die Anwendung funktioniert, ist auch eine
Anforderung.
  \end{itemize}
  
  \section{Proof of Concept - Umsetzung durch einen Prototypen}
  
  \chapter{Integration in die ZKB Infrastruktur möglich?}
  
  \chapter{Empfehlung für ein Java Web Framework}
  
  \chapter{Reflektion}
  
  Rückblickend war es schade, dass ich diese Arbeit zu diesem Zeitpunkt
  schreiben musste, da mein Sohn Linus im Dezember zur Welt gekommen ist, und
  ich insgesamt über 100 Stunden vor meinen Rechnern verbracht habe, um alles
  zu vollenden. Natürlich liegt die Schuld voll und ganz bei mir selber, da ich
  viel früher mit der Semesterarbeit hätte beginnen können.
  
  Schlussendlich war die Arbeit in meinen Auge ein voller Erfolg, da ich alle
  Ziele erreicht und viel gelernt habe, und dank der unkomplizierten Art und Weise von
  Beat Seeliger, in der Betreuerrolle, mein Bestes zum Vorschein gebracht habe.
  Ich bin glücklich, dass es vorbei ist, und bereit für die Diplomarbeit.
  
  % ===========================================================================
  % Anhang beginns here
  % ===========================================================================
  
  \appendix
  
%  \chapter{Ein Testanhang}
%  
%  Im Anhang kann auf Implementierungsaspekte wie Datenbankschemata
%  oder Programmcode eingegangen werden.
  
  % ===========================================================================
  % Abkürzungsverzeichnis beginns here
  % ===========================================================================
  
  \chapter{Abkürzungsverzeichnis}
  \begin{acronym}
    \setlength{\itemsep}{-\parsep}
    \acro{OO}{Objektorientierung}
    \acro{ABC}{Alphabeth}
    \acro{URL}{Uniform Resource Locator}
    \acro{RIA}{Rich Internet Application}
    \acro{Ajax}{Asynchronous JavaScript and XML}
  \end{acronym}
  
  % ===========================================================================
  % Abbildungsverzeichnis beginns here
  % ===========================================================================
  
  % Abbildungsverzeichnis
  \listoffigures
  
  % ===========================================================================
  % Tabellenverzeichnis beginns here
  % ===========================================================================
  
  % Tabellenverzeichnis
  \listoftables
  
  % ===========================================================================
  % Literaturverzeichnis beginns here
  % ===========================================================================
  
  % verwendet alpha
  \bibliographystyle{alpha}
  % verwendet Literaturverzeichnis.bib
  % \renewcommand\bibname{Literaturverzeichnis} % Titel überschreiben
  \cleardoublepage
  \bibliography{Literaturverzeichnis}

\end{document}