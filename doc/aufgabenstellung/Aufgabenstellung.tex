%
%  Antrag, Aufgabenstellung Semesterarbeit
%
%  Created by Roman Wuersch on 2010-12-04.
%
% =============================================================================
% Documentdefinition  beginns here
% =============================================================================

\documentclass[listof=totoc,bibliography=totoc]{scrreprt}
\usepackage[ngerman]{babel}

\usepackage{tocbasic}

% Use utf-8 encoding for foreign characters
\usepackage[utf8]{inputenc}
% \usepackage[applemac]{inputenc}

% Setup for fullpage use
\usepackage{fullpage}

% Running Headers and footers
%\usepackage{fancyhdr}

% Multipart figures
%\usepackage{subfigure}

% More symbols
%\usepackage{amsmath}
%\usepackage{amssymb}
%\usepackage{latexsym}

\makeatletter

\newcommand{\nobreakchap}{%
\renewcommand\chapter{%
\par\global\@topnum\z@
\@afterindentfalse
\secdef\@chapter\@schapter}
}

\newcommand{\normalchap}{%
\renewcommand\chapter{%
\if@openright\cleardoublepage\else\clearpage\fi
\thispagestyle{\chapterpagestyle}%
\global\@topnum\z@
\@afterindentfalse
\secdef\@chapter\@schapter}
}

\makeatother


% Surround parts of graphics with box
\usepackage{boxedminipage}

% Package for including code in the document
\usepackage{listings}

% If you want to generate a toc for each chapter (use with book)
\usepackage{minitoc}

% This is now the recommended way for checking for PDFLaTeX:
\usepackage{ifpdf}

\ifpdf
    \usepackage[pdftex]{graphicx}
\else
    \usepackage{graphicx}
\fi

% =============================================================================
% Documenttext beginns here
% =============================================================================

\title{Aufgabenstellung\\
    Diplomarbeit in Informatik}
    
\author{Studierender - Roman Würsch\\
    Projektbetreuer - Beat Seeliger\\
    Experte - tbd\\
    \\
    HSZ-T - Technische Hochschule Zürich}
    
\date{14. März 2010}

\begin{document}

    \ifpdf
        \DeclareGraphicsExtensions{.pdf, .jpg, .tif}
    \else
        \DeclareGraphicsExtensions{.eps, .jpg}
    \fi

    \maketitle

    \pagenumbering{arabic}

    % \tableofcontents
    
    \nobreakchap
    
    \chapter{Titel}
    Evaluation eines Java Web Frameworks zur Ablösung bestehender Java Swing
    Applikationen

    \chapter{Thema}
    Ablösung bestehender Java Swing Applikationen durch den Einsatz eines Java
    Web Frameworks
    
    \chapter{Ausgangslage}
    Die Zürcher Kantonalbank hat viele Applikationen welche als Client
    Applikationen in Swing implementiert sind. Dabei ist das Deployment der
    Clients und die Ausbreitung entsprechender Patches, innerhalb der Zürcher
    Kantonalbank, mit einem Mehraufwand verbunden. Client Applikationen, die
    einem Kunden zur Verfügug gestellt werden, sind an eine spezifische Java
    Version gebunden. Bei der Integration in die IT-Landschaft beim Kunden, 
    kann das zur Verzögerung durch einen erhöhten Testaufwand führen. In der
    Annahme, dass ein Webbrowser in der Zürcher Kantonalbank und bei deren
    Kunden eingesetzt wird, macht der Einsatz einer Weblösung Sinn.

    \chapter{Ziel der Arbeit}
    Es sollen bestehende Java Swing Applikationen der Zürcher Kantonalbank
    analysiert werden. In diesen Applikationen sollen die gemeinsame Muster
    genutzter Swingkomponenten erkannt und kategorisiert werden. Java Web
    Frameworks, welche sich am Markt etabliert haben, sollen auf einen
    möglichen Einsatz geprüft werden. Es soll geprüft werden, ob mit den
    jeweiligen Frameworks die genutzten Swingkomponenten equivalent umgesetzt
    werden können. Zudem soll geprüft werden, ob eine Integration in die
    bestehende IT Infrastruktur der Zürcher Kantonalbank möglich ist.
    
    \normalchap
    
    \chapter{Aufgabenstellung}
    Folgende Aufgaben sollen vom Studierenden während der Diplomarbeit
    durchgeführt werden
    
    \begin{itemize}
        \item Analyse bestehender Java Swing Applikationen der Zürcher
        Kantonalbank
        \item Erkennen und Kategorisieren der verwendeten Swingkomponenten
        \item Evaluation von Java Web Frameworks, welche sich am Markt
        etabliert haben
        \item Prüfen, ob eine Integration der evaluierten Java Web Frameworks,
        welche für eine Umsetzung geeignet sind, in der bestehenden IT
        Infrastruktur der Zürcher Kantonalbank möglich ist.
        \item Prüfen, ob eine Implementierung der erkannten Swingkomponenten in
        den evaluierten Java Web Frameworks möglich ist.
    \end{itemize}

    \nobreakchap

    \chapter{Erwartete Resultate}
    Der Studierende soll dem Auftraggeber ein Dokument erstellen, das folgendes
    beinhaltet:

    \begin{itemize}    
        \item Analyse bestehender Java Swing Applikationen der Zürcher
        Kantonalbank
        \item Kategorisierung von verwendeten Swingkomponenten
        \item Eine Auflistung von etablierten Java Web Frameworks
        \item Analyse, ob eine Integration der Java Web Frameworks, in der
        bestehenden IT Infrastruktur der Zürcher Kantonalbank möglich ist.
        \item Analyse der Java Web Frameworks, ob eine Implementierung, der
        erkannten Swingkomponenten, möglich ist.
        \item Proof of concept. Es soll anhand eines Prototypen gezeigt
        werden, dass die Implementierung möglich ist.
        \item Eine Empfehlung für ein Java Web Framework
    \end{itemize}
    
    \normalchap
    
	\chapter{Abgrenzung}
	Folgende Punkte werden abgegrenzt:
	
	\section{Formell}
	\begin{itemize}
	  \item Die Analysen beschränken sich auf Recherchen im Internet und
	  Büchern
	  \item Umfragen, Erhebungen sowie Feldstudien werden nicht durchgeführt
	\end{itemize}
	
	\section{Inhaltlich}
	\begin{itemize}
	  \item Die Auswahl, welche Java Swing Applikationen analysiert werden, soll
	  wärend der Arbeit durchgeführt werden.
	  \item Die Auswahl, welche Java Web Frameworks geprüft werden, soll wärend der
	  Arbeit durchgeführt werden.
	\end{itemize}
	
	\nobreakchap
	
	\chapter{Geplante Termine}
    Die Termine können zum Zeitpunkt des Antrages noch nicht definitiv 
    festgelegt werden. Sofern jedoch die Planung eingehalten werden kann und 
    freie Termine zur Verfügung stehen, sollten die Termine innerhalb der 
    angegebenen Monate liegen.

    \begin{tabbing}
        \hspace*{4cm}\= \kill
    	Kick-Off:               \> Ende März 2011 \\
    	Review:                 \> Ende April 2011 \\
    	Abgabe:                 \> Anfangs Juni 2011 \\
    	Schlusspräsentation:    \> Mitte Juni 2011 \\
    \end{tabbing}

    \normalchap

    \chapter{Genehmigung}
    Der Studierende, sein Projektbetreuer und der Studiengangsleiter 
    Informatik erklären sich mit der Aufgabenstellung einverstanden und geben 
    die Arbeit frei zur Erfassung im Einschreibesystem der Hochschule für 
    Technik Zürich.

    \begin{tabbing}
        \hspace*{10cm}\= \kill
    	Roman Würsch, Studierender \> Beat Seeliger, Projektbetreuer \\
    	\\
    	\\
        \line(1,0){150} \> \line(1,0){150} \\
        \\
        \\
    	Dr. Olaf Stern, Studiengangsleiter Informatik \\
    	\\
    	\\
        \line(1,0){150}
    \end{tabbing}    
\end{document}