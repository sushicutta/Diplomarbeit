%
%  Antrag, Aufgabenstellung Semesterarbeit
%
%  Created by Roman Wuersch on 2010-12-04.
%
% =============================================================================
% Documentdefinition  beginns here
% =============================================================================

\documentclass[listof=totoc,bibliography=totoc]{scrreprt}
\usepackage[ngerman]{babel}

\usepackage{tocbasic}

% Use utf-8 encoding for foreign characters
\usepackage[utf8]{inputenc}
% \usepackage[applemac]{inputenc}

% Setup for fullpage use
\usepackage{fullpage}

% Running Headers and footers
%\usepackage{fancyhdr}

% Multipart figures
%\usepackage{subfigure}

% More symbols
%\usepackage{amsmath}
%\usepackage{amssymb}
%\usepackage{latexsym}

% Surround parts of graphics with box
\usepackage{boxedminipage}

% Package for including code in the document
\usepackage{listings}

% If you want to generate a toc for each chapter (use with book)
\usepackage{minitoc}

% This is now the recommended way for checking for PDFLaTeX:
\usepackage{ifpdf}

\ifpdf
    \usepackage[pdftex]{graphicx}
\else
    \usepackage{graphicx}
\fi

% =============================================================================
% Documenttext beginns here
% =============================================================================

\title{Aufgabenstellung\\
    Diplomarbeit in Informatik}
    
\author{Studierender - Roman Würsch\\
    Projektbetreuer - Beat Seeliger\\
    \\
    HSZ-T - Technische Hochschule Zürich}
    
\date{10. März 2010}

\begin{document}

    \ifpdf
        \DeclareGraphicsExtensions{.pdf, .jpg, .tif}
    \else
        \DeclareGraphicsExtensions{.eps, .jpg}
    \fi

    \maketitle

    \pagenumbering{arabic}

    % \tableofcontents

    \chapter{Aufgabenstellung Diplomarbeit}

    \section{Thema}
    Evaluation eines Java Web Frameworks zur Ablösung bestehender Java Swing
    Applikationen
    
    \section{Ausgangslage}
    Die Zürcher Kantonalbank hat viele Applikationen welche als Client
    Applikationen in Swing implementiert sind. Dabei ist das Deployment der
    Clients und die Ausbreitung entsprechender Patches, innerhalb der Zürcher
    Kantonalbank, mit einem Mehraufwand verbunden. Client Applikationen, die
    einem Kunden zur Verfügug gestellt werden, sind an eine spezifische Java
    Version gebunden, was bei der Integration in die IT-Landschaft der Kunden,
    zu Verzögerungen durch einen erhöhten Testaufwand führen kann. In der
    Annahme, dass ein Webbrowser in der Zürcher Kantonalbank und bei deren
    Kunden eingesetzt wird, macht der Einsatz einer Weblösung sinn.

    \section{Ziel der Arbeit}
    Es sollen bestehende Java Swing Applikationen der Zürcher Kantonalbank
    analysiert werden. In diesen Applikationen sollen die Archetypen genutzter
    Swingkomponenten erkannt und kategorisiert werden. Java Web Frameworks,
    welche sich am Markt etabliert haben, sollen auf einen möglichen Einsatz
    geprüft werden. Es soll geprüft werden, ob mit dem jeweiligen Framework die
    Archetypen genutzter Swingkomponenten equivalent umgesetzt werden können.
    Zudem soll geprüft werden, ob eine Integration in die bestehende IT
    Infrastruktur der Zürcher Kantonalbank möglich ist.
    
    \section{Aufgabenstellung}
    Folgende Aufgaben sollen vom Studierenden während der Diplomarbeit
    durchgeführt werden
    
    \begin{itemize}
        \item Analyse bestehender Java Swing Applikationen der Zürcher
        Kantonalbank
        \item Erkennen und Kategorisieren der Archetypen der verwendeten
        Swingkomponenten
        \item Evaluation von Java Web Frameworks, welche sich am Markt
        etabliert haben
        \item Prüfen, ob eine Implementierung der erkannten Archetypen in den
        evaluierten Java Web Frameworks möglich ist.
        \item Prüfen, ob eine Integration der evaluierten Java Web Frameworks,
        welche für eine Umsetzung geeignet sind, in der bestehenden IT
        Infrastruktur der Zürcher Kantonalbank möglich ist.
    \end{itemize}

    \section{Erwartete Resultate}
    Der Studierende soll dem Auftraggeber ein Dokument erstellen, das folgendes
    beinhaltet:

    \begin{itemize}    
        \item Analyse bestehender Java Swing Applikationen der Zürcher
        Kantonalbank
        \item Archetypen von verwendeten Swingkomponenten
        \item Eine Auflistung von Java Web Frameworks, welche sich am
        Markt etabliert haben
        \item Analyse der Java Web Frameworks, ob eine Implementierung, der
        erkannten Archetypen von verwendeten Swingkomponenten, möglich ist.
        \item Analyse, ob eine Integration der evaluierten Java Web Frameworks,
        welche für eine Umsetzung geeignet sind, in der bestehenden IT
        Infrastruktur der Zürcher Kantonalbank möglich ist.
        \item Es soll eine Empfehlung für ein Java Web Framework ausgesprochen
        werden
    \end{itemize}

    \section{Geplante Termine}
    Die Termine können zum Zeitpunkt des Antrages noch nicht definitiv 
    festgelegt werden. Sofern jedoch die Planung eingehalten werden kann und 
    freie Termine zur Verfügung stehen, sollten die Termine innerhalb der 
    angegebenen Monate liegen.

    \begin{tabbing}
        \hspace*{4cm}\= \kill
    	Kick-Off:               \> Ende März 2011 \\
    	Review:                 \> Ende April 2011 \\
    	Abgabe:                 \> Anfangs Juni 2011 \\
    	Schlusspräsentation:    \> Mitte Juni 2011 \\
    \end{tabbing}

    \section{Genehmigung}
    Der Studierende, sein Projektbetreuer und der Studiengangsleiter 
    Informatik erklären sich mit der Aufgabenstellung einverstanden und geben 
    die Arbeit frei zur Erfassung im Einschreibesystem der Hochschule für 
    Technik Zürich.

    \begin{tabbing}
        \hspace*{10cm}\= \kill
    	Roman Würsch, Studierender \> Beat Seeliger, Projektbetreuer \\
    	\\
    	\\
        \line(1,0){150} \> \line(1,0){150} \\
        \\
        \\
    	Dr. Olaf Stern, Studiengangsleiter Informatik \\
    	\\
    	\\
        \line(1,0){150}
    \end{tabbing}    
\end{document}